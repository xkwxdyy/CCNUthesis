% 交终稿的时候需要单面打印,设置 oneside,奇偶页页面设置相同
\documentclass[oneside]{CCNUthesis}

% 答辩打印可以可以双面打印,设置 twoside,奇偶页页面设置不同
% \documentclass[twoside, draft]{CCNUthesis}


\ccnusetup{
  % 个人信息
  info = {
    % 主标题,会自动换行
    % 如果换行点不满意,可以用 \\ 手动换行
    title = {
      中文主标题中文主标题中文主标题中文主标题中文主标题中文主标题中文主标题中文主标题
    },
    % 副标题,如果有的话取消下面三行代码的注释并且填写内容
    subtitle = {
      这是一个副标题
    },
    % 论文的英文标题
    title* = {
      Thesis Title
    }, 
    % 学院
    department = {数学与统计学学院},
    % department = {数学},
    % 专业
    major = {数学与应用数学(试验)},
    % major = {数学},
    % 年级
    level = {2018级},
    % 姓名
    author = {你的姓名},
    % 学号
    student-id = {学号},
    % 指导教师
    supervisor = {xxx \quad 教授},
    % 论文中文关键词,用英文“,”隔开
    keywords = {
      关键词1,
      关键词2,
      关键词3
    },
    % 论文英文关键词,用英文“,”隔开
    keywords* = {
      keyword1,
      keyword2,
      keyword3
    },
    % 如有需要可以手动调整封面底的年和月,否则默认为「编译时」的年和月
    % year    = {2022},
    % month = {4}
  },
  style = {
    font = times,
      % 西文字体
      % 允许选项
      % font = newtx|times|stixtwo|xits|tg|none
    cjk-font = fandol,
      % 中文字体
      % 允许选项:
      %   cjk-font = adobe|fandol|founder|mac|sinotype|sourcehan|windows|none
      % 注意:
      %   1. 中文字体设置高度依赖于系统。各系统建议方案:
      %        windows:cjk-font = windows
      %        mac:    cjk-font = mac
      %        linux:  cjk-font = fandol(默认值)
      %   2. 除 fandol 和 sourcehan 外,其余字体均为商用字体,请注意版权问题
      %   3. 但 fandol 字体缺字比较严重,而 sourcehan 没有配备楷体和仿宋体
    caption-labelstyle = arabic,
      % 图表的标题 label 计数样式
      % 允许选项:
      %    arabic|hyphen
      %      arabic:样式为图1,图2,表1,表2...,并且跨 chapter 连续编号,即 上一个 chapter 的图编号若为4,下一个 chapter 的第一个图编号为 5
      %      hyphen:样式为图1-1,图2-1,表1-1...
      %      x-y 的 x 为 chapter 值,y 为图表的计数器值,新的 chapter 中 y 会清零从新计数
    caption-labelseperator = {colon},
      % 图表标题 label 和 标题内容 之内的分隔符
      % 允许选项:
      %    colon|space
      %      colon 表示 「:␣」,即一个西文冒号加一个空格
      %      space 表示 「␣␣」,即两个空格
    bibstyle = {ccnu-author-year},
      % bibstyle 表示参考文献的格式
      % 允许的选项:
      %   ccnu|gb7714-2015
      %   ccnu 表示按照学校给的标准
      %   gb7714-2015表示按照国标
    bib-resource = {CCNUthesis-main.bib},
      % 参考文献数据源,需要加bib后缀
    fullwidth-stop = catcode,
      % 是否把全角实心句点 “.” 作为默认的句号形状
      % 即正文中输入“。” 最终编译效果为“. ”
      % 一般科技类文章需要替换,防止“. ”与“。”混淆
      % 允许选项:
      %   fullwidth-stop = catcode|mapping|false
      % 说明:
      %   catcode   显式的 “。” 会被替换为 “.”(e.g. 不包括用宏定义保存的 “。”)
      %   mapping   所有的 “。” 会被替换为 “.”(使用 LuaLaTeX 编译则无效)
      %   false     不进行替换
  },
}



%%%%% 需要的宏包可以在此处自行调用 %%%%%



% 需要的命令环境可以自行定义
\newcommand{\upe}{\mathrm{e}}        % 直立的e,用于表示常量,如自然常数      
\newcommand{\upi}{\mathrm{i}}        % 直立的i,用于表示常量,如虚数单位
\newcommand{\hilbertH}{\mathcal{H}}



\begin{document}


% \frontmatter 开启论文前置部分
% 前置部分包含目录、中英文摘要以及符号表等
\frontmatter

% 目录
\tableofcontents


% 摘要
\newpage
\pagestyle{plain}
\setcounter{page}{1}
%����ժҪ
\zihao{-4}
\textbf{����ժҪ}:
\addcontentsline{toc}{section}{\protect����ժҪ}
  \myabstract

\textbf{�ؼ���:} \mykeywords
\addcontentsline{toc}{section}{\protect �ؼ���}
%Ӣ��ժҪ
\bigskip

\textbf{Title}: \mytitle
\addcontentsline{toc}{section}{\protect Title}
\medskip

\textbf{Abstract}:
\addcontentsline{toc}{section}{\protect Abstract}
 \myabstracten

\textbf{Keywords:} \mykeywordsen


\addcontentsline{toc}{section}{\protect Keywords}
\newpage



% 符号表,不需要的话将下面的代码注释掉即可
% 符号表

% 语法与 LaTeX 表格一致:列用 & 区分,行用 \\ 区分
% 如需修改格式,可以使用可选参数:
%   \begin{notation}[ll]
%     $x$ & 坐标 \\
%     $p$ & 动量
%   \end{notation}
% 可选参数与 LaTeX 标准表格的列格式说明语法一致
% 这里的 “ll” 表示两列均为自动宽度,并且左对齐


\begin{notation}[ll]
  $x$                  & 坐标        \\
  $p$                  & 动量        \\
  $\psi(x)$            & 波函数      \\
\end{notation}



% \mainmatter 进入论文主体部分
% 主体部分是论文的核心
\mainmatter

% 主体采用多文件编译的方式
% 即把每一章放进一个单独的 tex 文件里,并在这里用 \input 导入
% 例如 \chapter{引言}

\section{研究背景}
\begin{figure}[tbh]
  \centering
  \includegraphics[width = 5cm]{example-image-a}
  \caption{测试}
\end{figure}
\begin{table}[tbh]
  \caption{测试}
  \centering
  \begin{tblr}{|c|c|}
    11 & 22 \\
    33 & 44 
  \end{tblr}
\end{table}



\subsection{前人工作}


测试 \parencite{邱泽奇建构与分化}
\section{研究背景}

% 表示插入 main 所在目录中的 body 目录下的 chapter1.tex 文件


% !TeX root = ../main.tex

\chapter{常用命令环境示例}

此章用于展示一些常用的命令环境的效果,用户在具体写论文过程中可以用来参考效果,如果不需要此章出现在正文中,只需要将主文件 \verb|main.tex| 文件中的 \verb|% !TeX root = ../main.tex

\chapter{常用命令环境示例}

此章用于展示一些常用的命令环境的效果,用户在具体写论文过程中可以用来参考效果,如果不需要此章出现在正文中,只需要将主文件 \verb|main.tex| 文件中的 \verb|% !TeX root = ../main.tex

\chapter{常用命令环境示例}

此章用于展示一些常用的命令环境的效果,用户在具体写论文过程中可以用来参考效果,如果不需要此章出现在正文中,只需要将主文件 \verb|main.tex| 文件中的 \verb|\input{./body/chapter0.tex}| 代码注释掉即可(不建议删除,因为随时可以取消注释查看效果)。


\section{列表环境}


如果要修改 \verb|enumerate| 环境的 label 样式的话:

\begin{enumerate}
  \item 第一项
  \item 第二项
  \item 第三项
  \item 第四项
\end{enumerate}

\begin{enumerate}[1)]
  \item 第一项
  \item 第二项
  \item 第三项
  \item 第四项
\end{enumerate}

\begin{enumerate}[a.]
  \item 第一项
  \item 第二项
  \item 第三项
  \item 第四项
\end{enumerate}

\begin{enumerate}[(A)]
  \item 第一项
  \item 第二项
  \item 第三项
  \item 第四项
\end{enumerate}

test
\begin{enumerate}[(i)]
  \item 第一项
  \item 第二项
  \item 第三项
  \item 第四项
\end{enumerate}

\begin{enumerate}[I]
  \item 第一项
  \item 第二项
  \item 第三项
  \item 第四项
\end{enumerate}

\begin{enumerate}[label = \textbf{断言} \Alph*]
  \item 一般来说使用断言,推荐使用 claim 环境
  \item 但是如果真要有一些断言的层级分化的话
  \item 可以考虑用 enumerate 环境的 label 选项
\end{enumerate}

\begin{enumerate}[\textbf{断言} A]
  \item 1
  \item 2
\end{enumerate}


\section{已定义好的一些数学定理环境}

定理环境内的括号,不管是中文还是西文括号,都不会出现倾斜,不需要像旧模板一样需要用手动用 \verb|\textit| 调整
\begin{definition}[测度]
  (参见文献xxx) 这是一段文字 $E = m c^2$  (中文括号)和 (西文括号)
\end{definition}

\begin{theorem}
  这是一段文字 $E = m c^2$
\end{theorem}


\begin{proof}
  这是一段文字 $E = m c^2$
\end{proof}

\begin{proof}[定理xx的证明]
  这是一段文字 $E = m c^2$
\end{proof}

\begin{example}
  这是一段文字 $E = m c^2$
\end{example}

\begin{property}
  这是一段文字 $E = m c^2$
\end{property}

\begin{proposition}
  这是一段文字 $E = m c^2$
\end{proposition}

\begin{corollary}
  这是一段文字 $E = m c^2$
\end{corollary}

\begin{lemma}
  这是一段文字 $E = m c^2$
\end{lemma}

\begin{axiom}
  这是一段文字 $E = m c^2$
\end{axiom}

\begin{counterexample}
  这是一段文字 $E = m c^2$
\end{counterexample}

\begin{conjecture}
  这是一段文字 $E = m c^2$
\end{conjecture}

\begin{question}
  这是一段文字 $E = m c^2$
\end{question}

\begin{claim}
  这是一段文字 $E = m c^2$
\end{claim}

\begin{remark}
  这是一段文字 $E = m c^2$
\end{remark}

\begin{theorem}[Cauchy]\label{thm:test}
  这是一个定理
  \begin{equation}\label{eq:test1}
    a^2 + b^2 = c^2 \geq 0
  \end{equation}

  \begin{equation}\label{eq:test2}
    a^2 + b^2 = c^2 \geq 0
  \end{equation}
\end{theorem}

我想引用定理~\ref{thm:test} 和公式~\ref{eq:test2}


定理括号测试:

\begin{theorem}
  测试
  \begin{enumerate}
    \item 中文(括号)没输入空格的效果
    \item 中文 (括号) 输入空格的效果
    \item 西文(括号)没输入空格的效果
    \item 西文 (括号) 输入空格的效果
  \end{enumerate}
\end{theorem} 


\begin{proof}
  test
  \[
    a^2 + b^2 = c^2
  \]
\end{proof}

\begin{proof}
  test
  \[
    a^2 + b^2 = c^2  \qedhere
  \]
\end{proof}

\section{浮动体使用}

用 \verb|\label| 引用时,只需要将其放在 \verb|\caption| 的下一行即可。

和定理类环境的引用相同,建议 label 的名称格式为 \verb|figure:xxx| 或 \verb|table:xxx| 其中 \verb|xxx| 可以写中文,尽可能言简意赅地写这个图或表的内容描述,也尽可能写出图表的“独一无二性”,方便自己记忆,也防止在图表一多的时候不知道引用哪一个。

\verb|\figure| 的 \verb|\caption| 是放在 \verb|\includegraphics| 的下方,而 \verb|\table| 的 \verb|\caption| 是放在 \verb|tabular| 或 \verb|tblr| 环境的上方。

\begin{figure}[htbp]
  \centering
  \includegraphics[width = 5cm]{example-image-a}
  \caption{测试}
  \label{figure:test}
\end{figure}

\begin{table}[htbp]
  \centering
  \caption{测试}
  \label{table:test}
  \begin{tabular}{|c|c|}
    11 & 22 \\
    33 & 44 
  \end{tabular}
\end{table}

图 \ref{figure:test} 和表 \ref{table:test} 用来测试两个浮动体和交叉引用



\section{部分数学符号的输入}

本节主要是一些数学符号的输入介绍


\subsection{直体符号}

科技类论文中,建议一些数学符号使用直体(“up”前缀表示直体)
  \begin{itemize}
    \item 直立的 pi :\verb|\uppi| $\to \uppi$
    \item 直立的 e :\verb|\upe| $\to \upe$
    \item 直立的 i :\verb|\upi| $\to \upi$
  \end{itemize}


\section{参考文献引用}

\subsection{数学类}

行间\parencite[thm 3.1]{zurek2014quantum}

行间\parencite{zurek2014quantum}



\subsection{文科类}

上标\cite[test]{zurek2014quantum}

上标\cite{zurek2014quantum}



\section{《附件4:关于修订毕业论文注释与参考文献著录格式的通知》中的参考文献效果}

  text\parencite{李晓东rawtype}

  text\parencite{Ahnrawtype}

  text\parencite{Ahnrawtype}

  text\parencite{丁文祥rawtype}

  text\parencite{邱泽奇会议论文集rawtype}

  text\parencite{雷光春rawtype}

  text\parencite{zhangrawtype}

  text\parencite{邱泽奇会议论文rawtype}

  text\parencite{马克思rawtype}

  text\parencite{昂温rawtype}

  text\parencite{Fothrawtype}

  text\parencite{杨国枢rawtype}

  text\parencite{Morisonrawtype}

  text\parencite{张志祥rawtype}

  text\parencite{徐秀英rawtype}

  text\parencite{Aldemitarawtype}

  text\parencite{张凯军rawtype}

  text\parencite{Kosekrawtype}

  text\parencite{文献编写rawtype}

  text\parencite{国防白皮rawtype}

  text\parencite{federalrawtype}

  text\parencite{healthrawtype}

  text\parencite{江向东rawtype}

  text\parencite{萧钮rawtype}

  text\parencite{Dublinrawtype}

文字测试| 代码注释掉即可(不建议删除,因为随时可以取消注释查看效果)。


\section{列表环境}


如果要修改 \verb|enumerate| 环境的 label 样式的话:

\begin{enumerate}
  \item 第一项
  \item 第二项
  \item 第三项
  \item 第四项
\end{enumerate}

\begin{enumerate}[1)]
  \item 第一项
  \item 第二项
  \item 第三项
  \item 第四项
\end{enumerate}

\begin{enumerate}[a.]
  \item 第一项
  \item 第二项
  \item 第三项
  \item 第四项
\end{enumerate}

\begin{enumerate}[(A)]
  \item 第一项
  \item 第二项
  \item 第三项
  \item 第四项
\end{enumerate}

test
\begin{enumerate}[(i)]
  \item 第一项
  \item 第二项
  \item 第三项
  \item 第四项
\end{enumerate}

\begin{enumerate}[I]
  \item 第一项
  \item 第二项
  \item 第三项
  \item 第四项
\end{enumerate}

\begin{enumerate}[label = \textbf{断言} \Alph*]
  \item 一般来说使用断言,推荐使用 claim 环境
  \item 但是如果真要有一些断言的层级分化的话
  \item 可以考虑用 enumerate 环境的 label 选项
\end{enumerate}

\begin{enumerate}[\textbf{断言} A]
  \item 1
  \item 2
\end{enumerate}


\section{已定义好的一些数学定理环境}

定理环境内的括号,不管是中文还是西文括号,都不会出现倾斜,不需要像旧模板一样需要用手动用 \verb|\textit| 调整
\begin{definition}[测度]
  (参见文献xxx) 这是一段文字 $E = m c^2$  (中文括号)和 (西文括号)
\end{definition}

\begin{theorem}
  这是一段文字 $E = m c^2$
\end{theorem}


\begin{proof}
  这是一段文字 $E = m c^2$
\end{proof}

\begin{proof}[定理xx的证明]
  这是一段文字 $E = m c^2$
\end{proof}

\begin{example}
  这是一段文字 $E = m c^2$
\end{example}

\begin{property}
  这是一段文字 $E = m c^2$
\end{property}

\begin{proposition}
  这是一段文字 $E = m c^2$
\end{proposition}

\begin{corollary}
  这是一段文字 $E = m c^2$
\end{corollary}

\begin{lemma}
  这是一段文字 $E = m c^2$
\end{lemma}

\begin{axiom}
  这是一段文字 $E = m c^2$
\end{axiom}

\begin{counterexample}
  这是一段文字 $E = m c^2$
\end{counterexample}

\begin{conjecture}
  这是一段文字 $E = m c^2$
\end{conjecture}

\begin{question}
  这是一段文字 $E = m c^2$
\end{question}

\begin{claim}
  这是一段文字 $E = m c^2$
\end{claim}

\begin{remark}
  这是一段文字 $E = m c^2$
\end{remark}

\begin{theorem}[Cauchy]\label{thm:test}
  这是一个定理
  \begin{equation}\label{eq:test1}
    a^2 + b^2 = c^2 \geq 0
  \end{equation}

  \begin{equation}\label{eq:test2}
    a^2 + b^2 = c^2 \geq 0
  \end{equation}
\end{theorem}

我想引用定理~\ref{thm:test} 和公式~\ref{eq:test2}


定理括号测试:

\begin{theorem}
  测试
  \begin{enumerate}
    \item 中文(括号)没输入空格的效果
    \item 中文 (括号) 输入空格的效果
    \item 西文(括号)没输入空格的效果
    \item 西文 (括号) 输入空格的效果
  \end{enumerate}
\end{theorem} 


\begin{proof}
  test
  \[
    a^2 + b^2 = c^2
  \]
\end{proof}

\begin{proof}
  test
  \[
    a^2 + b^2 = c^2  \qedhere
  \]
\end{proof}

\section{浮动体使用}

用 \verb|\label| 引用时,只需要将其放在 \verb|\caption| 的下一行即可。

和定理类环境的引用相同,建议 label 的名称格式为 \verb|figure:xxx| 或 \verb|table:xxx| 其中 \verb|xxx| 可以写中文,尽可能言简意赅地写这个图或表的内容描述,也尽可能写出图表的“独一无二性”,方便自己记忆,也防止在图表一多的时候不知道引用哪一个。

\verb|\figure| 的 \verb|\caption| 是放在 \verb|\includegraphics| 的下方,而 \verb|\table| 的 \verb|\caption| 是放在 \verb|tabular| 或 \verb|tblr| 环境的上方。

\begin{figure}[htbp]
  \centering
  \includegraphics[width = 5cm]{example-image-a}
  \caption{测试}
  \label{figure:test}
\end{figure}

\begin{table}[htbp]
  \centering
  \caption{测试}
  \label{table:test}
  \begin{tabular}{|c|c|}
    11 & 22 \\
    33 & 44 
  \end{tabular}
\end{table}

图 \ref{figure:test} 和表 \ref{table:test} 用来测试两个浮动体和交叉引用



\section{部分数学符号的输入}

本节主要是一些数学符号的输入介绍


\subsection{直体符号}

科技类论文中,建议一些数学符号使用直体(“up”前缀表示直体)
  \begin{itemize}
    \item 直立的 pi :\verb|\uppi| $\to \uppi$
    \item 直立的 e :\verb|\upe| $\to \upe$
    \item 直立的 i :\verb|\upi| $\to \upi$
  \end{itemize}


\section{参考文献引用}

\subsection{数学类}

行间\parencite[thm 3.1]{zurek2014quantum}

行间\parencite{zurek2014quantum}



\subsection{文科类}

上标\cite[test]{zurek2014quantum}

上标\cite{zurek2014quantum}



\section{《附件4:关于修订毕业论文注释与参考文献著录格式的通知》中的参考文献效果}

  text\parencite{李晓东rawtype}

  text\parencite{Ahnrawtype}

  text\parencite{Ahnrawtype}

  text\parencite{丁文祥rawtype}

  text\parencite{邱泽奇会议论文集rawtype}

  text\parencite{雷光春rawtype}

  text\parencite{zhangrawtype}

  text\parencite{邱泽奇会议论文rawtype}

  text\parencite{马克思rawtype}

  text\parencite{昂温rawtype}

  text\parencite{Fothrawtype}

  text\parencite{杨国枢rawtype}

  text\parencite{Morisonrawtype}

  text\parencite{张志祥rawtype}

  text\parencite{徐秀英rawtype}

  text\parencite{Aldemitarawtype}

  text\parencite{张凯军rawtype}

  text\parencite{Kosekrawtype}

  text\parencite{文献编写rawtype}

  text\parencite{国防白皮rawtype}

  text\parencite{federalrawtype}

  text\parencite{healthrawtype}

  text\parencite{江向东rawtype}

  text\parencite{萧钮rawtype}

  text\parencite{Dublinrawtype}

文字测试| 代码注释掉即可(不建议删除,因为随时可以取消注释查看效果)。


\section{列表环境}


如果要修改 \verb|enumerate| 环境的 label 样式的话:

\begin{enumerate}
  \item 第一项
  \item 第二项
  \item 第三项
  \item 第四项
\end{enumerate}

\begin{enumerate}[1)]
  \item 第一项
  \item 第二项
  \item 第三项
  \item 第四项
\end{enumerate}

\begin{enumerate}[a.]
  \item 第一项
  \item 第二项
  \item 第三项
  \item 第四项
\end{enumerate}

\begin{enumerate}[(A)]
  \item 第一项
  \item 第二项
  \item 第三项
  \item 第四项
\end{enumerate}

test
\begin{enumerate}[(i)]
  \item 第一项
  \item 第二项
  \item 第三项
  \item 第四项
\end{enumerate}

\begin{enumerate}[I]
  \item 第一项
  \item 第二项
  \item 第三项
  \item 第四项
\end{enumerate}

\begin{enumerate}[label = \textbf{断言} \Alph*]
  \item 一般来说使用断言,推荐使用 claim 环境
  \item 但是如果真要有一些断言的层级分化的话
  \item 可以考虑用 enumerate 环境的 label 选项
\end{enumerate}

\begin{enumerate}[\textbf{断言} A]
  \item 1
  \item 2
\end{enumerate}


\section{已定义好的一些数学定理环境}

定理环境内的括号,不管是中文还是西文括号,都不会出现倾斜,不需要像旧模板一样需要用手动用 \verb|\textit| 调整
\begin{definition}[测度]
  (参见文献xxx) 这是一段文字 $E = m c^2$  (中文括号)和 (西文括号)
\end{definition}

\begin{theorem}
  这是一段文字 $E = m c^2$
\end{theorem}


\begin{proof}
  这是一段文字 $E = m c^2$
\end{proof}

\begin{proof}[定理xx的证明]
  这是一段文字 $E = m c^2$
\end{proof}

\begin{example}
  这是一段文字 $E = m c^2$
\end{example}

\begin{property}
  这是一段文字 $E = m c^2$
\end{property}

\begin{proposition}
  这是一段文字 $E = m c^2$
\end{proposition}

\begin{corollary}
  这是一段文字 $E = m c^2$
\end{corollary}

\begin{lemma}
  这是一段文字 $E = m c^2$
\end{lemma}

\begin{axiom}
  这是一段文字 $E = m c^2$
\end{axiom}

\begin{counterexample}
  这是一段文字 $E = m c^2$
\end{counterexample}

\begin{conjecture}
  这是一段文字 $E = m c^2$
\end{conjecture}

\begin{question}
  这是一段文字 $E = m c^2$
\end{question}

\begin{claim}
  这是一段文字 $E = m c^2$
\end{claim}

\begin{remark}
  这是一段文字 $E = m c^2$
\end{remark}

\begin{theorem}[Cauchy]\label{thm:test}
  这是一个定理
  \begin{equation}\label{eq:test1}
    a^2 + b^2 = c^2 \geq 0
  \end{equation}

  \begin{equation}\label{eq:test2}
    a^2 + b^2 = c^2 \geq 0
  \end{equation}
\end{theorem}

我想引用定理~\ref{thm:test} 和公式~\ref{eq:test2}


定理括号测试:

\begin{theorem}
  测试
  \begin{enumerate}
    \item 中文(括号)没输入空格的效果
    \item 中文 (括号) 输入空格的效果
    \item 西文(括号)没输入空格的效果
    \item 西文 (括号) 输入空格的效果
  \end{enumerate}
\end{theorem} 


\begin{proof}
  test
  \[
    a^2 + b^2 = c^2
  \]
\end{proof}

\begin{proof}
  test
  \[
    a^2 + b^2 = c^2  \qedhere
  \]
\end{proof}

\section{浮动体使用}

用 \verb|\label| 引用时,只需要将其放在 \verb|\caption| 的下一行即可。

和定理类环境的引用相同,建议 label 的名称格式为 \verb|figure:xxx| 或 \verb|table:xxx| 其中 \verb|xxx| 可以写中文,尽可能言简意赅地写这个图或表的内容描述,也尽可能写出图表的“独一无二性”,方便自己记忆,也防止在图表一多的时候不知道引用哪一个。

\verb|\figure| 的 \verb|\caption| 是放在 \verb|\includegraphics| 的下方,而 \verb|\table| 的 \verb|\caption| 是放在 \verb|tabular| 或 \verb|tblr| 环境的上方。

\begin{figure}[htbp]
  \centering
  \includegraphics[width = 5cm]{example-image-a}
  \caption{测试}
  \label{figure:test}
\end{figure}

\begin{table}[htbp]
  \centering
  \caption{测试}
  \label{table:test}
  \begin{tabular}{|c|c|}
    11 & 22 \\
    33 & 44 
  \end{tabular}
\end{table}

图 \ref{figure:test} 和表 \ref{table:test} 用来测试两个浮动体和交叉引用



\section{部分数学符号的输入}

本节主要是一些数学符号的输入介绍


\subsection{直体符号}

科技类论文中,建议一些数学符号使用直体(“up”前缀表示直体)
  \begin{itemize}
    \item 直立的 pi :\verb|\uppi| $\to \uppi$
    \item 直立的 e :\verb|\upe| $\to \upe$
    \item 直立的 i :\verb|\upi| $\to \upi$
  \end{itemize}


\section{参考文献引用}

\subsection{数学类}

行间\parencite[thm 3.1]{zurek2014quantum}

行间\parencite{zurek2014quantum}



\subsection{文科类}

上标\cite[test]{zurek2014quantum}

上标\cite{zurek2014quantum}



\section{《附件4:关于修订毕业论文注释与参考文献著录格式的通知》中的参考文献效果}

  text\parencite{李晓东rawtype}

  text\parencite{Ahnrawtype}

  text\parencite{Ahnrawtype}

  text\parencite{丁文祥rawtype}

  text\parencite{邱泽奇会议论文集rawtype}

  text\parencite{雷光春rawtype}

  text\parencite{zhangrawtype}

  text\parencite{邱泽奇会议论文rawtype}

  text\parencite{马克思rawtype}

  text\parencite{昂温rawtype}

  text\parencite{Fothrawtype}

  text\parencite{杨国枢rawtype}

  text\parencite{Morisonrawtype}

  text\parencite{张志祥rawtype}

  text\parencite{徐秀英rawtype}

  text\parencite{Aldemitarawtype}

  text\parencite{张凯军rawtype}

  text\parencite{Kosekrawtype}

  text\parencite{文献编写rawtype}

  text\parencite{国防白皮rawtype}

  text\parencite{federalrawtype}

  text\parencite{healthrawtype}

  text\parencite{江向东rawtype}

  text\parencite{萧钮rawtype}

  text\parencite{Dublinrawtype}

文字测试   % 常用命令环境示例,不需要时注释掉即可
\chapter{引言}

\section{研究背景}
\begin{figure}[tbh]
  \centering
  \includegraphics[width = 5cm]{example-image-a}
  \caption{测试}
\end{figure}
\begin{table}[tbh]
  \caption{测试}
  \centering
  \begin{tblr}{|c|c|}
    11 & 22 \\
    33 & 44 
  \end{tblr}
\end{table}



\subsection{前人工作}


测试 \parencite{邱泽奇建构与分化}
\section{研究背景}

% !TeX root = ../main.tex

\chapter{引用与链接}

\section{脚注}

华中师范大学《附件4:关于修订毕业论文注释与参考文献著录格式的通知》提到

\begin{itemize}
  \item 文科术科的论文注释使用脚注
  \item 理工科的论文注释不使用脚注
\end{itemize}

\subsection{测试}

\zhlipsum[1]



\section{引用文中小节}\label{sec:ref}

如引用小节~\ref{sec:ref}



\section{引用参考文献}

这是一个参考文献引用的范例:“\parencite{邱泽奇建构与分化}提出……”。还可以引用多个文献:“\parencite{丁文祥rawtype,李晓东rawtype}提出……”。


\section{链接相关}


模板使用了 hyperref 包处理相关链接,使用 \verb|\href| 可以生成超链接,默认不显示链接颜色。如果需要输出网址,可以使用 \verb|\url| 命令,示例:\url{https://github.com}。

% \zhlipsum[1-4]
% \zhlipsum[1-4]
% \zhlipsum[1-4]
% \zhlipsum[1-4]
% \zhlipsum[1-4]
% \zhlipsum[1-4]

\begin{equation}
  x^2
\end{equation}
\chapter{问题研究}


\section{已定义好的一些定理环境}

\begin{definition}[测度]
  (参见文献xxx)这是一段文字 $E = m c^2$
\end{definition}

\begin{theorem}
  这是一段文字 $E = m c^2$
\end{theorem}

\begin{proof}
  这是一段文字 $E = m c^2$
\end{proof}

\begin{proof}[定理xx的证明]
  这是一段文字 $E = m c^2$
  \[
    \int_{0}^{1} x^2 dx
  \]
  okkk
\end{proof}

\begin{example}
  这是一段文字 $E = m c^2$
\end{example}

\begin{property}
  这是一段文字 $E = m c^2$
\end{property}

\begin{proposition}
  这是一段文字 $E = m c^2$
\end{proposition}

\begin{corollary}
  这是一段文字 $E = m c^2$
\end{corollary}

\begin{lemma}
  这是一段文字 $E = m c^2$
\end{lemma}

\begin{axiom}
  这是一段文字 $E = m c^2$
\end{axiom}

\begin{antiexample}
  这是一段文字 $E = m c^2$
\end{antiexample}

\begin{conjecture}
  这是一段文字 $E = m c^2$
\end{conjecture}

\begin{question}
  这是一段文字 $E = m c^2$
\end{question}

\begin{claim}
  这是一段文字 $E = m c^2$
\end{claim}

\begin{remark}
  这是一段文字 $E = m c^2$
\end{remark}


\section{测试}

测试文字,测试文字测试文字测试文字,测试文字测试文字测试文字,测试文字测试文字测试文字,测试文字测试文字



\section{测试}


\subsection{测试}

测试文字,测试文字测试文字测试文字,测试文字测试文字测试文字,测试文字测试文字测试文字,测试文字测试文字


\subsection{测试}

测试文字,测试文字测试文字测试文字,测试文字测试文字测试文字,测试文字测试文字测试文字,测试文字测试文字


\subsection{测试}

测试文字,测试文字测试文字测试文字,测试文字测试文字测试文字,测试文字测试文字测试文字,测试文字测试文字


% !TeX root = ../main.tex

\chapter{公式}



\section{公式引用}

\begin{equation}\label{equation:test1}
  K \leq 
  \begin{dcases}
    2 \lfloor \frac{d}{2d - n} \rfloor    , & K \text{为偶数} ; \\
    2 \lfloor \frac{d}{2d - n} \rfloor - 1, & K \text{为奇数}. \\
  \end{dcases}
  % \left\{ 
  %   \begin{array}{cl}
  %     2 \lfloor \frac{d}{2d - n} \rfloor    , & K \text{为偶数} ; \\
  %     2 \lfloor \frac{d}{2d - n} \rfloor - 1, & K \text{为奇数}. \\
  %   \end{array}
  % \right. 
\end{equation}

公式~\eqref{equation:test1} 的引用



% \backmatter开启后置部分,包含参考文献、致谢、附录等
% \backmatter


%%%% 参考文献 %%%%
%%%%%%%%%%%%%%%%%%%%%%%%%%
%%% 行内引用:\parencite{} or \parencite[]{},下面两个情况要用行内引用
%    - 去掉这个引用句子结构不完整,比如“定理证明可参看[1]”
%    - 英文文献的引用
%%% 上标引用:\cite{} or \cite[]{},下面情况要用上标引用
%    - 去掉这个引用,句子结构完整,比如“作者提到,‘CCNUthesis真是一个好模版’^[1]”
%      其中“^[1]” 表示上标引用
%%%%%%%%%%%%%%%%%%%%%%%%%%
% 打印参考文献列表
\printbibliography


%%%% 致谢 %%%%
\chapter*{\centerline{致 \quad 谢}}
\addcontentsline{toc}{chapter}{\normalfont 致谢}   % 根据旧模版样式,如果有写致谢,则需要把致谢放进目录


感谢各位的使用,欢迎提出issue和bug!


%%%% 附录 %%%%
% 没有附录内容的把下面的代码注释掉即可
\chapter*{附录}
\addcontentsline{toc}{chapter}{\normalfont 附录}

此处可以写调查问卷,访谈记录等

用\verb|xchoices{}|环境可以排版任意个选项,用\verb|\item|分隔即可(现将代码注释掉,如有需要取消注释即可,如果不需要,可自行删除或不管(因为不影响编译效果)

% \begin{xchoices}
%   \item 选项1
%   \item 选项2
%   \item 选项3
%   \item 选项4
% \end{xchoices}

% \begin{xchoices}[
%   items = 2,            % 手动控制每行多少个选项
%   label-style = alph,   % label的样式:alph, Alph, roman, Roman, 
%                         %            quan(带圈数字), chinese, none
%   pre-label = {(},      % label前面的内容
%   post-label = {)},     % label后面的内容
% ]
%   \item 选项1
%   \item 选项2
%   \item 选项3
%   \item 选项4
% \end{xchoices}

% \begin{xchoices}[label-style = quan]
%   \item 选项1
%   \item 选项2
%   \item 选项3
%   \item 选项4
% \end{xchoices}

\end{document}
