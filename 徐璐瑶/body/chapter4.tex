\chapter{柯西不等式}

柯西不等式是法国数学家柯西在研究数学分析中“留数”中所得,但从历史上来讲,其应为柯西与布尼亚科夫斯基和施瓦茨的共同杰作,称为Cauchy-Buniakowsky-Schwarz不等式,因后面两位数学家彼此独立地在积分学中推而广之。柯西形式一般形式为 \[\sum_{i=1}^{n} a_{i}{ }^{2} \sum_{i=1}^{n} b_{i}^{2} \geq\left(\sum_{i=1}^{n} a_{i} b_{i}\right)^{2},\] 当且仅当 $\frac{b_{1}}{a_{1}}=\frac{b_{2}}{a_{2}}=\cdots=\frac{b_{n}}{a_{n}}$ 或 $a_{i},  b_{i},i=1,2, \cdots, n$ 中至少一方全为零时等号成立。柯西不等式的二维形式最 为常见, 为高中数学竞赛的基础, 除此之外, 其向量形式、三角形式概率形式等 也被广泛应用 \parencite{李梦2019数学竞赛中柯西不等式的教学研究} 。


\section{柯西不等式的证明}

柯西不等式的证明方法同样有多种,这里同上文证明方法,用数学归纳法进行证明。

\begin{proof}
  \begin{enumerate}
    \item 当 $n=1$ 时,结论成立;
    \item 当 $n=2$ 时, 即证 $\left(a_{1}^{2}+a_{2}^{2}\right)\left(b_{1}^{2}+b_{2}^{2}\right) \geq\left(a_{1} b_{1}+a_{2} b_{2}\right)^{2}$, 即证 $a_{1}{ }^{2} b_{2}{ }^{2}+$ $a_{2}^{2} b_{1}^{2} \geq 2 a_{1} b_{1} a_{2} b_{2}$, 即证 $\left(a_{1} b_{2}-a_{2} b_{1}\right)^{2} \geq 0$, 当且仅当 $\frac{b_{1}}{a_{1}}=\frac{b_{2}}{a_{2}}$ 时等号成立或 $a_{i}, b_{i}$ $i=1,2$ 中至少一方全为零时等号成立, 故结论成立;
    \item 不妨设 $n=k$ 时结论成立, 即 $\sum_{i=1}^{k} a_{i}{ }^{2} \sum_{i=1}^{k} b_{i}^{2} \geq\left(\sum_{i=1}^{k} a_{i} b_{i}\right)^{2}$, 当且仅当 $\frac{b_{1}}{a_{1}}=\frac{b_{2}}{a_{2}}=$ $\cdots=\frac{b_{k}}{a_{k}}$ 或 $a_{i}, b_{i}, i=1,2, \cdots, k$ 中至少一方全为零时等号成立。
    
    当 $n=k+1$ 时,
    \begin{align*}
      \sum_{i=1}^{k+1} a_{i}^{2} \sum_{i=1}^{k+1} b_{i}^{2} &=\left(\sum_{i=1}^{k} a_{i}^{2}+a_{k+1}^{2}\right)\left(\sum_{i=1}^{k} b_{i}^{2}+b_{k+1}^{2}\right) \\
      &=\sum_{i=1}^{k} a_{i}^{2} \sum_{i=1}^{k} b_{i}^{2}+a_{k+1}{ }^{2} \sum_{i=1}^{k} b_{i}^{2} \\
      &+b_{k+1}^{2} \sum_{i=1}^{k} a_{i}^{2}+a_{k+1} b_{k+1}^{2},
    \end{align*}且
    \begin{align*}
      \left(\sum_{i=1}^{k+1} a_{i} b_{i}\right)^{2} &=\left(\sum_{i=1}^{k} a_{i} b_{i}+a_{k+1} b_{k+1}\right)^{2} \\
      &=\left(\sum_{i=1}^{k} a_{i} b_{i}\right)^{2}+\left(a_{k+1} b_{k+1}\right)^{2}+2 a_{k+1} b_{k+1} \sum_{i=1}^{k} a_{i} b_{i}
    \end{align*}
    要证明 \[\sum_{i=1}^{k+1} a_{i}^{2} \sum_{i=1}^{k+1} b_{i}^{2} \geq\left(\sum_{i=1}^{k+1} a_{i} b_{i}\right)^{2},\]
    即证
    \[
      a_{k+1}^2 \sum_{i=1}^{k} b_{i}^{2}+b_{k+1}^{2} \sum_{i=1}^{k} a_{i}^{2} \geq 2 a_{k+1} b_{k+1} \sum_{i=1}^{k} a_{i} b_{i},
    \]
    即证 $\sum_{i=1}^{k}\left(a_{k+1} b_{i}-b_{k+1} a_{i}\right)^{2} \geq 0$, 当且仅当 $\frac{b_{k+1}}{a_{k+1}}=\frac{b_{i}}{a_{i}}, i=1,2, \cdots, k$ 即 $\frac{b_{1}}{a_{1}}=\frac{b_{2}}{a_{2}}=\cdots=\frac{b_{k+1}}{a_{k+1}}$ 或 $a_{i}, b_{i}, i=1,2, \cdots, k+1$ 中至少一方全为零时等号成立,故结论成立。
  \end{enumerate}

  综上所述,原命题得证。
\end{proof}



\section{柯西不等式的变式及推广}

柯西不等式的变式形式有多种,数分教材在第九章《定积分》习题中给出了两类其变式形式——施瓦茨(Schwarz)不等式和闵可夫斯基(Minkowski)不等式。

\begin{theorem}[施瓦茨(Schwarz)不等式]\label{thm:shwarzinequality}
  若 $f$ 和 $g$ 在 $[a, b]$ 上可积, 则 \[\left(\int_{a}^{b} f(x) g(x) \dd{x}\right)^{2} \leq \int_{a}^{b} f^{2}(x) \dd{x} \cdot \int_{a}^{b} g^{2}(x) \dd{x}.\]
\end{theorem}


\begin{theorem}[闵可夫斯基(Minkowski)不等式]
  若 $f$ 和 $g$ 在 $[a, b]$ 上可积, 则 \[\left[\int_{a}^{b}(f(x)+g(x))^{2} \dd{x}\right]^{\frac{1}{2}} \leq\left[\int_{a}^{b} f^{2}(x) \dd{x}\right]^{\frac{1}{2}}+\left[\int_{a}^{b} g^{2}(x) \dd{x}\right]^{\frac{1}{2}}.\]
\end{theorem}


施瓦茨不等式可利用柯西不等式进行证明,下文会给出示例;闵可夫斯基不等式可利用施瓦茨不等式进行证明,具体证明方法可见数学分析教材的习题精解。施瓦茨不等式和闵可夫斯基不等式也是两类重要的数学不等式,可在各类学科中运用,在各类理论研究中占据重要地位。

除教材上给出的推广,柯西不等式还可推广出赫尔德(Hölder)不等式.


\begin{theorem}[赫尔德(Holder)不等式]
  设 $a_{i}>0, b_{i}>0(i=1,2, \cdots, n), p>0, q>0$,满足 $\frac{1}{p}+\frac{1}{q}=1$, 则有 \[\sum_{i=1}^{n} a_{i} b_{i} \leq\left(\sum_{i=1}^{n} a_{i}^{p}\right)^{\frac{1}{p}}\left(\sum_{i=1}^{n} b_{i}^{q}\right)^{\frac{1}{q}}.\] 等号成立当且仅当 $a_{i}^{p}=\lambda b_{i}^{q}(i=1,2, \cdots, n, \lambda>0)$.
\end{theorem}

该不等式在Young不等式的基础上证明较为简便,Young不等式这里不再给出,有兴趣的读者可自行查阅资料进行解答。


\section{柯西不等式例题分析}

柯西不等式尤得高中数学竞赛的青睐,高中数学竞赛中一直都有柯西不等式的影子,现给出实例。

\begin{example}
  \parencite{李梦2019数学竞赛中柯西不等式的教学研究} 设实数 $a, b, c$ 满足 $a^{2}+2 b^{2}+3 c^{2}=\frac{3}{2}$, 求证: $3^{-a}+9^{-b}+27^{-c} \geq 1$. 
\end{example}

\begin{proof}
  由 
  \begin{gather*}
    3^{-a}+9^{-b}+27^{-c}=3^{-a}+3^{-2 b}+3^{-3 c} \geq 3 \sqrt[3]{3^{-a-2 b-3 c}}=3 \sqrt[3]{3^{-(a+2 b+3 c)}}\\
    (a+2 b+3 c)^{2}\leq\left[1+(\sqrt{2})^{2}+(\sqrt{3})^{2}\right]\left[a^{2}+(\sqrt{2} b)^{2}+(\sqrt{3} c)^{2}\right]=9
  \end{gather*}
  可知 $a+2 b+3 c \leq 3$, 所以有 \[3^{-a}+9^{-b}+27^{-c} \geq 3 \sqrt[3]{3^{-(a+2 b+3 c)}}=1,\] 故原命题得证。
\end{proof}


\begin{analysis}
  该题目是均值不等式和柯西不等式的综合应用,第一步的化简为后面柯西不等式三维形式的给出创造了条件。学生求解该题最重要的便是构造出与题目条件相关的 $a, b, c$ 之间的关系式,并变换形式以使用柯西不等式得出答案。
\end{analysis}


\begin{example}
  \parencite{朱小扣2020聚焦柯西不等式在竞赛中四大运用} 函数 $y=\sqrt{x-5}+\sqrt{24-3 x}$ 的最大值为? 
\end{example}

\begin{proof}
  \begin{align*}
    (\sqrt{x-5}+\sqrt{24-3 x})^{2}&=\left(\frac{1}{a} \sqrt{a^{2} x-5 a^{2}}+\frac{1}{b} \sqrt{24 b^{2}-3 b^{2} x}\right)^{2} \\ & \leq\left(\frac{1}{a^{2}}+\frac{1}{b^{2}}\right)\left(\left(a^{2}-3 b^{2}\right) x+24 b^{2}-5 a^{2}.\right) 
  \end{align*}
  令 $a^{2}-3 b^{2}=0$, 即 $a^{2}=3 b^{2}$, 所以有
  \begin{align*}
    (\sqrt{x-5}+\sqrt{24-3 x})^{2} &=\left(\frac{1}{3 b^{2}} \sqrt{3 b^{2} x-15 b^{2}}+\frac{1}{b} \sqrt{24 b^{2}-3 b^{2} x}\right)^{2} \\
    & \leq\left(\frac{1}{3 b^{2}}+\frac{1}{b^{2}}\right)\left(24 b^{2}-15 b^{2}\right)=12
  \end{align*}
  当且仅当 $\frac{\frac{1}{\sqrt{b}}}{\frac{1}{\sqrt{3 b}}}=\frac{\sqrt{24 b^2 - 3 b^2 x}}{\sqrt{3 b^{2} x-15 b^{2}}}$ 即 $x=\frac{23}{4}$ 时等号成立,从而 $\sqrt{x-5}+\sqrt{24-3 x} \leq 2 \sqrt{3}$,故原命题最大值为 $2 \sqrt{3}$.
\end{proof}

\begin{analysis}
  该题目是柯西不等式在最值方面上的应用,难点在于运用待定系数法消掉未知量,并构造出柯西不等式作用的形式与条件。运用于该题,此方法可求出含有两个根号的函数的最值,熟能生巧以后,也可为含有多个根号的函数最值指明一条解题道路。
\end{analysis}


\begin{example}
  \parencite{华东师大数学分析} 证明施瓦茨(Schwarz)不等式(定理~\ref{thm:shwarzinequality})
\end{example}


\begin{proof}
  因为 $f$ 和 $g$ 在 $[a, b]$ 上可积, 所以根据定积分的定义可知:
  \[
  \int_{a}^{b} f(x) g(x) d x=\sum_{i=1}^{n} f\left(\varepsilon_{i}\right) g\left(\varepsilon_{i}\right) \cdot \frac{b-a}{n}
  \]
  其中对 $[a, b]$ 给出分割 $T=\left\{\Delta_{1}, \Delta_{2}, \cdots, \Delta_{n}\right\}$,任取 $\varepsilon_{i} \in \Delta_{i}, i=1,2, \cdots, n.$
  同理可得,
  \[
  \begin{aligned}
  &\int_{a}^{b} f^{2}(x) d x=\sum_{i=1}^{n} f^{2}\left(\varepsilon_{i}\right) \cdot \frac{b-a}{n} \\
  &\int_{a}^{b} g^{2}(x) d x=\sum_{i=1}^{n} g^{2}\left(\varepsilon_{i}\right) \cdot \frac{b-a}{n}
  \end{aligned}
  \]
  现只需要证 \[\left(\sum_{i=1}^{n} f\left(\varepsilon_{i}\right) g\left(\varepsilon_{i}\right) \cdot \frac{b-a}{n}\right)^{2} \leq\left[\sum_{i=1}^{n} f^{2}\left(\varepsilon_{i}\right) \cdot \frac{b-a}{n}\right]\left[\sum_{i=1}^{n} g^{2}\left(\varepsilon_{i}\right) \cdot \frac{b-a}{n}\right].\]

  由柯西不等式可知:
  \[
    \begin{aligned} 
    \left(\sum_{i=1}^{n} f\left(\varepsilon_{i}\right) g\left(\varepsilon_{i}\right) \cdot \frac{b-a}{n}\right)^{2}=&\left[\sum_{i=1}^{n}\left(f\left(\varepsilon_{i}\right) \sqrt{\frac{b-a}{n}}\right)\left(g\left(\varepsilon_{i}\right) \sqrt{\frac{b-a}{n}}\right)\right]^{2} \\ 
    \leq &\left[\sum_{i=1}^{n}\left(f\left(\varepsilon_{i}\right) \sqrt{\frac{b-a}{n}}\right)^{2}\right]\left[\sum_{i=1}^{n}\left(g\left(\varepsilon_{i}\right) \sqrt{\frac{b-a}{n}}\right)^{2}\right]\\
    =& {\left(\sum_{i=1}^{n} f^{2}\left(\varepsilon_{i}\right) \cdot \frac{b-a}{n}\right)\left(\sum_{i=1}^{n} g^{2}\left(\varepsilon_{i}\right) \cdot \frac{b-a}{n}\right) }.
  \end{aligned}\]
  故原命题得证。
\end{proof}


\begin{analysis}
  该例题为数学分析教材《第九章》总复习习题中的一道,在理解了定积分原始定义的基础上利用柯西不等式进行解题。该例题也可通过构造积分不等式进行求解。
\end{analysis}