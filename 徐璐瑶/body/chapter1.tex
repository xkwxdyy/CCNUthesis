\chapter{引言}

\section{研究背景}

《数学分析第四版》华东师范大学数学系编是大学里数学专业学生的专业教材,用于数学分析课程的学习。数学分析的创立始于17世纪以牛顿和莱布尼兹为代表的开创性工作,而完成于19世纪以柯西和威尔斯特拉斯为代表的奠基性工作。

数学分析以函数为研究对象,它从局部和整体这两个方面研究函数的基本形态,从而形成微分学和积分学的基本内容。微分学研究变化率等函数的局部特征,导数和微分是它的主要概念,求导数的过程就是微分法。微分学的主要内容为围绕着导数与微分的性质、计算和直接应用。积分学则从总体上研究微小变化积累的总效果,尤其重视对非均匀变化的研究,以原函数和定积分为基本概念,而其求积分的过程就构成了积分法。积分学的主要内容涉及到积分的性质、计算、推广与直接应用。谈及历史,为数学分析的建立与成型做出伟大贡献的两位数学家,牛顿和莱布尼兹,他们在1670年左右,总结出了求导数与求积分的一系列基本法则,发现了求导数与求积分这两种互逆的运算,并通过后来以他们的名字命名的著名公式——牛顿·莱布尼兹公式——反映了这种互逆关系,从而使得本来各自独立发展的微分学和积分学结合而成一门新的学科—微积分学。而后,又在他们的后继数学家的推动下,如欧拉,使得本来仅为少数数学家所了解,只能较为粗略地解决一系列简单且具体的问题的微分与积分方法,成为一种常人稍加训练即可掌握的唾手可得的方法,由此便打开了把它广泛应用于各类科学技术领域的大门,产生不可估量的作用与影响。因此,微积分的出现与发展被认为是人类文明史上划时代的数学事件之一。数学分析中,无穷级数也有所涉及。历史上,无穷级数的使用由来已久,但只在成为数学分析的一部分后,才得到真正的发展和广泛应用。在数学分析中,无穷级数与微积分从来都是密不可分和相辅相成的,但与积分的连续性相比,无穷级数虽也是微小量的叠加和积累,却采取了离散的形式。

《数学分析第四版》华东师范大学数学系编分上下两册,共二十三章。与上文类似,全书以实数集与函数开头,对数列以及函数的极限进行定义探寻,导数和微分与高中知识接轨,接着从不定积分、定积分、反常积分中对积分进行系统研究,紧接着级数、多元函数的极限与连续性、多元函数微分学登场,含参量积分、曲线积分、重积分、曲面积分将积分与几何紧密联系。这些内容也构成了数学分析课程的主要内容。全书知识结构紧凑,为后续更深入细化的数学分支的学习奠定了基础。同时,该书的内容设置,在一定程度上也展现出数学分析这门学问变迁的脉络。



\section{研究发展现状}

不等式有关理论作为古老的自然科学一大分支,数年来都是学者们研究的热门。国内外对于不等式的研究趋于高深、复杂、多极化。从历史上来看,中国数学家在不等式领域做出了卓越的贡献,如华罗庚先生;近年来,中国的数学学者们也不断活跃在国际数学不等式及其应用的领域中。单一类别的不等式研究,如Hilbert不等式,对其参量化、积分形式、积分形式的推广等都有不同研究报告的呈现;再到各类不等式证明方法的多样化研究,如针对中等数学的用向量法、概率方法等证明不等式,到高等数学中运用数学分析方法的证明;再有将数学文化融入不等式的研究与教学中,如从信息技术入手和课堂艺术出发……不等式的研究多维且深入,拓展到生活的方方面面。

均值不等式的研究发展现状可谓是极为全面,其妙用不只是在数学方面呈现,物理等其他学科也均有涉足。均值不等式的研究涵盖了其适用条件、性质、证明方法等诸多方面,均值不等式也被运用到求解不同类型的题目中,如最值问题、数列问题等[3]。戴伟在《关于均值不等式的一个证明》中更是利用均值不等式给出了酉矩阵的一个刻画\parencite{戴伟2020关于均值不等式的一个证明}。谈及国外,Davit Harutyunyan对几何-算术平均值不等式进行了加强\parencite{harutyunyan2018cauchy};Mohammad Sal Moslehian等人更是给出了不定型的算术-几何均值不等式的形式\parencite{moslehian2021arithmetic}。

伯努利不等式由瑞士数学家伯努利提出,其一般形式笔者将在后文中给出。国外的研究中,其涉及到原函数和双曲函数等多种函数形式,也对负指数形式进行了探索;在国内,马玉梅等在《微积分教学中几个问题的思考》中对的奇偶性进行探讨,得出的取值范围,并将伯努利不等式应用到Jacobsthal不等式\parencite{马玉梅2020微积分教学中几个问题的思考};杨克昌在《一类不等式的加强与综合》中利用伯努利不等式解决含参变量的不等式\parencite{杨克昌1997一类不等式的加强与综合}……伯努利不等式的研究主要运用到其推广形式,在高深的数学领域中依旧有深入探讨的必要。

柯西不等式的一般形式及变式形式有多种,笔者将会在后文给出一二。柯西不等式就其发展进程、证明方法、运用形式、改进方式都是研究的方向,如陈明在《柯西不等式的几点注记》中对柯西不等式的证明,以及其在函数求最值、证明不等式、几何上的广泛应用进行了阐述,展现出柯西不等式在数学理论中的重要地位\parencite{陈明2018柯西不等式的几点注记}。
随着不等式理论的不断发展,多种多样的多项式也出现在人们视线中。均值不等式、伯努利不等式、柯西不等式作为最基础的三类重要的不等式,融入到各类不等式的研究中,体现出不同性质下的价值。



\section{研究内容}

本文主要对《数学分析第四版》(华东师范大学编)中所涉及的三类不等式:均值不等式、伯努利不等式、柯西不等式的证明、推广变式及例题进行分析,并进一步给出教材以外的相应高中数学或大学数学例题,展现出不等式在不同阶段的运用,加深学生对此三类不等式的理解与掌握,从而体现出不等式的应用和教育价值。



\section{研究方法}

通过对以上所涉及知识的研究与分析,本文运用的主要方法为文献参阅法:通过查阅资料研究上文所提及的三类不等式,了解其发展,掌握其证明,熟悉其应用,查阅的主要资料为期刊论文、博硕士学位论文等。



\section{研究目的及意义}

不等式无论从其基础性还是重要性而言都具有重大的研究意义,在解题中巧妙运用不等式总会让人眼前一亮,但碍于自身数学能力的不足,总会有很多人难以发现不同不等式的构造与使用方法,尤其是当题目变得复杂时。本文具体从《数学分析第四版》(华东师范大学编)出发,总结出三类不等式:均值不等式、伯努利不等式、柯西不等式,将其在书中例题的应用整合并推广,体现出该三类不等式在数分中的重要性的同时,给出更多相应例题,让例题从高中数学跨越至大学数学,不仅展现出三类不等式在数分教材外的突出作用和应用办法,也表现出三类不等式在不同阶段下的不同妙用,从而整体达到对不等式及其应用更深入的认识,体现出不等式的应用和教育价值。