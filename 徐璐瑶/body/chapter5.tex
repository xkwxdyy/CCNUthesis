\chapter{总结}

\section{研究的不足与展望}

因才疏学浅,篇幅限制,此次论文中涉及到的三类不等式,笔者均未给出较为详细、充分的证明或是推广,所给例题量也不足以展现出三类不等式的重要程度。该三类不等式单是从数学这一门学科的角度来看,其应用远不止上文例题所涉,笔者在此论文结束后,也不会停止继续探索此几类不等式的脚步,希望能挖掘出更多不等式的精彩。



\section{结束语}

全文虽只是对均值、伯努利和柯西不等式进行分门别类的总结,但若用心便可发现,这三类不等式实则你中有我,我中有你,环环相扣。每一类不等式进行适当变形都可推出另一类的一些形式出来,所以在解题时往往不只一类不等式可以派上用场。这就要求我们开放思维,放宽眼界,从不同的角度入手,找寻运用不等式的方式。同时,三类不等式的例题虽是从高中阶段解题到大学阶段解题的跨越,加入了更多的数学分析课程所涉知识点,如数列极限、积分等,但不等式作用的式子结构和条件却是极为相似,万变不离其宗。这也是要求我们在掌握不等式的同时,牢牢记住不等式成立的条件以及构造出可以运用不等式的式子结构。当然,不等式的有关理论与应用远不止上文所及,还望读者们能一直保持着对数学的热情,对不等式的好奇,在数学学习的道路上获得更多的数学理论知识与学习经验,提升数学素养的同时,获得学习数学的乐趣。