% !TeX root = ../CCNUthesis-doc.tex

\section{使用说明}

\subsection{基本用法}

以下是一份简单的 \TeX{} 文档,它演示了 \cls{CCNUthesis} 的最基本用法:

\begin{latexexample}[deletetexcs={\documentclass},
    moretexcs={\chapter},morekeywords={\documentclass},
    emph={[2]document}]
  % main.tex
  \documentclass{CCNUthesis}
  \begin{document}
    \chapter{欢迎}
    \section{Welcome to CCNUthesis!}
    你好,\LaTeX{}!
  \end{document}
\end{latexexample}


按照~\ref{subsec:编译方式} 小节中的方式编译该文档,您应当得到一篇 3 页的文章。

% 英文模板可以用类似的方式使用:

% \begin{latexexample}[deletetexcs={\documentclass},
%     moretexcs={\chapter},morekeywords={\documentclass},
%     emph={[2]document}]
%   % thesis-en.tex
%   \documentclass{CCNUthesis-en}
%   \begin{document}
%     \chapter{Welcome}
%     \section{Welcome to CCNUthesis!}
%     Hello, \LaTeX{}!
%   \end{document}
% \end{latexexample}

% 英文模板只对正文部分进行了改动,封面、指导小组成员以及声明页仍将
% 显示为中文。

\subsection{编译方式} \label{subsec:编译方式}

本模板不支持 \pdfTeX{} 引擎,仅支持使用 \XeLaTeX{} 。为了生成正确的目录、脚注以及交叉引用,您至少需要连续编译两次。

以下代码中,假设您的 \TeX{} 源文件名为 \file{main.tex}。请在命令行中执行
\begin{shellexample}[morekeywords={xelatex}]
  xelatex main
\end{shellexample}

如果您想要编译参考文献,在 \file{CCNUthesis-main.bib} 中输入正确的条目信息并在正文中正确引用之后(引用方式见~\ref{para:参考文献引用}),请使用“\pkg{biber}”编译链,即在命令行中一次执行下面四条命令:
\begin{shellexample}[morekeywords={biber}]
  xelatex main
  biber main
  xelatex main
  xelatex main
\end{shellexample}
或使用 \pkg{latexmk}:
\begin{shellexample}[morekeywords={latexmk}]
  latexmk main
\end{shellexample}


\subsection{模板选项} \label{subsec:模版选项}

所谓“模板选项”,指需要在引入文档类的时候指定的选项:
\begin{latexexample}[deletetexcs={\documentclass},
    morekeywords={\documentclass}]
  \documentclass(*\oarg{模板选项}*){CCNUthesis}
\end{latexexample}

有些模板选项为布尔型,它们只能在 \opt{true} 和 \opt{false}
中取值。对于这些选项,\kvopt{\meta{选项}}{true} 中的“|= true|”
可以省略。

下面用形如“【本|硕|博】”表示该命令或环境的适用范围,比如“【硕|博】”表示输入之后仅对硕博模版起作用,对本科模版不起作用(作用范围的设置往往来源于格式规范的要求等),其余“【本】”等同理解释。


\begin{function}{type}
  \begin{ccnusyntax}[emph={[1]type}]
    type = (*<doctor|master|(bachelor)>*)
  \end{ccnusyntax}
  选择论文类型。三种选项分别代表博士学位论文、硕士学位论文和本科
  毕业论文。
\end{function}

\begin{function}[updated = 2024-03-19]{version}
  \begin{ccnusyntax}[emph={[1]version}]
    version = (*<(electronic)|print-master-oneside|print-master-twoside|print-doctor>*)
  \end{ccnusyntax}
  文档版本。
\end{function}

\begin{itemize}
  \item electronic:电子版,无空白页(本科选择这个即可,单双打印皆可,答辩时可以双面打印,纸质版终稿教务处要求单面打印)
  \item print-master-oneside:打印版,硕士,无空白页,单面打印,封面的 logo 和页眉的 logo 变成黑色
  \item print-master-twoside:打印版,硕士,有空白页,双面打印,封面的 logo 和页眉的 logo 变成黑色
  \item print-doctor:打印版,博士,有空白页,双面打印,封面的 logo 和页眉的 logo 变成黑色
\end{itemize}

注意:对于硕博的模板,\parencite{研究生院研究生学位论文规范} 中提到的“封面、原创性声明和使用授权书采用单面印刷,中英文扉页、摘要及后续内容采用双面印刷(硕士学位论文可以单面印刷)。” 实现方式即为在所需要单面印刷的后面加一页空白页,然后全部双面打印即可,这个就是上面键值的设计想法,用户不需要自己手动加空白页,只需要选择自己对应所需版本就会对应自动判断是否添加空白页。


\begin{function}[added = 2023-04-07,updated = 2023-05-10]{blind-version}
  \begin{ccnusyntax}[emph={[1]blind-version}]
    blind-version = (*<true|false|remove-partial-schoolname|remove-all-schoolname>*)
  \end{ccnusyntax}
  【本|硕|博】盲审版本。
  \begin{itemize}
    \item 【本|硕|博】\kvopt{blind-version}{true} 或省略 \opt{true} (即只写 \opt{blind-version}) 都表示开启盲审版本,校名、姓名、导师姓名、职称、版权声明页会去掉,并且去掉页眉。
    \item 【本|硕|博】\kvopt{blind-version}{false} 或者不写 \opt{blind-version} 键值表示不开启。
    \item 【本】\kvopt{blind-version}{remove-partial-schoolname}:去掉封面和版权声明页第一行中“华中师范”四个字(此键值只是为了实现邓国泰老师在旧模版中的做法)
    \item 【本】\kvopt{blind-version}{remove-all-schoolname}:去掉封面的“华中师范大学”校名以及版权声明页第一行中“华中师范大学”但保留版权声明页
  \end{itemize}
  \emph{注意致谢、参考文献等的盲审处理需要用户自行处理,本模版提供了 \tn{blind} 命令(见 \ref{subsubsec:豹尾}节)来处理致谢中的人名,不过一切以学院的要求为准。}
\end{function}


% \begin{function}{oneside,twoside}
%   指明论文的单双面模式,默认为 \opt{twoside}。该选项会影响每章
%   的开始位置,还会影响页眉样式。
% \end{function}

% 在双面模式(\opt{twoside})下,按照通常的排版惯例,每章应只从
% 奇数页(在右)开始;而在单页模式(\opt{oneside})下,则可以从
% 任意页面开始。本模板中,目录、摘要、符号表等均视作章,也按相同
% 方式排版。

% 双面模式下,正文部分偶数页(在左)的左页眉显示章标题,奇数页
% (在右)的右页眉显示节标题;前置部分的页眉按同样格式显示,但文字
% 均为对应标题(如“目录”、“摘要”等)。
% 而在单面模式下,正文部分则页面不分奇偶,均同时显示左、右页眉,
% 文字分别为章标题和节标题;前置部分只有中间页眉,显示对应标题。

\begin{function}{draft}
  \begin{ccnusyntax}[emph={[1]draft}]
    draft = (*<\TFF>*)
  \end{ccnusyntax}
  【本|硕|博】选择是否开启草稿模式,默认关闭。
\end{function}

草稿模式为全局选项,会影响到很多宏包的工作方式。
开启之后,主要的变化有:
\begin{itemize}
  \item 把行溢出的盒子显示为黑色方块;
  \item 不实际插入图片,只输出一个占位方框;
  \item 关闭超链接渲染,也不再生成 PDF 书签;
  \item 显示页面边框。
\end{itemize}

% \begin{function}[added=2018-01-31]{config}
%   \begin{ccnusyntax}[emph={[1]config}]
%     config = (*\marg{文件}*)
%   \end{ccnusyntax}
%   用户配置文件的文件名。默认为空,即不载入用户配置文件。
% \end{function}


\subsection{参数设置}

\begin{function}{\ccnusetup}
  \begin{ccnusyntax}[morekeywords={\ccnusetup}]
    \ccnusetup(*\marg{键值列表}*)
  \end{ccnusyntax}
  【本|硕|博】本模板提供了一系列选项,可由您自行配置。载入文档类之后,以下
  所有选项均可通过统一的命令 \cs{ccnusetup} 来设置。
\end{function}

\cs{ccnusetup} 的参数是一组由(英文)逗号隔开的选项列表,列表中的
选项通常是 \kvopt{\meta{key}}{\meta{value}} 的形式。部分选项的
\meta{value} 可以省略。对于同一项,后面的设置将会覆盖前面的设置。
在下文的说明中,将用\textbf{粗体}表示默认值。

\cs{ccnusetup} 采用 \LaTeX3 风格的键值设置,支持不同类型以及多种
层次的选项设定。键值列表中,“|=|”左右的空格不影响设置;但需注意,
参数列表中\emph{不可以出现空行}。

与模板选项相同,布尔型的参数可以省略 \kvopt{\meta{选项}}{true}
中的“\kvopt{}{true}”。

另有一些选项包含子选项,如 \opt{style} 和 \opt{info} 等。它们可以
按如下两种等价方式来设定:

\begin{latexexample}[morekeywords={\ccnusetup},
    emph={[1]style,cjk-font,info,title,title*,author,author*,department}]
  \ccnusetup{
    style = {cjk-font = mac},
    info  = {
      title      = {论动体的电动力学},
      title*     = {On the Electrodynamics of Moving Bodies},
      author     = {阿尔伯特·爱因斯坦},
      author*    = {Albert Einstein},
      department = {物理学系}
    }
  }
\end{latexexample}

或者

\begin{latexexample}[morekeywords={\ccnusetup},
    emph={[1]style,cjk-font,info,title,title*,author,author*,department}]
  \ccnusetup{
    style/cjk-font  = mac,
    info/title      = {论动体的电动力学},
    info/title*     = {On the Electrodynamics of Moving Bodies},
    info/author     = {阿尔伯特·爱因斯坦},
    info/author*    = {Albert Einstein},
    info/department = {物理学系}
  }
\end{latexexample}


注意 “|/|” 的前后均不可以出现空白字符。


\subsubsection{论文格式} \label{subsubsec:论文格式}

\begin{function}{style}
  \begin{ccnusyntax}[emph={[1]style}]
    style = (*\marg{键值列表}*)
    style/(*\meta{key}*) = (*\meta{value}*)
  \end{ccnusyntax}
  【本|硕|博】该选项包含许多子项目,用于设置论文格式。具体内容见下。
\end{function}


\begin{function}{style/font}
  \begin{ccnusyntax}[emph={[1]font}]
    font = (*<newtx|(times)|stixtwo|xits|tg|none>*)
  \end{ccnusyntax}
  【本|硕|博】设置西文字体(包括数学字体)。具体配置见表~\ref{tab:font}。
\end{function}

% \begin{table}[ht]
% \centering
% \begin{threeparttable}
%   \caption{西文字体配置}
%   \label{tab:font}
%   \small
%   \begin{tabular}{ccccc}
%     \toprule
%       & \textbf{正文字体} & \textbf{无衬线字体} & \textbf{等宽字体} & \textbf{数学字体} \\
%     \midrule
%       |garamond|        & EB Garamond         & Libertinus Sans & LM Mono\tnote{a} & Garamond Math   \\
%       |libertinus|      & Libertinus Serif    & Libertinus Sans & LM Mono          & Libertinus Math \\
%       |lm|              & LM Roman            & LM Sans         & LM Mono          & LM Math         \\
%       |palatino|        & TG Pagella\tnote{b} & Libertinus Sans & LM Mono          & TG Pagella Math \\
%       |times|           & XITS                & TG Heros        & TG Cursor        & XITS Math       \\
%       |times*|\tnote{c} & Times New Roman     & Arial           & Courier New      & XITS Math       \\
%     \bottomrule
%   \end{tabular}
%   \begin{tablenotes}
%     \item[a] “LM”是 Latin Modern 的缩写。
%     \item[b] “TG”是 TeX Gyre 的缩写。
%     \item[c] 本行中,Times New Roman、Arial 和 Courier New 是商业字体,
%       不包含在 \TeXLive{} 发行版中,但在 Windows 和 macOS 系统上均默认安装。
%   \end{tablenotes}
% \end{threeparttable}
% \end{table}
\begin{table}[htbp]
  \centering
  \begin{threeparttable}
    \caption{西文字体配置}
    \label{tab:font}
    \small
    \begin{tabular}{ccccc}
      \toprule
        & \textbf{正文字体} & \textbf{无衬线字体} & \textbf{等宽字体} & \textbf{数学字体} \\
      \midrule
      |stixtwo| & STIX Two Text   & TG Heros\tnote{a} & TG Cursor   & STIX Two Math \\
      |xits |   & XITS            & TG Heros & TG Cursor   & XITS Math \\
      |times|\tnote{b}   & Times New Roman & Arial    & Courier New & newtxmath \\
      |newtx|   & TG Termes   & TG Heros & TG Cursor   & newtxmath \\
      |tg|      & TG Termes       & TG Heros & TG Cursor   & TG Termes Math \\
      \bottomrule
    \end{tabular}
    \begin{tablenotes}
      % \item[a] “LM”是 Latin Modern 的缩写。
      \item[a] “TG”是 TeX Gyre 的缩写。
      \item[b] 本行中,Times New Roman、Arial 和 Courier New 是商业字体,
        不包含在 \TeXLive{} 发行版中,但在 Windows 和 macOS 系统上均默认安装。
    \end{tablenotes}
  \end{threeparttable}
  \end{table}


\begin{function}{style/cjk-font}
  \begin{ccnusyntax}[emph={[1]cjk-font}]
    cjk-font = (*<adobe|(fandol)|founder|mac|sinotype|sourcehan|windows|none>*)
  \end{ccnusyntax}
  【本|硕|博】设置中文字体。具体配置见表~\ref{tab:cjk-font}。
\end{function}

\begin{table}[htbp]
  \caption{中文字体配置}
  \label{tab:cjk-font}
  \centering
  \small
  \begin{tabular}{ccccc}
    \toprule
      & \textbf{正文字体(宋体)} & \textbf{无衬线字体(黑体)} & \textbf{等宽字体(仿宋)} & \textbf{楷体} \\
    \midrule
      |adobe|     & Adobe 宋体      & Adobe  黑体     & Adobe  仿宋  & Adobe 楷体      \\
      |fandol|    & Fandol 宋体     & Fandol 黑体     & Fandol 仿宋  & Fandol 楷体     \\
      |founder|   & 方正书宋        & 方正黑体        & 方正仿宋     & 方正楷体        \\
      |mac|       & (华文)宋体-简 & (华文)黑体-简 & 华文仿宋     & (华文)楷体-简 \\
      |sinotype|  & 华文宋体        & 华文黑体        & 华文仿宋     & 华文楷体        \\
      |sourcehan| & 思源宋体        & 思源黑体        & ---          & ---             \\
      |windows|   & (中易)宋体    & (中易)黑体    & (中易)仿宋 & (中易)楷体    \\
    \bottomrule
  \end{tabular}
\end{table}

启用 \kvopt{font}{none} 或 \kvopt{cjk-font}{none} 之后,模板将关闭
默认西文 / 中文字体设置。此时,您需要自行使用 \cs{setmainfont}、
\cs{setCJKmainfont}、\cs{setmathfont} 等命令来配置字体。


% \begin{function}{style/font-size}
%   \begin{ccnusyntax}[emph={[1]font-size}]
%     font-size = (*<(-4)|5>*)
%   \end{ccnusyntax}

%   设置论文的基础字号。
% \end{function}


\begin{function}{style/fullwidth-stop}
  \begin{ccnusyntax}[emph={[1]fullwidth-stop}]
    fullwidth-stop = (*<catcode|mapping|(false)>*)
  \end{ccnusyntax}
  【本|硕|博】选择是否把全角实心句点\FSFW 作为默认的句号形状。
  这种句号一般用于科技类文章,以避免与下标“$_o$”或“$_0$”混淆。
\end{function}

选择 \kvopt{fullwidth-stop}{catcode} 或 \opt{mapping} 后,都会实现上述效果。
有所不同的是,在选择 \opt{catcode} 后,只有\emph{显式的}\FSID 会被替换
为\FSFW;但在选择 \opt{mapping} 后,\emph{所有的}\FSID 都会被替换。例如,
如果您用宏保存了一些含有\FSID 的文字,那么在选择 \opt{catcode} 时,其中
的\FSID 不会将被替换为\FSFW。

选项 \kvopt{fullwidth-stop}{mapping} 只在 \XeTeX{} 下有效。

如果您在选择 \kvopt{fullwidth-stop}{mapping} 后仍需要临时显示\FSID,
可以按如下方法操作:
\begin{latexexample}[moretexcs={\CJKfontspec},emph={[1]Mapping}]
  % 请使用 XeTeX 编译
  % 外侧的花括号表示分组
  这是一个句号{\CJKfontspec{(*\meta{字体名}*)}[Mapping=full-stop]。}
\end{latexexample}

\begin{function}{style/footnote-style}
% 这里奇怪的东西是用来控制对齐的。ccnusyntax 会吃掉开头的几个空格,因此这里用 X 来占位。
  \begin{ccnusyntax}[emph={[1]footnote-style}]
    footnote-style = (*<plain|\\
      XXXX\mbox{}~~~~~~~~~~~~~~~~~libertinus|libertinus*|libertinus-sans|\\
      XXXX\mbox{}~~~~~~~~~~~~~~~~~pifont|pifont*|pifont-sans|pifont-sans*|\\
      XXXX\mbox{}~~~~~~~~~~~~~~~~~xits|xits-sans|xits-sans*>*)
  \end{ccnusyntax}
  【本|硕|博】设置脚注编号样式。西文字体设置会影响其默认取值(见
  表~\ref{tab:footnote-font})。因此,要使得该选项生效,需将其
  放置在 \opt{font} 选项之后。带有 |sans| 的为相应的无衬线字体
  版本;带有 |*| 的为阴文样式(即黑底白字)。
\end{function}

\begin{table}[ht]
  \caption{西文字体与脚注编号样式默认值的对应关系}
  \label{tab:footnote-font}
  \centering
  \small
  \begin{tabular}{ccccc}
    \toprule
      \textbf{西文字体设置} &
        |libertinus| & |lm|     & |palatino| & |times| \\
    \midrule
      \textbf{脚注编号样式默认值} &
        |libertinus| & |pifont| & |pifont|   & |xits|  \\
    \bottomrule
  \end{tabular}
\end{table}

\begin{function}{style/caption-labelstyle}
  \begin{ccnusyntax}[emph={[1]caption-labelstyle}]
    caption-labelstyle = (*<arabic|(hyphen)|dot>*)
  \end{ccnusyntax}
  【本|硕|博】图表标题 label 计数样式。

  \begin{itemize}
    \item \opt{arabic}:图1,图2... 跨 chapter 连续编号,即若上一个 chapter 的图编号为 4,下一个 chapter 的第一个图编号为5
    \item \opt{hyphen}:图1-1,图1-2,图2-1...其中图 $x$-$y$,$x$ 表示 chapter 的值,$y$ 表示该 chapter 的计数器值(通俗理解就是此 chapter 的第 $y$ 张图或表),$y$ 在新的 chapter 会自动重新开始计数
    \item \opt{dot}:图1.1,图1.2,图2.1...其中图 $x$.$y$,$x,y$ 的含义同上
  \end{itemize}
\end{function}

\begin{function}{style/caption-labelseperator}
  \begin{ccnusyntax}[emph={[1]caption-labelseperator}]
    caption-labelseperator = (*<(colon)|space>*)
  \end{ccnusyntax}
  【本|硕|博】图表标题 label 和标题内容之间的分隔符

  \begin{itemize}
    \item \opt{colon}:表示 「:␣」,即一个西文冒号加一个空格
    \item \opt{space}:表示 「␣␣」,即两个空格
  \end{itemize}
\end{function}

\begin{function}[added = 2022-06-04]{style/chapter-breakstyle}
  \begin{ccnusyntax}[emph={[1]chapter-breakstyle}]
    caption-labelseperator = (*<(continuous)|newpage>*)
  \end{ccnusyntax}
  【本】控制 \tn{chapter} 是否新起一页。根据 \parencite{研究生院研究生学位论文规范},硕博模版的 \tn{chapter} 是新起一页的,故此键值设计仅对本科模版生效。

  \begin{itemize}
    \item \opt{continuous}:不新起一页,接着上一个 \tn{chapter} 连续排版
    \item \opt{newpage}:\tn{chapter} 会新起一页开始排版
  \end{itemize}
\end{function}

\begin{function}[added = 2022-06-20]{style/show-head}
  \begin{ccnusyntax}[emph={[1]show-head}]
    show-head = (*<\TTF>*)
  \end{ccnusyntax}
  【硕|博】是否显示页眉。统一控制页眉 logo 和页眉线的显示与否。
\end{function}


\begin{function}[updated = 2022-06-20]{style/show-headlogo}
  \begin{ccnusyntax}[emph={[1]show-headlogo}]
    show-headlogo = (*<\TFF>*)
  \end{ccnusyntax}
  【硕|博】是否显示页眉 logo。具体 logo 样式见图~\ref{figure:headlogo}。
\end{function}

\begin{figure}[htbp]
  \centering
  \begin{minipage}{0.45\textwidth}
    \includegraphics[height = 2cm]{masterlogo.png}
  \end{minipage}
  \begin{minipage}{0.45\textwidth}
    \includegraphics[height = 2cm]{doctorlogo.png}
  \end{minipage}
  \caption{headlogo}
  \label{figure:headlogo}
\end{figure}

\begin{function}{style/headline}
  \begin{ccnusyntax}[emph={[1]headline}]
    headline = (*<single|double|thin-thick|thick-thin|(none)>*)
  \end{ccnusyntax}
  【硕|博】页眉线的样式。

  \begin{itemize}
    \item \opt{single}:一条线
    \item \opt{double}:两条线,一样粗细
    \item \opt{thin-thick}:两条线,上细下粗(文武线)
    \item \opt{thick-thin}:两条线,上粗下细(武文线)
    \item \opt{none}:页眉没有线
  \end{itemize}
\end{function}

\begin{function}[added = 2022-6-19]{style/head-scope}
  \begin{ccnusyntax}[emph={[1]head-scope}]
    head-scope = (*<all|(main)>*)
  \end{ccnusyntax}
  【硕|博】页眉线的作用范围。

  \begin{itemize}
    \item \opt{all}:除了全文的第一页封面,其他页面均有页眉。
    \item \opt{main}:正文开始才有页眉。
  \end{itemize}
\end{function}

\begin{function}[added = 2022-06-18]{style/keywords-newline}
  \begin{ccnusyntax}[emph={[1]keywords-newline}]
    keywords-newline = (*\TTF*)(硕|博)
    keywords-newline = (*\TFF*)(本)
  \end{ccnusyntax}
  【本|硕|博】(中英统一控制)摘要和关键词之间是否空一行。

  \begin{itemize}
    \item \opt{true}:摘要和关键词之间空一行
    \item \opt{false}:摘要新起一段后为关键词
  \end{itemize}
\end{function}

\begin{function}[added = 2022-06-19]{style/listoffigures-show,style/listoftables-show}
  \begin{ccnusyntax}[emph={[1]listoffigures-show,listoftables-show}]
    listoffigures-show = (*\TTF*)(硕|博)
    listoftables-show = (*\TTF*)(硕|博)
    listoffigures-show = (*\TFF*)(本)
    listoftables-show = (*\TFF*)(本)
  \end{ccnusyntax}
  【本|硕|博】控制是否显示图表目录。若都显示则紧跟文章目录后,且图目录在表目录前。
\end{function}

\begin{function}[added = 2022-06-19]{style/listoffigures-name,style/listoftables-name}
  \begin{ccnusyntax}[emph={[1]listoffigures-name,listoftables-name}]
    listoffigures-name = (*<\marg{图目录的节标题}>*)
    listoftables-name = (*<\marg{表目录的节标题}>*)
  \end{ccnusyntax}
  【本|硕|博】图表目录的节标题。分别默认为 |插 \quad 图| 和 |表 \quad 格|
\end{function}


\begin{function}{style/hyperlink}
  \begin{ccnusyntax}[emph={[1]hyperlink}]
    hyperlink = (*<color|(none)>*)
  \end{ccnusyntax}
  【本|硕|博】设置超链接样式。\opt{color} 表示用彩色显示超链接;\opt{none} 表示没有特殊装饰,可用于生成最终的打印版文稿。
\end{function}

\begin{function}{style/hyperlink-color}
  \begin{ccnusyntax}[emph={[1]hyperlink-color}]
    hyperlink-color = (*<(default)|classic|material|graylevel|prl>*)
  \end{ccnusyntax}
  【本|硕|博】设置超链接颜色。该选项在 \kvopt{hyperlink}{none} 时无效。
  各选项所代表的颜色见表~\ref{tab:hyperlink-color}。
\end{function}


\begin{table}[ht]
\centering
\small

\newcommand\linkcolorexam[3]{%
  {\small 图~\textcolor[HTML]{#1}{1-2},
    (\textcolor[HTML]{#1}{3.4})~式} &
  {\small \textcolor[HTML]{#2}{\texttt{https://g.cn}}} &
  {\small 文献~[\textcolor[HTML]{#3}{1}],
    (\textcolor[HTML]{#3}{Knuth~1986})}}
\begin{threeparttable}
\caption{预定义的超链接颜色方案}
\label{tab:hyperlink-color}
\begin{tabular}{c*{3}{>{\hspace{0.2cm}}c<{\hspace{0.2cm}}}}
  \toprule
    \textbf{选项} & \textbf{链接} & \textbf{URL} & \textbf{引用} \\

  \midrule
    \opt{default}            & \linkcolorexam{990000}{0000B2}{007F00} \\
    \opt{classic}            & \linkcolorexam{FF0000}{0000FF}{00FF00} \\
    \opt{material}\tnote{a}  & \linkcolorexam{E91E63}{009688}{4CAF50} \\
    \opt{graylevel}\tnote{a} & \linkcolorexam{616161}{616161}{616161} \\
    \opt{prl}\tnote{b}       & \linkcolorexam{2D3092}{2D3092}{2D3092} \\
  \bottomrule
\end{tabular}
\begin{tablenotes}

  \item[a] 取自 Material 色彩方案
    (见 \url{https://material.io/guidelines/style/color.html})。
  \item[b] \textit{Physical Review Letter} 杂志配色。

\end{tablenotes}
\end{threeparttable}
\end{table}



% \begin{function}{style/bib-backend}
%   \begin{ccnusyntax}[emph={[1]bib-backend}]
%     bib-backend = (*<bibtex|biblatex>*)
%   \end{ccnusyntax}

%   选择参考文献的支持方式。选择 \opt{bibtex} 后,将使用 \BibTeX{}
%   处理文献,样式由 \pkg{natbib} 宏包负责;选择 \opt{biblatex} 后,
%   将使用 \biber{} 处理文献,样式则由 \pkg{biblatex} 宏包负责。
% \end{function}

\begin{function}[updated = 2023-03-11]{style/bib-style}
  \begin{ccnusyntax}[emph={[1]bib-style}]
    bib-style = (*<(ccnu-bachelor-numerical)|ccnu-bachelor-author-year|ccnu-master|ccnu-doctor|gb7714-2015|gb7714-2015ay>*)
  \end{ccnusyntax}
  【本|硕|博】设置参考文献样式。
  \begin{itemize}
    \item \opt{ccnu-bachelor-numerical}:参考 \parencite{本科生院毕业论文注释与参考文献著录格式} 定制的参考文献格式,顺序编码制
    \item \opt{ccnu-bachelor-author-year}:和 \opt{ccnu-bachelor-numerical} 格式相同,著者—出版年制
    \item \opt{ccnu-master}:国家标准 GB/T 7714--2015 的顺序编码制
    \item \opt{ccnu-doctor}:国家标准 GB/T 7714--2015 的顺序编码制
    \item \opt{gb7714-2015}:国家标准 GB/T 7714--2015 的顺序编码制
    \item \opt{gb7714-2015ay}:国家标准 GB/T 7714--2015 的作者-年制
  \end{itemize}
  % \opt{author-year} 和 \opt{numerical} 分别对应国家标准 GB/T 7714--2015 \cite{gb-t-7714-2015} 中的著者—出版年制和顺序编码制。
  % 选择 \meta{其他样式} 时,如果 \kvopt{bib-backend}{bibtex},需保证相应的 \file{.bst} 格式文件能被调用;而如果\kvopt{bib-backend}{biblatex},则需保证相应的 \file{.bbx} 格式文件能被调用。
\end{function}

% \begin{function}{style/cite-style}
%   \begin{ccnusyntax}[emph={[1]cite-style}]
%     cite-style = (*\marg{引用样式}*)
%   \end{ccnusyntax}
%   选择引用格式。默认为空,即与参考文献样式(著者—出版年制或顺序
%   编码制)保持一致。如果手动填写,需保证相应的 \file{.cbx} 格式文件
%   能被调用。该选项在 \kvopt{bib-backend}{bibtex} 时无效。
% \end{function}

\begin{function}{style/bib-resource}
  \begin{ccnusyntax}[emph={[1]bib-resource}]
    bib-resource = (*\marg{文件}*)
  \end{ccnusyntax}
  【本|硕|博】参考文献数据源。可以是单个文件,也可以是用英文逗号隔开的一组文件。\emph{必须明确给出 \file{.bib} 后缀名}。默认为 \file{CCNUthesis-main.bib}。
\end{function}

\begin{function}[added = 2024-02-22]{style/bib-keyval}
  \begin{ccnusyntax}[emph={[1]bib-keyval}]
    bib-keyval = (*\marg{键值对}*)
  \end{ccnusyntax}
  【本|硕|博】用于接入 \pkg{biblatex} 宏包的选项,内容为键值对,键值对之间用西文逗号隔开。比如 |bib-keyval={doi=false}| 表示去掉参考文献中的 DOI 信息。 具体可输入的选项请参考 \pkg{biblatex-gb7714-2015} 宏包的文档(命令行输入 |texdoc biblatex-gb7714-2015|)。
\end{function}

\begin{function}[added = 2024-03-07]{style/cover_ii_only_title_content}
  \begin{ccnusyntax}[emph={[1]cover_ii_only_title_content}]
    cover_ii_only_title_content = (*\TFF*)
  \end{ccnusyntax}
  【硕|博】第二个封面的标题是否去掉“论文标题:”并且将标题居中。默认为 |false|,即保留“论文标题:”并且标题左对齐。
\end{function}



\subsubsection{信息录入} \label{subsubsec:信息录入}

\begin{function}{info}
  \begin{ccnusyntax}[emph={[1]info}]
    info = (*\marg{键值列表}*)
    info/(*\meta{key}*) = (*\meta{value}*)
  \end{ccnusyntax}
  【本|硕|博】该选项包含许多子项目用于录入论文信息。具体内容见下。以下带“|*|”的项目表示对应的\emph{英文}字段或\emph{拼音}字段。
\end{function}

\begin{function}{info/cover-type}
  \begin{ccnusyntax}[emph={[1]cover-type}]
    cover-type  = (*<word|(math)>*)
  \end{ccnusyntax}
  【本|硕|博】封面类型。\opt{word} 表示几乎完全按照教务处给的 word 版本模版处理;\opt{math} 表示华中师范大学数学与统计学学院在 word 版本模版基础上进行部分细节调整。
\end{function}

\begin{function}[added = 2022-12-10]{info/title-line-type}
  \begin{ccnusyntax}[emph={[1]title-line-type}]
    title-line-type  = (*<variable|(constant)>*)
  \end{ccnusyntax}
  【本|硕|博】封面中文标题下划线类型。\opt{variable} 表示下划线只有文字出现的地方才有,也就是不会出现空白处有下划线的情况;\opt{constant} 表示下划线是恒定长度,可能会出现空白处处有下划线。
\end{function}

\begin{function}{info/title,info/title*}
  \begin{ccnusyntax}[emph={[1]title,title*}]
    title  = (*\marg{中文标题}*)
    title* = (*\marg{英文标题}*)
  \end{ccnusyntax}
  【本|硕|博】论文标题。默认会在约 21 个汉字(本科,硕博在 14 个汉字)字宽处自动断行,但为了语义的连贯以及排版的美观,如果您的标题长于一行,可以根据语义使用“|\\|”手动断行。如果您的标题中有副标题,使用“|\\|”手动断行并以“——”开头,如
\begin{latexexample}
  title = {
    如何使用 \LaTeX 编写论文模版 \\
    ——以华中师范大学为例
  }
\end{latexexample}
\end{function}

\begin{function}{info/author,info/author*}
  \begin{ccnusyntax}[emph={[1]author,author*}]
    author  = (*\marg{姓名}*)【本|硕|博】
    author* = (*\marg{姓名拼音)}*)【硕|博】
  \end{ccnusyntax}
  作者姓名。
\end{function}

\begin{function}{info/student-id}
  \begin{ccnusyntax}[emph={[1]student-id}]
    student-id  = (*\marg{学号}*)
  \end{ccnusyntax}
  【本】作者学号。
\end{function}

\begin{function}{info/level}
  \begin{ccnusyntax}[emph={[1]level}]
    level  = (*\marg{年级}*)
  \end{ccnusyntax}
  【本】年级。
\end{function}

\begin{function}{info/supervisor,info/supervisor*-name,info/supervisor*-academic-title}
  \begin{ccnusyntax}[emph={[1]supervisor,supervisor*-name,supervisor*-academic-title}]
    supervisor = (*\marg{姓名}*)【本|硕|博】
    supervisor*-name = (*\marg{姓名拼音}*)【硕|博】
    supervisor*-academic-title = (*\marg{职称英文}*)【硕|博】
  \end{ccnusyntax}
  导师姓名、职称。
\end{function}

\begin{function}{info/department,info/department*}
  \begin{ccnusyntax}[emph={[1]department,department*}]
    department = (*\marg{学院名称}*)【本|硕|博】
    department* = (*\marg{学院英文名称}*)【硕|博】
  \end{ccnusyntax}
  学院名称。
\end{function}

\begin{function}{info/major,info/major*}
  \begin{ccnusyntax}[emph={[1]major,major*}]
    major = (*\marg{专业名称}*)【本|硕|博】
    major* = (*\marg{专业英文名称}*)【硕|博】
  \end{ccnusyntax}
  专业名称。
\end{function}

\begin{function}{info/research-area,info/research-area*}
  \begin{ccnusyntax}[emph={[1]research-area,research-area*}]
    research-area = (*\marg{研究方向名称}*)【硕|博】
    research-area* = (*\marg{研究方向英文名称}*)【博】
  \end{ccnusyntax}
  作者研究方向。
\end{function}

\begin{function}{info/degree-type,info/degree-type*}
  \begin{ccnusyntax}[emph={[1]degree-type,degree-type*}]
    degree-type = (*\marg{申请学位学生类别}*)【硕|博】
    degree-type* = (*\marg{英文申请学位学生类别缩写}*)【硕】
  \end{ccnusyntax}
  申请学位学生类别。如
  \begin{itemize}
    \item 教育硕士|应用统计硕士|全日制硕士|同等学力人员|高校教师在职攻读硕士学位人员|专业学位人员
    \item 博士
  \end{itemize}
  英文缩写比如:M.S.
\end{function}


\begin{function}{info/keywords,info/keywords*}
  \begin{ccnusyntax}[emph={[1]keywords,keywords*}]
    keywords  = (*\marg{中文关键字}*)
    keywords* = (*\marg{英文关键字}*)
  \end{ccnusyntax}
  【本|硕|博】关键字列表。各关键字之间需使用 \emph{英文逗号} 隔开。
\end{function}


\begin{function}{info/year,info/month}
  \begin{ccnusyntax}[emph={[1]year,month}]
    year = (*\marg{年}*)
    month = (*\marg{月份}*)
  \end{ccnusyntax}
  【本|硕|博】论文完成的年月。默认值为文档编译时的年和月。
\end{function}



\subsection{正文编写}

\begin{quotation}
  喬孟符(吉)博學多能,以樂府稱。嘗云:「作樂府亦有法,
  曰\CJKunderdot{鳳頭、豬肚、豹尾}六字是也。」大概起要美麗,中要浩蕩,
  結要響亮。尤貴在首尾貫穿,意思清新。苟能若是,斯可以言樂府矣。
\end{quotation}
\hfill ——陶宗儀《南村輟耕錄·作今樂府法》


\subsubsection{凤头}

\begin{function}{\frontmatter}
  【本|硕|博】声明前置部分开始。
\end{function}

在本模板中,前置部分包含目录、中英文摘要以及符号表等。硕博模版的前置部分的页码采用小写罗马字母,并且与正文分开计数;本科模版采用阿拉伯数字,并与正文连续计数。本硕博模版目录均无页码。

目录会自动生成,无需用户手动控制。

\begin{function}{abstract}
  \begin{ccnusyntax}[emph={[2]abstract}]
    % abstract.tex
    \begin{abstract}
      (*\meta{中文摘要}*)
    \end{abstract}
  \end{ccnusyntax}
  【本|硕|博】中文摘要。
\end{function}
\begin{function}{abstract*}
  \begin{ccnusyntax}[emph={[2]abstract*}]
    % abstract.tex
    \begin{abstract*}
      (*\meta{英文摘要}*)
    \end{abstract*}
  \end{ccnusyntax}
  【本|硕|博】英文摘要。
\end{function}


\begin{function}{notation}
  \begin{ccnusyntax}[emph={[2]notation}]
    \begin{notation}(*\oarg{列格式说明}*)
      (*\meta{符号 1}*)  &  (*\meta{说明}*)  \\
      (*\meta{符号 2}*)  &  (*\meta{说明}*)  \\
      (*\phantom{\meta{符号 $n$}}*)  (*$\vdots$*)
      (*\meta{符号\ \kern-0.1em$n$}*)  &  (*\meta{说明}*)
    \end{notation}
  \end{ccnusyntax}
  【本|硕|博】符号表。基于 \pkg{tabularray} 宏包的 \env{longtblr} 环境,可选参数 \meta{列格式说明} 和 \env{longtblr} 环境的可选参数接口相同,并设置默认为
\begin{latexexample}
  width   = 0.3\textwidth,
  colspec = {X[1,c]X[1,c]}
\end{latexexample}
  如果效果不满意,请您命令行输入 \cmd{texdoc tabularray} 自行查阅 \pkg{tabularray} 宏包的用户手册了解更多使用参数和细节。
\end{function}

\subsubsection{猪肚}

\begin{function}{\mainmatter}
  【本|硕|博】声明主体部分开始。
\end{function}

主体部分是论文的核心,您可以分章节撰写。如有需求,也可以采用
多文件编译的方式。主体部分的页码采用阿拉伯数字。

\begin{function}{\footnote}
  \begin{ccnusyntax}[deletetexcs={\footnote},morekeywords={\footnote}]
    \footnote(*\marg{脚注文字}*)
  \end{ccnusyntax}
  【本|硕|博】插入脚注。脚注编号样式可利用 \opt{style/footnote-style} 选项控制,
  具体见 \ref{subsubsec:论文格式}~小节。

  需要注意的是,\parencite{本科生院毕业论文注释与参考文献著录格式} 中指出
  \begin{itemize}
    \item 文科术科的论文注释使用脚注
    \item 理工科的论文注释\textcolor{red}{\emph{不使用}}脚注
  \end{itemize}
\end{function}

\begin{function}{\caption}
  \begin{ccnusyntax}[deletetexcs={\caption},morekeywords={\caption}]
    \caption(*\marg{图表标题}*)
  \end{ccnusyntax}
  【本|硕|博】插入图表标题。
\end{function}

按照排版惯例,建议您将表格的标题放置在绘制表格的命令之前,
而将图片的标题放置在绘图或插图的命令之后。另需注意,
\tn{caption} 命令必须放置在浮动体环境(如 \env{table} 和
\env{figure})中。


\paragraph{参考文献引用}\label{para:参考文献引用}

\begin{function}{\parencite}
  \begin{ccnusyntax}[deletetexcs={\parencite},morekeywords={\parencite}]
    \parencite(*\marg{文献标签}*)
    \parencite(*\oarg{postnote}\marg{文献标签}*)
    \parencite(*\oarg{prenote}\oarg{postnote}\marg{文献标签}*)
  \end{ccnusyntax}
\end{function}

\begin{function}{\cite}
  \begin{ccnusyntax}[deletetexcs={\cite},morekeywords={\cite}]
    \cite(*\marg{文献标签}*)
    \cite(*\oarg{postnote}\marg{文献标签}*)
    \cite(*\oarg{prenote}\oarg{postnote}\marg{文献标签}*)
  \end{ccnusyntax}
\end{function}

【本|硕|博】插入所引用的文献。\meta{prenote} 和 \meta{posnote} 由名称可看出,一个是出现在前方,一个出现在后方。绝大部分情况只需用到可选参数 \meta{postnote},可用来标注引文的页码或引用的定理。

\tn{parencite} 是行内引用,\tn{cite} 是上标引用。通常情况下

\begin{itemize}
  \item 下面两种情况要用行内引用 \tn{parencite}:
    \begin{enumerate}
      \item 去掉这个引用句子结构不完整,比如“定理证明可参看[1]”
      \item 英文文献的引用
    \end{enumerate}
  \item 下面情况用上标引用 \tn{cite}:
    \begin{itemize}
      \item 去掉这个引用后句子结构仍然完整,比如“作者提到,CCNUthesis 是一个非常好的好模版$^{[1]}$。”
    \end{itemize}
\end{itemize}

如果您对上述的说法觉得不赞同或有所补充,欢迎提出!(可按~\ref{subsec:提issues} 节的链接到 \cls{CCNUthesis} 项目主页提 issues)

效果举例:

\begin{latexexample}
  % CCNUthesis-main.bib
  % @book{feynman2011,
  %   title     = {The Feynman lectures on physics, Vol. I: The new millennium edition: mainly mechanics, radiation, and heat},
  %   author    = {Feynman, Richard P and Leighton, Robert B and Sands, Matthew},
  %   volume    = {1},
  %   year      = {2011},
  %   publisher = {Basic books},
  %   pages     = {2-8},
  %   edition   = {7},
  %   url       = {https://arxiv.org/abs/2201.00067}
  % }

  % main.tex
  英文文献 \parencite{feynman2011}

  英文文献 \parencite[12]{feynman2011}

  英文文献 \parencite[Thm1]{feynman2011}

  英文文献 \parencite[12][Thm1]{feynman2011}

  英文文献 \cite{feynman2011}

  英文文献 \cite[12]{feynman2011}

  英文文献 \cite[Thm1]{feynman2011}

  英文文献 \cite[12][Thm1]{feynman2011}
\end{latexexample}

最终效果见图~\ref{figure:cite-parencite效果}(其中“4”仅为测试效果,具体取决于 \cmd{bib-style} 的样式及引用顺序等)。

\begin{figure}[htbp]
  \centering
  \includegraphics[width = 0.4\textwidth]{cite-parencite效果.png}
  \caption{\tn{cite} 和 \tn{parencite} 的效果}
  \label{figure:cite-parencite效果}
\end{figure}


\paragraph{定理类环境}


\begin{function}{axiom,counterexample,claim,corollary,conjecture,definition,example,lemma,property,proof,proposition,question,remark,theorem}
  \begin{ccnusyntax}[emph={[2]proof}]
    \begin{proof}(*\oarg{小标题}*)
      (*\meta{证明过程}*)
    \end{proof}
  \end{ccnusyntax}
  【本|硕|博】一系列预定义的数学环境。具体含义见表~\ref{tab:theorem}。
\end{function}

\begin{table}[ht]
  \caption{预定义的数学环境} \label{tab:theorem}
  \centering
  \small
  \begin{tblr}{
    width = \textwidth,
    columns = {c},
    hline{1,2,Z} = {solid}
  }
    \textbf{名称} &
      \env{axiom} & \env{counterexample} & \env{claim} & \env{corollary} & \env{conjecture} & \env{definition} & \env{example} \\
    \textbf{含义} &
      公理 & 反例 & 断言 & 推论 &
      猜想 & 定义 & 例 \\
  \end{tblr}

  \medskip

  \begin{tblr}{
    width = \textwidth,
    columns = {c},
    hline{1,2,Z} = {solid}
  }
    \textbf{名称} &
      \env{lemma} & \env{property} & \env{proof} & \env{proposition} & \env{question} & \env{remark} & \env{theorem} \\
    \textbf{含义} &
      引理 & 性质 & 证明 & 命题 &
      问题 & 注  & 定理 \\
  \end{tblr}
\end{table}

\begin{function}{\qedhere}
  【本|硕|博】证明环境(\env{proof})的最后会添加证毕符号“$\square$”。对于证明如果以公式结尾或其它某些情况时“$\square$”可能会出现在新的空白行的行尾,如果想要“$\square$”出现在“有内容的”行尾,可以在想要出现的地方使用 \tn{qedhere} 命令。
\end{function}

比如您可以在模版中测试下面代码查看效果:

\begin{latexexample}
  \begin{proof}
    \[
      x^2
    \]
  \end{proof}
  \begin{proof}
    \[
      x^2  \qedhere
    \]
  \end{proof}
\end{latexexample}


\begin{function}[added = 2022-06-04]{\ccnunewtheorem,\ccnunewtheorem*}
  \begin{ccnusyntax}[deletetexcs={\ccnunewtheorem,\ccnunewtheorem*},morekeywords={\ccnunewtheorem,\ccnunewtheorem*}]
    \ccnunewtheorem(*\oarg{计数器样式设置}\marg{环境中文名称}\marg{环境名}*)
    \ccnunewtheorem*(*\oarg{计数器样式设置}\marg{环境中文名称}\marg{环境名}*)
  \end{ccnusyntax}
  【本|硕|博】自定义定理类环境的命令。带星号的表示新定义的环境无计数器,类似于 \env{proof} 环境。
\end{function}

该命令的主要用法有下面四种:
\begin{latexexample}
  \ccnunewtheorem{测试}{test}
  \ccnunewtheorem*{测试试}{testt}
  \ccnunewtheorem[sibling = theorem]{测试试试}{testtt}
  \ccnunewtheorem[within = chapter]{测试试试试}{testttt}
\end{latexexample}

\begin{itemize}
  \item |\ccnunewtheorem{测试}{test}| 定义了一个 \env{test} 环境,label 名为 \env{测试}。环境自己用自己的计数器,并且跨 chapter 连续编号
  \item |\ccnunewtheorem*{测试试}{testt}| 定义了一个 \env{testt} 环境,label 名为 \env{测试试}。环境不编号。
  \item |\ccnunewtheorem[sibling = theorem]{测试试试}{testtt}| 定义了一个 \env{testtt} 环境,label 名为 \env{测试试试}。环境和 theorem 环境共用一个计数器。
  \item |\ccnunewtheorem[within = chapter]{测试试试试}{testttt}| 定义了一个 \env{testttt} 环境,label 名为 \env{测试试试试}。环境计数器的值和章节有关,形如 $x.y$,$x$ 表示章节的计数器的值,子计数器 $y$ 随新章节清零重新计数。
\end{itemize}

您可以看图~\ref{figure:ccnunewtheorem} 的效果来更好地理解四种用法的效果,

\begin{figure}[htbp]
  \centering
  \includegraphics[width = 0.7\textwidth]{ccnunewtheorem.png}
  \caption{\tn{ccnunewtheorem} 示例}
  \label{figure:ccnunewtheorem}
\end{figure}



\subsubsection{豹尾}  \label{subsubsec:豹尾}

\begin{function}{\backmatter}
  【本|硕|博】声明后置部分开始。
\end{function}

后置部分包含参考文献、声明页等。

\begin{function}{\printbibliography}
  \begin{ccnusyntax}[morekeywords={\printbibliography}]
    \printbibliography(*\oarg{选项}*)
  \end{ccnusyntax}
  【本|硕|博】打印参考文献列表。该命令由 \pkg{biblatex} 宏包直接提供,可用选项请参阅其文档。一般来说,用户不需要做任何改动。
\end{function}

\begin{function}{\acknowledgements}
  【本|硕|博】开启致谢章节。

\begin{latexexample}
  % acknowledgements.tex
  \acknowledgements
  
  <致谢内容>
\end{latexexample}
\end{function}

\begin{function}[added = 2023-05-10]{\blind}
  【本|硕|博】在致谢或研究成果中需要去掉姓名、信息等的命令。

\begin{latexexample}
  我要感谢 CCNUthesis 的模版开发者 \blind{夏康玮}。
\end{latexexample}
如果开启盲审选项(见 \ref{subsec:模版选项} 节的 \opt{blind-version} 键值),即当文档类选项中(\tn{documentclass} 的可选参数中)是以下四种情况之一:
\begin{enumerate}
  \item \opt{blind-version}
  \item \kvopt{blind-version}{true}
  \item \kvopt{blind-version}{remove-partial-schoolname}
  \item \kvopt{blind-version}{remove-all-schoolname}
\end{enumerate}
则编译效果为
\begin{latexexample}
  我要感谢 CCNUthesis 的模版开发者***。
\end{latexexample}
否则就原封不动输出
\begin{latexexample}
  我要感谢 CCNUthesis 的模版开发者夏康玮。
\end{latexexample}
\end{function}

\begin{function}{signature}
  \begin{ccnusyntax}
    \begin{signature}
      <落款>
    \end{signature}
  \end{ccnusyntax}
  【本|硕|博】落款签名。整体右对齐,内部居中对齐。可用|\\|换行
\begin{latexexample}
  \begin{signature}
    夏康玮 \\
    2022年6月5日于珞珈山
  \end{signature}
\end{latexexample}
\end{function}


\begin{function}{\appendix}
  【本|硕|博】开启附录章节。
  
\begin{latexexample}
  % appendix.tex
  \appendix
  
  \chapter{<附录标题>}
  <附录内容>

  \chapter{<附录标题>}
  <附录内容>
\end{latexexample}

  附录和正文的章节层级相同,也是 \tn{chapter} 开始。需注意:\emph{硕博模版的附录在致谢前,本科模版的附录在致谢后。}

  根据 \parencite{研究生院研究生学位论文规范} 的要求“附录另起一页,“附录”两字居中,中间空两格,三号黑体加粗。如有多个附录,可用附录1、附录2区别并加以排序。” 模版已经做了自动化处理,即
  \begin{itemize}
    \item 如果 \file{appendix.tex} 中只有一个 \tn{chapter}\marg{内容} 的话,章节标题和目录中显示“附录 \meta{内容}”
    \item 如果 \file{appendix.tex} 中有两个及两个以上的 \tn{chapter}\marg{内容} 的话,章节标题和目录中显示“附录A \meta{内容}”、“附录B \meta{内容}”
  \end{itemize}
  但和一般的交叉引用不同,附录的此自动化处理,一般来说可能需要编译两次甚至三次,这个取决于用户在何时进行编译(比如编译之后又写了一个 \tn{chapter}\marg{内容} 的话,就需要三次编译才可以达到正确的章节标题和目录内容),但是三次(只要附录外的内容没问题,附录内没有报错)一定可以达到正确的编译效果。
\end{function}


\begin{function}{\publication}
  【博】开启“攻读学位期间取得的研究成果”章节。
\end{function}

\begin{function}{publications}
  \begin{ccnusyntax}[emph={[2]publications}]
    \begin{publications}
      \item (*\meta{研究成果1}*)
      \item (*\meta{研究成果2}*)
      ...
    \end{publications}
  \end{ccnusyntax}
  【博】研究成果列表环境。研究成果的格式等需要用户自行输入,无法像参考文献一样自动化,具体的字体字样等命令请自行查阅 \href{https://ctan.math.illinois.edu/info/lshort/chinese/lshort-zh-cn.pdf}{lshort-zh-cn}。
\end{function}

博士模版使用只需要取消 \file{main.tex} 中的
\begin{latexexample}
  % \include{./back/publications.tex}
\end{latexexample}
的代码注释,并且在 \file{publications.tex} 文件中输入相应内容然后编译即可。

\begin{function}{choices}
  \begin{ccnusyntax}[emph={[2]choices}]
    \begin{choices}(*\oarg{可选参数}*)
      \item (*\meta{选项1}*)
      \item (*\meta{选项2}*)
      ...
    \end{choices}
  \end{ccnusyntax}
  【本|硕|博】选项排版环境。对于有排版试卷、问卷等需求的用户,此环境能方便地帮您进行任意个选项的排版,并可以方便地调整选项 label 的样式。
\end{function}

用法和列表环境相同,使用 \tn{item} 分隔选项。label 的样式支持

\begin{itemize}
  \item arabic(阿拉伯数字)
  \item alph(小写英文)
  \item Alph(大写英文)
  \item roman(小写罗马数字)
  \item Roman(大写罗马数字)
  \item circlednumber(带圈数字)
\end{itemize}

如
\begin{latexexample}
  \begin{choices}[label = \arabic*)]
    \item 选项1
    \item 选项2
    \item 选项3
    \item 选项4
  \end{choices}
  
  \begin{choices}[label = (\alph*]
    \item 选项1
    \item 选项2
    \item 选项3
    \item 选项4
  \end{choices}
  
  \begin{choices}[label = \Alph*.]
    \item 选项1
    \item 选项2
    \item 选项3
    \item 选项4
  \end{choices}
  
  \begin{choices}[label = \roman*:]
    \item 选项1
    \item 选项2
    \item 选项3
    \item 选项4
  \end{choices}
  
  \begin{choices}[label = \Roman*-]
    \item 选项1
    \item 选项2
    \item 选项3
    \item 选项4
  \end{choices}
  
  \begin{choices}[label = \circlednumber*]
    \item 选项1
    \item 选项2
    \item 选项3
    \item 选项4
    \item 选项5
    \item 选项6
    \item 选项7
    \item 选项8
  \end{choices}
\end{latexexample}


还可以修改 \opt{columns} 键值来决定每行排多少个
\begin{latexexample}
  \begin{choices}[
    columns = 3,            % 手动控制每行多少个选项,否则自己根据宽度自动排版
    label   = (\arabic*)      % label 的样式,支持 arabic, alph, Alph, roman, Roman, circlednumber
  ]
    \item 选项1
    \item 选项2
    \item 选项3
    \item 选项4
    \item 选项5
    \item 选项6
  \end{choices}
\end{latexexample}

具体效果可见~\ref{figure:choices环境示例}。更多关于 \env{choices} 环境的精细调整可以查看 \url{https://gitee.com/zepinglee/exam-zh}。

\begin{figure}[htbp]
  \centering
  \includegraphics[width = 0.9\textwidth]{choices环境.png}
  \caption{\env{choices} 环境示例}
  \label{figure:choices环境示例}
\end{figure}