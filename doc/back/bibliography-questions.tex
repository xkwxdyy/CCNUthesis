% !TeX root = ../CCNUthesis.tex

\subsection{参考文献的相关问题}

\subsubsection{ bib 文件如何使用}

主要参考 \href{https://ctan.math.illinois.edu/info/lshort/chinese/lshort-zh-cn.pdf}{lshort-zh-cn} 的 6.1.2 节,但在这多说一点,参考文献的数据库(\file{.bib} 文件)中文献条目的获取方式,可以通过 google 学术等搜索文献,然后“导出 bib 格式”,或是用 zotero 等软件管理参考文献进行导出 bib 格式。

但要注意的事,这样的导出方式,经常出现文献的类型不匹配,所以需要自己根据~\ref{subsec:参考文献的类型示例} 节的数据类型自行甄别。


\subsubsection{作者姓氏的多音字问题}

在设置了
\begin{latexexample}
  \ccnusetup{
    style = {
      bib-style = ccnu-bachelor-author-year
    }
  }
\end{latexexample}
后,即此时参考文献的格式为作者-年制,但是有的时候会发现一些中文作者文献的顺序“出乎意料”,比如第一作者为“曾xx”的文献竟然会排在非 Z 开头的文献前,这主要是姓氏的多音字问题,比如“曾”就有“\pinyin{ceng2}”和“\pinyin{zeng1}”两种读音,而默认会取字母靠前的作为读音,就导致了自动把“曾xx”的“曾”作为“\pinyin{ceng2}”进行排序所以才有了上述的结果。

那怎么解决呢?这个时候需要用到 \cmd{key} 域

\begin{latexexample}[emph={key}]
  @book{曾谨言2013量子力学,
    title     = {量子力学: 卷 I},
    author    = {曾谨言},
    year      = {2013},
    publisher = {科学出版社},
    address   = {北京},
    pages     = {1-10},
    edition   = {2},
    key       = {zeng1jin3yan3}
  }
\end{latexexample}

内容就是第一作者名的拼音,用法一看这个示例便知(和 \pkg{xpinyin} 宏包的使用相同)。然后按照正确的参考文献编译方式(因为 \LaTeX 要重新写 \file{.bbl} 文件)重新编译即可获得正确的排序。