% !TeX root = ../CCNUthesis-doc.tex

\section{关于模版作者}

我是华中师范大学数学与统计学学院 2017 级数学与应用数学(试验)专业 1705 班夏康玮,2021 级应届毕业生。目前就读于武汉大学基础数学专业。

在我 2021 年写毕业论文的时候使用的就是邓国泰老师 CTeX 套装编写的模版,当时周围的人,包括我自己,在使用的过程中经常遇到一些奇怪的问题,比如莫名的报错,表格中插图问题等等。但是当时的我只能算个 \LaTeX{} 的半吊子,在大二的时候磕磕绊绊学了一点点 \LaTeX{},不过主要学的是数学公式那一块,其它方面也没怎么涉及,最终也没能够解决模版的问题。

但是在那个时候我就萌发了要写新模版的想法,可是当时能力不够,且重心放在论文的内容上,于是这件事就被我搁置了。

在本科毕业的那个暑假,我花了一个多月“集训” \LaTeX{},并且初步学习了 \LaTeX3。终于在 2022 年的 2 月份,基于 \cls{fduthesis},我开始开发 \cls{CCNUthesis},因为时间紧张,当时就是在 \cls{fduthesis} 的基础上“缝缝补补”,初步实现了邓国泰老师的本科旧模版的效果。

在给 2018 级使用的过程中,也不断出现新问题,新需求,但是在当时只能是临时缝补,因为要做大改动的话,一些 \LaTeX{} 接触少的用户会比较难调整。

终于在 2018 级交完终稿之后,我开始对本科模版的代码进行了几乎整个的推翻重构,并且在使用过程中出现的一些有分歧的需求控制做成了键值控制。同时也在 2022 年 6 月初基本完成了硕博模版的开发工作。

目前模版的主要维护者是我本人。但在开发过程中离不开很多人的帮助:
\begin{itemize}
  \item 感谢 \href{https://github.com/syvshc}{syvshc} 帮忙开发了西文字体的部分
  \item 感谢 \href{https://github.com/stone-zeng}{stone-zeng} 开发的 \cls{fduthesis} (模版开发初期的代码框架是基于 \cls{fduthesis})
  \item 感谢 \href{https://github.com/zepinglee}{zepinglee} 在 \cls{CCNUthesis} 开发过程中提供的帮助。
\end{itemize}

俗话说,投资效率是最好的投资。如果您觉得本模版对您有帮助,提高了您写论文的效率,不妨通过图~\ref{figure:alipay-wechat} 小额赞助我一下,这会支持我继续维护和完善模版。

\begin{figure}[htbp]
  \centering
  \includegraphics[width = \textwidth]{夏大鱼羊支付宝微信收款码.png}
  \caption{支付宝和微信收款码}
  \label{figure:alipay-wechat}
\end{figure}