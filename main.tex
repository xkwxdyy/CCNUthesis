% !TeX encoding = UTF-8
% !TeX program = xelatex
% !TeX spellcheck = en_US

%********************************************
% CCNUthesis: 华中师范大学论文模板(非官方)
% update date: 2025-02-08
% version: v1.4.5
%
% 重要提示:
%   1. 请确保使用 UTF-8 编码保存
%   2. 请使用 XeLaTeX 或 latexmk 编译
%   3. 请仔细阅读用户文档
%   4. 不需要的注释可以尽情删除
%   5. 本模板并非官方模板,请谨慎选择使用,且使用时请注意教务处、学院的要求,不同学院的细节要求可能不同
%********************************************


\documentclass[type = bachelor]{CCNUthesis}
% \documentclass[type = bachelor, blind-version=remove-partial-schoolname]{CCNUthesis}
% \documentclass[type = master]{CCNUthesis}
% \documentclass[type = master, copyright-version=new]{CCNUthesis}
% \documentclass[type = master, blind-version=blind-schoolname]{CCNUthesis}
% \documentclass[type = master,version = print-master-oneside]{CCNUthesis}
% \documentclass[type = doctor]{CCNUthesis}
% \documentclass[type = doctor,  copyright-version=new]{CCNUthesis}
% \documentclass[type = doctor, blind-version=blind-schoolname]{CCNUthesis}
% \documentclass[type = doctor, copyright-version=new, version = print-doctor]{CCNUthesis}


% type
% 学术类型
%   可选选项:bachelor|master|doctor
%     默认:bachelor

% version
% 文档版本
%   可选选项:electronic|print-master-oneside|print-master-twoside|
%           print-doctor
%     默认:electronic
%     electronic:电子版,无空白页
%     print-master-oneside:打印版,硕士,无空白页,单面打印
%     print-master-twoside:打印版,硕士,有空白页,双面打印
%     print-doctor:打印版,博士,有空白页,双面打印

% blind-version
% 盲审版本
%    可选选项:true|false|remove-partial-schoolname|remove-all-schoolname|blind-schoolname
%    默认:false
%    【本|硕|博】blind-version 或 blind-version = true:开启盲审版本,姓名、导师姓名等个人信息会去掉,去掉校名和版权声明页
%    【本|硕|博】blind-version = false 或者不填"blind-version":正常编译
%    【本】blind-version = remove-partial-schoolname:去掉个人信息,保留版权声明页,但是去掉校名和版权声明页中的“华中师范”四个字(此为邓国泰老师在旧模板中的做法)
%    【本】blind-version = remove-all-schoolname:去掉个人信息和校名,保留版权声明页,但是去掉版权声明页中出现的“华中师范大学”六个字
%    【硕|博】blind-version = blind-schoolname:去掉个人信息和校名,保留版权声明页,但是版权声明页中出现的“华中师范大学”六个字变成“XXXXXX”

% copyright-version
% 【硕|博】版权声明页版本,由于 2023 年后增加了新的版权页,故增加此键值让用户选择是否更新为最新的版本
%    可选选项:old|new
%    默认:old
%    old:旧版版权声明页,对应 copyright 目录下的 Originality_Copyright_master_doctor_old.pdf
%    new:新版版权声明页,对应 copyright 目录下的 Originality_Copyright_master_doctor_new.pdf


% 加载用户的个人信息和论文相关参数设置的配置文件
% !TeX root = ./main.tex


\ccnusetup{
  % 个人信息
  %   注意:\ccnusetup 中不能出现空行
  info = {
    % cover-type = word,
    cover-type = math,
      % 封面的类型
      %   适用学位类型:【本|硕|博】
      %   可选选项:word|math
      %     word:教务处的 word 封面格式
      %     math:数学与统计学学院的封面格式
      %     默认:math
      %
    title-line-type = constant,
      % 封面下划线的类型
      %   适用学位类型:【本|硕|博】
      %   可选选项:variable|constant
      %     variable: 只有在文字的地方才有下划线
      %     constant: 下划线是恒定长度
      %     默认:constant
      %
    title = {
      华中师范大学学位论文 \LaTeX{} 模板
    },
      % 中文-标题
      %   适用学位类型:【本|硕|博】
      %   会自动换行,如果换行点不满意,可以用 \\ 手动换行
      %   能概括整个学位论文的中心内容,简明、扼要
      %   论文题目一般不超过25个字,必要时可加副标题(在题目下一行以“——”打头)
      %
    title* = {
      CCNU thesis \LaTeX{} template 
    }, 
      % 英文-标题
      %   适用学位类型:【本|硕|博】
      %
    author = {你的姓名},
      % 中文-作者姓名
      %   适用学位类型:【本|硕|博】
      %
    author* = {Xing Ming},
      % 拼音-作者姓名
      %   适用学位类型:【硕|博】
      %
    supervisor = {教师姓名 \quad 职称},
      % 中文-指导老师姓名+职称
      %   适用学位类型:【本|硕|博】
      %   职称:讲师|副教授|教授|副研究员|研究员
      %
    supervisor*-name = {Jiao shi},
      % 拼音-指导老师姓名
      %   适用学位类型:【硕|博】
      %
    supervisor*-academic-title = {Professor},
      % 英文-指导老师职称
      %   适用学位类型:【硕|博】
      %
    level = {2018级},
      % 年级
      %   适用学位类型:【本】
      %
    student-id = {学号},
      % 学号
      %   适用学位类型:【本】
      %
    department = {数学与统计学学院},
      % 中文-学院名称
      %   适用学位类型:【本|硕|博】
      %
    department* = {School of Mathematics and Statistics},
      % 英文-学院名称
      %   适用学位类型:【硕|博】
      %
    major = {应用统计},
      % 中文-专业
      %   适用学位类型:【本|硕|博】
      %   选项参考:
      %     本:数学与应用数学(试验)|数学与应用数学(师范)|统计学
      %     硕:应用统计
      %   【硕博】「学术型学位学科专业」填写《授予博士、硕士学位和培养研究生的学科、专业目录》中的二级学科或我校自主设置专业;「专业型学位学科专业」填写「专业学位领域」,无领域学科不填
      %
    major* = {Mathematics},
      % 英文-专业
      %   适用学位类型:【硕|博】
      %   选项参考:
      %     硕:Mathematics|Applied Statistics
      %
    research-area = {教育大数据},
      % 中文-研究方向
      %   适用学位类型:【博】
      %   选项参考:
      %     硕:教育大数据
      %
    research-area* = {Education big data},
      % 英文-研究方向
      %   适用学位类型:【硕|博】
      %
    degree-type = {应用统计硕士},
      % 中文-申请学位学生类别
      %   适用学位类型:【硕|博】
      %   选项参考:
      %     硕:教育硕士|应用统计硕士|全日制硕士|同等学力人员|
      %        高校教师在职攻读硕士学位人员|专业学位人员
      %     博:博士
      %
    degree-type* = {M.S.},
      % 英文-申请学位学生类别,缩写
      %   适用学位类型:【硕】
      %   选项参考:
      %     硕:M.S.
      %
    keywords = {
      关键词1,
      关键词2,
      关键词3
    },
      % 中文-论文关键词
      %   适用学位类型:【本|硕|博】
      %   请注意,关键词之间用西文逗号 “,” 隔开,模板自动处理并在关键词之间加中文分号
      %
    keywords* = {
      keyword1,
      keyword2,
      keyword3
    },
      % 英文-论文关键词
      %   适用学位类型:【本|硕|博】
      %   请注意,关键词之间用西文逗号 “,” 隔开,模板自动处理并在关键词之间加西文分号
      %
    % year  = {2022},
      % 年份
      %   适用学位类型:【本|硕|博】
      %   如果不手动调整,则默认是「编译时」的年份
      %
    % month = {5},
      % 月份
      %   适用学位类型:【本|硕|博】
      %   如果不手动调整,则默认是「编译时」的月份
  },
  style = {
    cover_ii_only_title_content = false,
      % 第二个封面的标题是否去掉“论文标题:”并且将标题居中
      %   适用学位类型:【硕|博】
      %   可选选项:true|false
      %     默认:false
      %
    font = times,
      % 西文字体
      %   适用学位类型:【本|硕|博】
      %   可选选项:newtx|times|stixtwo|xits|tg|none
      %     目前的字体配置为: (TG = TeX Gyre, 默认值为 times)
      %     *******************************************************************
      %     选项名   : serif,            sans,     mono,         math
      %     *******************************************************************
      %     stixtwo : STIX Two Text,    TG Heros, TG Cursor,    STIX Two Math
      %     xits    : XITS,             TG Heros, TG Cursor,    XITS Math
      %     times   : Times New Roman,  Arial,    Courier New,  newtxmath
      %     newtx   : TG Termes,        TG Heros, TG Cursor,    newtxmath
      %     tg      : TG Termes,        TG Heros, TG Cursor,    TG Termes Math
      %     *******************************************************************
    footnote-style = xits,
      % 脚注编号样式
      %   可选选项:plain|libertinus|libertinus*|libertinus-sans|
      %           pifont|pifont*|pifont-sans|pifont-sans*|
      %           xits|xits-sans|xits-sans*
      %   默认与西文字体保持一致
      %   注意:对于本科生,《附件4:关于毕业论文注释与参考文献著录格式修订的通知》中指出“文科术科的论文注释「使用」脚注”及“理工科的论文注释「不使用」脚注”
    cjk-font = fandol,
      % 中文字体
      %   适用学位类型:【本|硕|博】
      %   可选选项:adobe|fandol|founder|mac|sinotype|sourcehan|windows|none
      %   默认:fandol
      %   注意:
      %     1. 中文字体设置高度依赖于系统。各系统建议方案:
      %          windows:cjk-font = windows
      %          mac:    cjk-font = mac
      %          linux:  cjk-font = fandol(默认值)
      %     2. 除 fandol 和 sourcehan 外,其余字体均为商用字体,请注意版权问题
      %     3. 但 fandol 字体缺字比较严重,而 sourcehan 没有配备楷体和仿宋体
      %
    % chapter-breakstyle = continuous,
    % chapter-breakstyle = newpage,
      % chapter 是否要新起一页开始
      %   适用学位类型:【本】
      %   可选选项:continuous|newpage
      %   默认:continuous
      %     continuous:chapter 不新起一页开始
      %     newpage:chapter 新起一页开始
      %
    caption-labelstyle = dot,
      % 图表标题 label 计数样式
      %   适用学位类型:【本|硕|博】
      %     【硕|博】应选择 dot (根据研究生学位论文规范)
      %   可选选项:arabic|hyphen|dot
      %     arabic:样式为图1,图2,表1,表2...,并且跨 chapter 连续编号,即 上一个 chapter 的图编号若为4,下一个 chapter 的第一个图编号为 5
      %     hyphen:样式为图1-1,图2-1,表1-1...
      %       x-y 的 x 为 chapter 值,y 为图表的计数器值,新的 chapter 中 y 会清零重新计数
      %     dot:样式为图1.1,图2.1,表1.1...
      %       x.y 的 x 为 chapter 值,y 为图表的计数器值,新的 chapter 中 y 会清零重新计数
      %     默认:dot
      %
    caption-labelseperator = space,
      % 图表标题 label 和 标题内容 之内的分隔符
      %   适用学位类型:【本|硕|博】
      %   可选选项:colon|space
      %     colon 表示 「:␣」,即一个西文冒号加一个空格
      %     【硕|博】space 表示 「␣␣」,即两个空格
      %     【本】space 表示 1em 长的空格
      %     默认:space
      %
    show-head = true,
    % show-head = false,
      % 是否显示页眉
      %   适用学位类型:【硕|博】
      %
    show-headlogo = true,
      % 是否显示页眉 logo
      %   适用学位类型:【硕|博】
      %   可选选项:true|false
      %     默认:false
      %   logo 内容:华中师范大学校徽 + 硕士/博士学位论文中英文字样
      %
    headline = thick-thin,
      % 页眉的线的类型
      %   适用学位类型:【硕|博】
      %   可选选项:single|double|thin-thick|thick-thin|none
      %     single:一条线
      %     double:两条线,一样粗细
      %     thin-thick:两条线,上细下粗
      %     thick-thin:两条线,上粗下细
      %     none:没有页眉的线
      %     默认:none
      %
    % head-scope = all,
    head-scope = main,
      % 页眉的作用范围
      %   适用学位类型:【硕|博】
      %   可选选项:all|main
      %     all:除了封面外所有
      %     main:在正文开始才有页眉
      %   默认:main
    % keywords-newline = true,
      % 摘要和关键词之间是否空一行
      %   适用学位类型:【本|硕|博】
      %   可选选项:true|false
      %   默认:本:false,硕博:true
      %
    % listoffigures-show = true,
      % 是否显示图目录
      %   适用学位类型:【本|硕|博】
      %   可选选项:true|false
      %   默认:本:false,硕博:true
      %
    % listoftables-show = true,
      % 是否显示表目录
      %   适用学位类型:【本|硕|博】
      %   可选选项:true|false
      %   默认:本:false,硕博:true
      %
    listoffigures-name = {插 \quad 图},
    listoftables-name  = {表 \quad 格},
      % 图表目录的节标题
      %   适用学位类型:【本|硕|博】
      %
    bib-style = ccnu-bachelor-author-year,
      % bib-style 表示参考文献的格式
      %   适用学位类型:【本|硕|博】
      %   可选选项:
      %     ccnu-bachelor-numerical|ccnu-bachelor-author-year
      %     ccnu-master|ccnu-doctor|gb7714-2015|gb7714-2015ay
      %       ccnu-bachelor-numerical:【本】学校标准,顺序编码制
      %       ccnu-bachelor-author-year:【本】学校标准,作者-年制
      %       ccnu-master:【硕】国标,顺序编码制
      %       ccnu-doctor:【博】国标,顺序编码制
      %       gb7714-2015:国标,顺序编码制
      %       gb7714-2015ay:国标,作者-年制
      %     默认:ccnu-bachelor-author-year
      %
    bib-resource = {CCNUthesis-main.bib},
      % 参考文献数据源
      %   适用学位类型:【本|硕|博】
      %   注意:需要加 bib 后缀
      %   默认:CCNUthesis-main.bib
      %
    % bib-keyval = {doi=false},
      % biblatex 宏包的键值接口
      %   适用学位类型:【本|硕|博】
      %   具体可输入选项请参考 biblatex-gb7714-2015 宏包文档(命令行输入 `texdoc biblatex-gb7714-2015`)
      %   比如 doi=false 表示不显示 doi
      %
    hyperlink = none,
      % 超链接样式
      %   可选选项:color|none
      %   默认:none
      %
    hyperlink-color = default,
      % 超链接颜色
      %   可选选项:default|classic|material|graylevel|prl
      %   默认:default
      %
    theorem-bodyfont = normal,
      % 定理环境的主体字体样式
      %   可选选项:normal|kaishu
      %   默认:normal
      %   normal:正常字体,和正文相同
      %   kaishu:中文变成楷书,英文不变
  }
}


\ccnusetup{
  style / fullwidth-stop = catcode,
  % 标点的自动替换
  %   适用学位类型:【本|硕|博】
  %   可选选项:catcode|mapping|false
  %     catcode:显式的 “。” 会被替换为 “.”(e.g. 不包括用宏定义保存的 “。”)
  %     mapping:所有的 “。” 会被替换为 “.”(使用 LuaLaTeX 编译则无效)
  %     false:不进行替换
  %     默认:catcode
  %   作用:是否把全角实心句点 “.” 作为默认的句号形状,即正文中输入“。” 最终编译效果为“. ”
  %   说明:一般科技类文章需要替换,防止“. ”与“。”混淆
  %
}


% 需要的额外宏包可以在此处自行调用
%   关于模板已经载入的宏包请参看手册「宏包依赖情况」



% 需要的命令环境可以自行定义
\newcommand{\upe}{\mathrm{e}}   % 直立的e,用于表示常量,如自然常数      
\newcommand{\upi}{\mathrm{i}}   % 直立的i,用于表示常量,如虚数单位


\begin{document}


% \frontmatter 开启论文前置部分
% 前置部分包含目录、中英文摘要以及符号表等
\frontmatter


% 摘要
%   适用学位类型:【本|硕|博】
\input{./front/abstract.tex}


% 符号表
%   适用学位类型:【本|硕|博】
%   不需要的话将 \input{./front/notation.tex} 注释掉或删除
\input{./front/notation.tex}



% \mainmatter 进入论文主体部分
\mainmatter



% 主体采用多文件编译的方式
% 即把每一章放进一个单独的 tex 文件里,并在这里用 \input 导入
% 例如 \input{./body/chapter1.tex}
% 表示插入 main.tex 所在目录中的 body 目录下的 chapter1.tex 文件

% 正文
% 适用学位类型:【本|硕|博】
\input{./body/chapter0.tex}  % 代码示例文件,不需要的话注释掉或者删掉即可

\input{./body/chapter1.tex}
% !TeX root = ../main.tex

\chapter{引用与链接}

\section{脚注}

华中师范大学《附件4:关于修订毕业论文注释与参考文献著录格式的通知》提到

\begin{itemize}
  \item 文科术科的论文注释使用脚注
  \item 理工科的论文注释不使用脚注
\end{itemize}

\subsection{测试}
\subsection{测试}
\subsection{测试}
\subsection{测试}



\section{引用文中小节}\label{sec:ref}

如引用小节~\ref{sec:ref}



\section{引用参考文献}

这是一个参考文献引用的范例:“\parencite{邱泽奇建构与分化}提出……”。还可以引用多个文献:“\parencite{丁文祥rawtype,李晓东rawtype}提出……”。

英文文献 \parencite{feynman2011}

英文文献 \parencite[12]{feynman2011}

英文文献 \parencite[Thm1]{feynman2011}

英文文献 \parencite[12][Thm1]{feynman2011}

英文文献 \cite{feynman2011}

英文文献 \cite[12]{feynman2011}

英文文献 \cite[Thm1]{feynman2011}

英文文献 \cite[12][Thm1]{feynman2011}

\section{链接相关}


模板使用了 hyperref 包处理相关链接,使用 \verb|\href| 可以生成超链接,默认不显示链接颜色。如果需要输出网址,可以使用 \verb|\url| 命令,示例:\url{https://github.com}。
% !TeX root = ../main.tex

\chapter{图表}

推荐图片都放在 figures 目录中,下面的图片可以不写后缀名,也不需要写上级目录
\begin{figure}[htbp]
  \centering
  % 推荐图片都放在 figures 目录中,下面的图片可以不写后缀名,也不需要写上级目录
  % \includegraphics[width = 5cm]{ccnulogo}
  \caption{插入图片}
  \label{figure:test1}
\end{figure}
推荐图片都放在 figures 目录中,下面的图片可以不写后缀名,也不需要写上级目录推荐图片都放在 figures 目录中,下面的图片可以不写后缀名,也不需要写上级目录

推荐图片都放在 figures 目录中,下面的图片可以不写后缀名,也不需要写上级目录
\begin{table}[htbp]
  \centering
  \caption{测试}
  \label{table:test1}
  \begin{tabular}{|c|c|}
    11 & 22 \\
    33 & 44 
  \end{tabular}
\end{table}
推荐图片都放在 figures 目录中,下面的图片可以不写后缀名,也不需要写上级目录


图~\ref{figure:test1} 和表~\ref{table:test1} 的引用
% !TeX root = ../main.tex

\chapter{公式}


\section{公式引用}

\begin{equation}\label{equation:test1}
  K \leq 
  \begin{dcases}
    2 \lfloor \frac{d}{2d - n} \rfloor    , & K \text{为偶数} ; \\
    2 \lfloor \frac{d}{2d - n} \rfloor - 1, & K \text{为奇数}. \\
  \end{dcases}
\end{equation}

公式~\eqref{equation:test1} 的引用
\[2 \lfloor \frac{d}{2d - n} \rfloor \]
公式~\eqref{equation:test1} 的引用
\[2 \lfloor \frac{d}{2d - n} \rfloor \]
\[2 \lfloor \frac{d}{2d - n} \rfloor \]
\[2 \lfloor \frac{d}{2d - n} \rfloor \]
\begin{proof}
  \[
    x^2
  \]
\end{proof}
\begin{proof}
  \[
    x^2  \qedhere
  \]
\end{proof}



% 论文后文部分,参考文献、致谢、附录等
\backmatter


% 参考文献
%********************************************
% 行内引用:\parencite{} or \parencite[]{},下面两个情况要用行内引用
%    - 去掉这个引用句子结构不完整,比如“定理证明可参看[1]”
%    - 英文文献的引用
% 上标引用:\cite{} or \cite[]{},下面情况要用上标引用
%    - 去掉这个引用,句子结构完整,比如“作者提到,‘CCNUthesis 真是一个好模板。’^[1]”
%      其中“^[1]” 表示上标引用
%********************************************

% 输出参考文献
%   适用学位类型:【本|硕|博】
%   无须用户进行任何操作
%   用户要想正确输出参考文献:
%     1. 在 bib 文件中输入正确的参考文献条目信息
%     2. 在正文中正确地使用 \parencite 或 \cite 引用
%     3. 使用 xelatex-biber-xelatex*2 编译链或 latexmk 方式编译 main.tex 文件
\printbibliography


% 附录
% 【硕|博】附录在致谢前
% !TeX root = ../main.tex

\appendix


\chapter{调查问卷}

\begin{enumerate}
  \item 我是一个问题题干
    \begin{choices}
      \item 选项1
      \item 选项2
      \item 选项3
      \item 选项4
      \item 选项1
      \item 选项2
      \item 选项3
      \item 选项4
    \end{choices}
  \item 我是一个问题题干
    \begin{choices}
      \item 选项1
      \item 选项2
      \item 选项3
      \item 选项4
    \end{choices}
\end{enumerate}

\begin{enumerate}[(a)]
  \item 我是一个问题题干
    \begin{choices}
      \item 选项1
      \item 选项2
      \item 选项3
      \item 选项4
    \end{choices}
  \item 我是一个问题题干
    \begin{choices}
      \item 选项1
      \item 选项2
      \item 选项3
      \item 选项4
    \end{choices}
\end{enumerate}

用 \verb|choices| 环境可以排版 \emph{任意个} 选项,只需要像罗列环境 \verb|enumerate| 环境等一样用 \verb|\item| 分隔即可。

\verb|choices| 环境的 label 可以方便地进行调整
\begin{itemize}
  \item arabic(阿拉伯数字)
  \item alph(小写英文)
  \item Alph(大写英文)
  \item roman(小写罗马数字)
  \item Roman(大写罗马数字)
  \item circlednumber(带圈数字)
\end{itemize}

更多关于 \verb|choices| 环境的精细调整可以查看 \url{https://gitee.com/xkwxdyy/exam-zh}。

\begin{choices}[label = \arabic*)]
  \item 选项1
  \item 选项2
  \item 选项3
  \item 选项4
\end{choices}

\begin{choices}[label = (\alph*]
  \item 选项1
  \item 选项2
  \item 选项3
  \item 选项4
\end{choices}

\begin{choices}[label = \Alph*.]
  \item 选项1
  \item 选项2
  \item 选项3
  \item 选项4
\end{choices}

\begin{choices}[label = \roman*:]
  \item 选项1
  \item 选项2
  \item 选项3
  \item 选项4
\end{choices}

\begin{choices}[label = \Roman*-]
  \item 选项1
  \item 选项2
  \item 选项3
  \item 选项4
\end{choices}

\begin{choices}[label = \circlednumber*]
  \item 选项1
  \item 选项2
  \item 选项3
  \item 选项4
  \item 选项5
  \item 选项6
  \item 选项7
  \item 选项8
\end{choices}


还可以修改 \verb|columns| 键值来决定每行排多少个
\begin{choices}[
  columns = 3,            % 手动控制每行多少个选项,否则自己根据宽度自动排版
  label = (\arabic*)      % label 的样式,支持 arabic, alph, Alph, roman, Roman, circlednumber
]
  \item 选项1
  \item 选项2
  \item 选项3
  \item 选项4
  \item 选项5
  \item 选项6
\end{choices}



\chapter{访谈记录}


\section{与 A 的访谈记录}

\begin{figure}[htbp]
  \centering
  \includegraphics[width = 5cm]{example-image-a}
  \caption{测试}
  \label{figure:test2}
\end{figure}

\begin{table}[htbp]
  \centering
  \caption{测试}
  \label{table:test2}
  \begin{tabular}{|c|c|}
    11 & 22 \\
    33 & 44 
  \end{tabular}
\end{table}

\begin{figure}[htbp]
  \centering
  \includegraphics[width = 5cm]{example-image-a}
  \caption{测试}
  \label{figure:test3}
\end{figure}

\begin{table}[htbp]
  \centering
  \caption{测试}
  \label{table:test3}
  \begin{tabular}{|c|c|}
    11 & 22 \\
    33 & 44 
  \end{tabular}
\end{table}

\section{与 B 的访谈记录}

% 攻读学位期间取得的研究成果(博)
% \include{./back/publications.tex}


% 致谢
\include{./back/acknowledgements.tex}


% 附录
% 【本】附录在致谢后
% % !TeX root = ../main.tex

\appendix


\chapter{调查问卷}

\begin{enumerate}
  \item 我是一个问题题干
    \begin{choices}
      \item 选项1
      \item 选项2
      \item 选项3
      \item 选项4
      \item 选项1
      \item 选项2
      \item 选项3
      \item 选项4
    \end{choices}
  \item 我是一个问题题干
    \begin{choices}
      \item 选项1
      \item 选项2
      \item 选项3
      \item 选项4
    \end{choices}
\end{enumerate}

\begin{enumerate}[(a)]
  \item 我是一个问题题干
    \begin{choices}
      \item 选项1
      \item 选项2
      \item 选项3
      \item 选项4
    \end{choices}
  \item 我是一个问题题干
    \begin{choices}
      \item 选项1
      \item 选项2
      \item 选项3
      \item 选项4
    \end{choices}
\end{enumerate}

用 \verb|choices| 环境可以排版 \emph{任意个} 选项,只需要像罗列环境 \verb|enumerate| 环境等一样用 \verb|\item| 分隔即可。

\verb|choices| 环境的 label 可以方便地进行调整
\begin{itemize}
  \item arabic(阿拉伯数字)
  \item alph(小写英文)
  \item Alph(大写英文)
  \item roman(小写罗马数字)
  \item Roman(大写罗马数字)
  \item circlednumber(带圈数字)
\end{itemize}

更多关于 \verb|choices| 环境的精细调整可以查看 \url{https://gitee.com/xkwxdyy/exam-zh}。

\begin{choices}[label = \arabic*)]
  \item 选项1
  \item 选项2
  \item 选项3
  \item 选项4
\end{choices}

\begin{choices}[label = (\alph*]
  \item 选项1
  \item 选项2
  \item 选项3
  \item 选项4
\end{choices}

\begin{choices}[label = \Alph*.]
  \item 选项1
  \item 选项2
  \item 选项3
  \item 选项4
\end{choices}

\begin{choices}[label = \roman*:]
  \item 选项1
  \item 选项2
  \item 选项3
  \item 选项4
\end{choices}

\begin{choices}[label = \Roman*-]
  \item 选项1
  \item 选项2
  \item 选项3
  \item 选项4
\end{choices}

\begin{choices}[label = \circlednumber*]
  \item 选项1
  \item 选项2
  \item 选项3
  \item 选项4
  \item 选项5
  \item 选项6
  \item 选项7
  \item 选项8
\end{choices}


还可以修改 \verb|columns| 键值来决定每行排多少个
\begin{choices}[
  columns = 3,            % 手动控制每行多少个选项,否则自己根据宽度自动排版
  label = (\arabic*)      % label 的样式,支持 arabic, alph, Alph, roman, Roman, circlednumber
]
  \item 选项1
  \item 选项2
  \item 选项3
  \item 选项4
  \item 选项5
  \item 选项6
\end{choices}



\chapter{访谈记录}


\section{与 A 的访谈记录}

\begin{figure}[htbp]
  \centering
  \includegraphics[width = 5cm]{example-image-a}
  \caption{测试}
  \label{figure:test2}
\end{figure}

\begin{table}[htbp]
  \centering
  \caption{测试}
  \label{table:test2}
  \begin{tabular}{|c|c|}
    11 & 22 \\
    33 & 44 
  \end{tabular}
\end{table}

\begin{figure}[htbp]
  \centering
  \includegraphics[width = 5cm]{example-image-a}
  \caption{测试}
  \label{figure:test3}
\end{figure}

\begin{table}[htbp]
  \centering
  \caption{测试}
  \label{table:test3}
  \begin{tabular}{|c|c|}
    11 & 22 \\
    33 & 44 
  \end{tabular}
\end{table}

\section{与 B 的访谈记录}


\end{document}