% !TeX encoding = UTF-8
% !TeX program = xelatex
% !TeX spellcheck = en_US

%********************************************
% CCNUthesis: 华中师范大学论文模板(非官方)
% update date: 2024-03-27
% version: v1.3.1
%
% 重要提示:
%   1. 请确保使用 UTF-8 编码保存
%   2. 请使用 XeLaTeX 或 latexmk 编译
%   3. 请仔细阅读用户文档
%   4. 不需要的注释可以尽情删除
%   5. 本模板并非官方模板,请谨慎选择使用,且使用时请注意教务处、学院的要求,不同学院的细节要求可能不同
%********************************************


\documentclass[type = bachelor]{CCNUthesis}
% \documentclass[type = bachelor, blind-version=remove-partial-schoolname]{CCNUthesis}
% \documentclass[type = master]{CCNUthesis}
% \documentclass[type = master, blind-version]{CCNUthesis}
% \documentclass[type = master, version = print-master-oneside]{CCNUthesis}
% \documentclass[type = master, version = print-master-twoside]{CCNUthesis}
% \documentclass[type = doctor]{CCNUthesis}
% \documentclass[type = doctor, blind-version]{CCNUthesis}
% \documentclass[type = doctor, version = print-doctor]{CCNUthesis}


% type
% 学术类型
%   可选选项:bachelor|master|doctor
%     默认:bachelor

% version
% 文档版本
%   可选选项:electronic|print-master-oneside|print-master-twoside|
%           print-doctor
%     默认:electronic
%     electronic:电子版,无空白页
%     print-master-oneside:打印版,硕士,无空白页,单面打印
%     print-master-twoside:打印版,硕士,有空白页,双面打印
%     print-doctor:打印版,博士,有空白页,双面打印

% blind-version
% 盲审版本
%    可选选项:true|false|remove-partial-schoolname|remove-all-schoolname|remove-all-schoolname-and-copyright
%    默认:false
%    【本|硕|博】blind-version 或 blind-version = true:开启盲审版本,姓名、导师姓名等个人信息会去掉,去掉校名和版权声明页
%    【本|硕|博】blind-version = false 或者不填"blind-version":正常编译
%    【本】blind-version = remove-partial-schoolname:去掉个人信息,保留校名和版权声明页,但是去掉校名和版权声明页中的“华中师范”四个字(此为邓国泰老师在旧模板中的做法)
%    【本】blind-version = remove-all-schoolname:去掉个人信息和校名,保留版权声明页,但是去掉版权声明页中出现的“华中师范大学”六个字


% 加载用户的个人信息和论文相关参数设置的配置文件
\input{ccnu-setup.tex}


% 需要的额外宏包可以在此处自行调用
%   关于模板已经载入的宏包请参看手册「宏包依赖情况」
\usepackage{mathtools}



% 需要的命令环境可以自行定义
\newcommand{\upe}{\mathrm{e}}   % 直立的e,用于表示常量,如自然常数      
\newcommand{\upi}{\mathrm{i}}   % 直立的i,用于表示常量,如虚数单位


\begin{document}


% \frontmatter 开启论文前置部分
% 前置部分包含目录、中英文摘要以及符号表等
\frontmatter


% 摘要
%   适用学位类型:【本|硕|博】
\newpage
\pagestyle{plain}
\setcounter{page}{1}
%����ժҪ
\zihao{-4}
\textbf{����ժҪ}:
\addcontentsline{toc}{section}{\protect����ժҪ}
  \myabstract

\textbf{�ؼ���:} \mykeywords
\addcontentsline{toc}{section}{\protect �ؼ���}
%Ӣ��ժҪ
\bigskip

\textbf{Title}: \mytitle
\addcontentsline{toc}{section}{\protect Title}
\medskip

\textbf{Abstract}:
\addcontentsline{toc}{section}{\protect Abstract}
 \myabstracten

\textbf{Keywords:} \mykeywordsen


\addcontentsline{toc}{section}{\protect Keywords}
\newpage



% 符号表
%   适用学位类型:【本|硕|博】
%   不需要的话将 % 符号表

% 语法与 LaTeX 表格一致:列用 & 区分,行用 \\ 区分
% 如需修改格式,可以使用可选参数:
%   \begin{notation}[ll]
%     $x$ & 坐标 \\
%     $p$ & 动量
%   \end{notation}
% 可选参数与 LaTeX 标准表格的列格式说明语法一致
% 这里的 “ll” 表示两列均为自动宽度,并且左对齐


\begin{notation}[ll]
  $x$                  & 坐标        \\
  $p$                  & 动量        \\
  $\psi(x)$            & 波函数      \\
\end{notation} 注释掉或删除
% 符号表

% 语法与 LaTeX 表格一致:列用 & 区分,行用 \\ 区分
% 如需修改格式,可以使用可选参数:
%   \begin{notation}[ll]
%     $x$ & 坐标 \\
%     $p$ & 动量
%   \end{notation}
% 可选参数与 LaTeX 标准表格的列格式说明语法一致
% 这里的 “ll” 表示两列均为自动宽度,并且左对齐


\begin{notation}[ll]
  $x$                  & 坐标        \\
  $p$                  & 动量        \\
  $\psi(x)$            & 波函数      \\
\end{notation}



% \mainmatter 进入论文主体部分
\mainmatter



% 主体采用多文件编译的方式
% 即把每一章放进一个单独的 tex 文件里,并在这里用 \input 导入
% 例如 \chapter{引言}

\section{研究背景}
\begin{figure}[tbh]
  \centering
  \includegraphics[width = 5cm]{example-image-a}
  \caption{测试}
\end{figure}
\begin{table}[tbh]
  \caption{测试}
  \centering
  \begin{tblr}{|c|c|}
    11 & 22 \\
    33 & 44 
  \end{tblr}
\end{table}



\subsection{前人工作}


测试 \parencite{邱泽奇建构与分化}
\section{研究背景}

% 表示插入 main.tex 所在目录中的 body 目录下的 chapter1.tex 文件

% 正文
% 适用学位类型:【本|硕|博】
% % !TeX root = ../main.tex

\chapter{常用命令环境示例}

此章用于展示一些常用的命令环境的效果,用户在具体写论文过程中可以用来参考效果,如果不需要此章出现在正文中,只需要将主文件 \verb|main.tex| 文件中的 \verb|% !TeX root = ../main.tex

\chapter{常用命令环境示例}

此章用于展示一些常用的命令环境的效果,用户在具体写论文过程中可以用来参考效果,如果不需要此章出现在正文中,只需要将主文件 \verb|main.tex| 文件中的 \verb|% !TeX root = ../main.tex

\chapter{常用命令环境示例}

此章用于展示一些常用的命令环境的效果,用户在具体写论文过程中可以用来参考效果,如果不需要此章出现在正文中,只需要将主文件 \verb|main.tex| 文件中的 \verb|\input{./body/chapter0.tex}| 代码注释掉即可(不建议删除,因为随时可以取消注释查看效果)。


\section{列表环境}


如果要修改 \verb|enumerate| 环境的 label 样式的话:

\begin{enumerate}
  \item 第一项
  \item 第二项
  \item 第三项
  \item 第四项
\end{enumerate}

\begin{enumerate}[1)]
  \item 第一项
  \item 第二项
  \item 第三项
  \item 第四项
\end{enumerate}

\begin{enumerate}[a.]
  \item 第一项
  \item 第二项
  \item 第三项
  \item 第四项
\end{enumerate}

\begin{enumerate}[(A)]
  \item 第一项
  \item 第二项
  \item 第三项
  \item 第四项
\end{enumerate}

test
\begin{enumerate}[(i)]
  \item 第一项
  \item 第二项
  \item 第三项
  \item 第四项
\end{enumerate}

\begin{enumerate}[I]
  \item 第一项
  \item 第二项
  \item 第三项
  \item 第四项
\end{enumerate}

\begin{enumerate}[label = \textbf{断言} \Alph*]
  \item 一般来说使用断言,推荐使用 claim 环境
  \item 但是如果真要有一些断言的层级分化的话
  \item 可以考虑用 enumerate 环境的 label 选项
\end{enumerate}

\begin{enumerate}[\textbf{断言} A]
  \item 1
  \item 2
\end{enumerate}


\section{已定义好的一些数学定理环境}

定理环境内的括号,不管是中文还是西文括号,都不会出现倾斜,不需要像旧模板一样需要用手动用 \verb|\textit| 调整
\begin{definition}[测度]
  (参见文献xxx) 这是一段文字 $E = m c^2$  (中文括号)和 (西文括号)
\end{definition}

\begin{theorem}
  这是一段文字 $E = m c^2$
\end{theorem}


\begin{proof}
  这是一段文字 $E = m c^2$
\end{proof}

\begin{proof}[定理xx的证明]
  这是一段文字 $E = m c^2$
\end{proof}

\begin{example}
  这是一段文字 $E = m c^2$
\end{example}

\begin{property}
  这是一段文字 $E = m c^2$
\end{property}

\begin{proposition}
  这是一段文字 $E = m c^2$
\end{proposition}

\begin{corollary}
  这是一段文字 $E = m c^2$
\end{corollary}

\begin{lemma}
  这是一段文字 $E = m c^2$
\end{lemma}

\begin{axiom}
  这是一段文字 $E = m c^2$
\end{axiom}

\begin{counterexample}
  这是一段文字 $E = m c^2$
\end{counterexample}

\begin{conjecture}
  这是一段文字 $E = m c^2$
\end{conjecture}

\begin{question}
  这是一段文字 $E = m c^2$
\end{question}

\begin{claim}
  这是一段文字 $E = m c^2$
\end{claim}

\begin{remark}
  这是一段文字 $E = m c^2$
\end{remark}

\begin{theorem}[Cauchy]\label{thm:test}
  这是一个定理
  \begin{equation}\label{eq:test1}
    a^2 + b^2 = c^2 \geq 0
  \end{equation}

  \begin{equation}\label{eq:test2}
    a^2 + b^2 = c^2 \geq 0
  \end{equation}
\end{theorem}

我想引用定理~\ref{thm:test} 和公式~\ref{eq:test2}


定理括号测试:

\begin{theorem}
  测试
  \begin{enumerate}
    \item 中文(括号)没输入空格的效果
    \item 中文 (括号) 输入空格的效果
    \item 西文(括号)没输入空格的效果
    \item 西文 (括号) 输入空格的效果
  \end{enumerate}
\end{theorem} 


\begin{proof}
  test
  \[
    a^2 + b^2 = c^2
  \]
\end{proof}

\begin{proof}
  test
  \[
    a^2 + b^2 = c^2  \qedhere
  \]
\end{proof}

\section{浮动体使用}

用 \verb|\label| 引用时,只需要将其放在 \verb|\caption| 的下一行即可。

和定理类环境的引用相同,建议 label 的名称格式为 \verb|figure:xxx| 或 \verb|table:xxx| 其中 \verb|xxx| 可以写中文,尽可能言简意赅地写这个图或表的内容描述,也尽可能写出图表的“独一无二性”,方便自己记忆,也防止在图表一多的时候不知道引用哪一个。

\verb|\figure| 的 \verb|\caption| 是放在 \verb|\includegraphics| 的下方,而 \verb|\table| 的 \verb|\caption| 是放在 \verb|tabular| 或 \verb|tblr| 环境的上方。

\begin{figure}[htbp]
  \centering
  \includegraphics[width = 5cm]{example-image-a}
  \caption{测试}
  \label{figure:test}
\end{figure}

\begin{table}[htbp]
  \centering
  \caption{测试}
  \label{table:test}
  \begin{tabular}{|c|c|}
    11 & 22 \\
    33 & 44 
  \end{tabular}
\end{table}

图 \ref{figure:test} 和表 \ref{table:test} 用来测试两个浮动体和交叉引用



\section{部分数学符号的输入}

本节主要是一些数学符号的输入介绍


\subsection{直体符号}

科技类论文中,建议一些数学符号使用直体(“up”前缀表示直体)
  \begin{itemize}
    \item 直立的 pi :\verb|\uppi| $\to \uppi$
    \item 直立的 e :\verb|\upe| $\to \upe$
    \item 直立的 i :\verb|\upi| $\to \upi$
  \end{itemize}


\section{参考文献引用}

\subsection{数学类}

行间\parencite[thm 3.1]{zurek2014quantum}

行间\parencite{zurek2014quantum}



\subsection{文科类}

上标\cite[test]{zurek2014quantum}

上标\cite{zurek2014quantum}



\section{《附件4:关于修订毕业论文注释与参考文献著录格式的通知》中的参考文献效果}

  text\parencite{李晓东rawtype}

  text\parencite{Ahnrawtype}

  text\parencite{Ahnrawtype}

  text\parencite{丁文祥rawtype}

  text\parencite{邱泽奇会议论文集rawtype}

  text\parencite{雷光春rawtype}

  text\parencite{zhangrawtype}

  text\parencite{邱泽奇会议论文rawtype}

  text\parencite{马克思rawtype}

  text\parencite{昂温rawtype}

  text\parencite{Fothrawtype}

  text\parencite{杨国枢rawtype}

  text\parencite{Morisonrawtype}

  text\parencite{张志祥rawtype}

  text\parencite{徐秀英rawtype}

  text\parencite{Aldemitarawtype}

  text\parencite{张凯军rawtype}

  text\parencite{Kosekrawtype}

  text\parencite{文献编写rawtype}

  text\parencite{国防白皮rawtype}

  text\parencite{federalrawtype}

  text\parencite{healthrawtype}

  text\parencite{江向东rawtype}

  text\parencite{萧钮rawtype}

  text\parencite{Dublinrawtype}

文字测试| 代码注释掉即可(不建议删除,因为随时可以取消注释查看效果)。


\section{列表环境}


如果要修改 \verb|enumerate| 环境的 label 样式的话:

\begin{enumerate}
  \item 第一项
  \item 第二项
  \item 第三项
  \item 第四项
\end{enumerate}

\begin{enumerate}[1)]
  \item 第一项
  \item 第二项
  \item 第三项
  \item 第四项
\end{enumerate}

\begin{enumerate}[a.]
  \item 第一项
  \item 第二项
  \item 第三项
  \item 第四项
\end{enumerate}

\begin{enumerate}[(A)]
  \item 第一项
  \item 第二项
  \item 第三项
  \item 第四项
\end{enumerate}

test
\begin{enumerate}[(i)]
  \item 第一项
  \item 第二项
  \item 第三项
  \item 第四项
\end{enumerate}

\begin{enumerate}[I]
  \item 第一项
  \item 第二项
  \item 第三项
  \item 第四项
\end{enumerate}

\begin{enumerate}[label = \textbf{断言} \Alph*]
  \item 一般来说使用断言,推荐使用 claim 环境
  \item 但是如果真要有一些断言的层级分化的话
  \item 可以考虑用 enumerate 环境的 label 选项
\end{enumerate}

\begin{enumerate}[\textbf{断言} A]
  \item 1
  \item 2
\end{enumerate}


\section{已定义好的一些数学定理环境}

定理环境内的括号,不管是中文还是西文括号,都不会出现倾斜,不需要像旧模板一样需要用手动用 \verb|\textit| 调整
\begin{definition}[测度]
  (参见文献xxx) 这是一段文字 $E = m c^2$  (中文括号)和 (西文括号)
\end{definition}

\begin{theorem}
  这是一段文字 $E = m c^2$
\end{theorem}


\begin{proof}
  这是一段文字 $E = m c^2$
\end{proof}

\begin{proof}[定理xx的证明]
  这是一段文字 $E = m c^2$
\end{proof}

\begin{example}
  这是一段文字 $E = m c^2$
\end{example}

\begin{property}
  这是一段文字 $E = m c^2$
\end{property}

\begin{proposition}
  这是一段文字 $E = m c^2$
\end{proposition}

\begin{corollary}
  这是一段文字 $E = m c^2$
\end{corollary}

\begin{lemma}
  这是一段文字 $E = m c^2$
\end{lemma}

\begin{axiom}
  这是一段文字 $E = m c^2$
\end{axiom}

\begin{counterexample}
  这是一段文字 $E = m c^2$
\end{counterexample}

\begin{conjecture}
  这是一段文字 $E = m c^2$
\end{conjecture}

\begin{question}
  这是一段文字 $E = m c^2$
\end{question}

\begin{claim}
  这是一段文字 $E = m c^2$
\end{claim}

\begin{remark}
  这是一段文字 $E = m c^2$
\end{remark}

\begin{theorem}[Cauchy]\label{thm:test}
  这是一个定理
  \begin{equation}\label{eq:test1}
    a^2 + b^2 = c^2 \geq 0
  \end{equation}

  \begin{equation}\label{eq:test2}
    a^2 + b^2 = c^2 \geq 0
  \end{equation}
\end{theorem}

我想引用定理~\ref{thm:test} 和公式~\ref{eq:test2}


定理括号测试:

\begin{theorem}
  测试
  \begin{enumerate}
    \item 中文(括号)没输入空格的效果
    \item 中文 (括号) 输入空格的效果
    \item 西文(括号)没输入空格的效果
    \item 西文 (括号) 输入空格的效果
  \end{enumerate}
\end{theorem} 


\begin{proof}
  test
  \[
    a^2 + b^2 = c^2
  \]
\end{proof}

\begin{proof}
  test
  \[
    a^2 + b^2 = c^2  \qedhere
  \]
\end{proof}

\section{浮动体使用}

用 \verb|\label| 引用时,只需要将其放在 \verb|\caption| 的下一行即可。

和定理类环境的引用相同,建议 label 的名称格式为 \verb|figure:xxx| 或 \verb|table:xxx| 其中 \verb|xxx| 可以写中文,尽可能言简意赅地写这个图或表的内容描述,也尽可能写出图表的“独一无二性”,方便自己记忆,也防止在图表一多的时候不知道引用哪一个。

\verb|\figure| 的 \verb|\caption| 是放在 \verb|\includegraphics| 的下方,而 \verb|\table| 的 \verb|\caption| 是放在 \verb|tabular| 或 \verb|tblr| 环境的上方。

\begin{figure}[htbp]
  \centering
  \includegraphics[width = 5cm]{example-image-a}
  \caption{测试}
  \label{figure:test}
\end{figure}

\begin{table}[htbp]
  \centering
  \caption{测试}
  \label{table:test}
  \begin{tabular}{|c|c|}
    11 & 22 \\
    33 & 44 
  \end{tabular}
\end{table}

图 \ref{figure:test} 和表 \ref{table:test} 用来测试两个浮动体和交叉引用



\section{部分数学符号的输入}

本节主要是一些数学符号的输入介绍


\subsection{直体符号}

科技类论文中,建议一些数学符号使用直体(“up”前缀表示直体)
  \begin{itemize}
    \item 直立的 pi :\verb|\uppi| $\to \uppi$
    \item 直立的 e :\verb|\upe| $\to \upe$
    \item 直立的 i :\verb|\upi| $\to \upi$
  \end{itemize}


\section{参考文献引用}

\subsection{数学类}

行间\parencite[thm 3.1]{zurek2014quantum}

行间\parencite{zurek2014quantum}



\subsection{文科类}

上标\cite[test]{zurek2014quantum}

上标\cite{zurek2014quantum}



\section{《附件4:关于修订毕业论文注释与参考文献著录格式的通知》中的参考文献效果}

  text\parencite{李晓东rawtype}

  text\parencite{Ahnrawtype}

  text\parencite{Ahnrawtype}

  text\parencite{丁文祥rawtype}

  text\parencite{邱泽奇会议论文集rawtype}

  text\parencite{雷光春rawtype}

  text\parencite{zhangrawtype}

  text\parencite{邱泽奇会议论文rawtype}

  text\parencite{马克思rawtype}

  text\parencite{昂温rawtype}

  text\parencite{Fothrawtype}

  text\parencite{杨国枢rawtype}

  text\parencite{Morisonrawtype}

  text\parencite{张志祥rawtype}

  text\parencite{徐秀英rawtype}

  text\parencite{Aldemitarawtype}

  text\parencite{张凯军rawtype}

  text\parencite{Kosekrawtype}

  text\parencite{文献编写rawtype}

  text\parencite{国防白皮rawtype}

  text\parencite{federalrawtype}

  text\parencite{healthrawtype}

  text\parencite{江向东rawtype}

  text\parencite{萧钮rawtype}

  text\parencite{Dublinrawtype}

文字测试| 代码注释掉即可(不建议删除,因为随时可以取消注释查看效果)。


\section{列表环境}


如果要修改 \verb|enumerate| 环境的 label 样式的话:

\begin{enumerate}
  \item 第一项
  \item 第二项
  \item 第三项
  \item 第四项
\end{enumerate}

\begin{enumerate}[1)]
  \item 第一项
  \item 第二项
  \item 第三项
  \item 第四项
\end{enumerate}

\begin{enumerate}[a.]
  \item 第一项
  \item 第二项
  \item 第三项
  \item 第四项
\end{enumerate}

\begin{enumerate}[(A)]
  \item 第一项
  \item 第二项
  \item 第三项
  \item 第四项
\end{enumerate}

test
\begin{enumerate}[(i)]
  \item 第一项
  \item 第二项
  \item 第三项
  \item 第四项
\end{enumerate}

\begin{enumerate}[I]
  \item 第一项
  \item 第二项
  \item 第三项
  \item 第四项
\end{enumerate}

\begin{enumerate}[label = \textbf{断言} \Alph*]
  \item 一般来说使用断言,推荐使用 claim 环境
  \item 但是如果真要有一些断言的层级分化的话
  \item 可以考虑用 enumerate 环境的 label 选项
\end{enumerate}

\begin{enumerate}[\textbf{断言} A]
  \item 1
  \item 2
\end{enumerate}


\section{已定义好的一些数学定理环境}

定理环境内的括号,不管是中文还是西文括号,都不会出现倾斜,不需要像旧模板一样需要用手动用 \verb|\textit| 调整
\begin{definition}[测度]
  (参见文献xxx) 这是一段文字 $E = m c^2$  (中文括号)和 (西文括号)
\end{definition}

\begin{theorem}
  这是一段文字 $E = m c^2$
\end{theorem}


\begin{proof}
  这是一段文字 $E = m c^2$
\end{proof}

\begin{proof}[定理xx的证明]
  这是一段文字 $E = m c^2$
\end{proof}

\begin{example}
  这是一段文字 $E = m c^2$
\end{example}

\begin{property}
  这是一段文字 $E = m c^2$
\end{property}

\begin{proposition}
  这是一段文字 $E = m c^2$
\end{proposition}

\begin{corollary}
  这是一段文字 $E = m c^2$
\end{corollary}

\begin{lemma}
  这是一段文字 $E = m c^2$
\end{lemma}

\begin{axiom}
  这是一段文字 $E = m c^2$
\end{axiom}

\begin{counterexample}
  这是一段文字 $E = m c^2$
\end{counterexample}

\begin{conjecture}
  这是一段文字 $E = m c^2$
\end{conjecture}

\begin{question}
  这是一段文字 $E = m c^2$
\end{question}

\begin{claim}
  这是一段文字 $E = m c^2$
\end{claim}

\begin{remark}
  这是一段文字 $E = m c^2$
\end{remark}

\begin{theorem}[Cauchy]\label{thm:test}
  这是一个定理
  \begin{equation}\label{eq:test1}
    a^2 + b^2 = c^2 \geq 0
  \end{equation}

  \begin{equation}\label{eq:test2}
    a^2 + b^2 = c^2 \geq 0
  \end{equation}
\end{theorem}

我想引用定理~\ref{thm:test} 和公式~\ref{eq:test2}


定理括号测试:

\begin{theorem}
  测试
  \begin{enumerate}
    \item 中文(括号)没输入空格的效果
    \item 中文 (括号) 输入空格的效果
    \item 西文(括号)没输入空格的效果
    \item 西文 (括号) 输入空格的效果
  \end{enumerate}
\end{theorem} 


\begin{proof}
  test
  \[
    a^2 + b^2 = c^2
  \]
\end{proof}

\begin{proof}
  test
  \[
    a^2 + b^2 = c^2  \qedhere
  \]
\end{proof}

\section{浮动体使用}

用 \verb|\label| 引用时,只需要将其放在 \verb|\caption| 的下一行即可。

和定理类环境的引用相同,建议 label 的名称格式为 \verb|figure:xxx| 或 \verb|table:xxx| 其中 \verb|xxx| 可以写中文,尽可能言简意赅地写这个图或表的内容描述,也尽可能写出图表的“独一无二性”,方便自己记忆,也防止在图表一多的时候不知道引用哪一个。

\verb|\figure| 的 \verb|\caption| 是放在 \verb|\includegraphics| 的下方,而 \verb|\table| 的 \verb|\caption| 是放在 \verb|tabular| 或 \verb|tblr| 环境的上方。

\begin{figure}[htbp]
  \centering
  \includegraphics[width = 5cm]{example-image-a}
  \caption{测试}
  \label{figure:test}
\end{figure}

\begin{table}[htbp]
  \centering
  \caption{测试}
  \label{table:test}
  \begin{tabular}{|c|c|}
    11 & 22 \\
    33 & 44 
  \end{tabular}
\end{table}

图 \ref{figure:test} 和表 \ref{table:test} 用来测试两个浮动体和交叉引用



\section{部分数学符号的输入}

本节主要是一些数学符号的输入介绍


\subsection{直体符号}

科技类论文中,建议一些数学符号使用直体(“up”前缀表示直体)
  \begin{itemize}
    \item 直立的 pi :\verb|\uppi| $\to \uppi$
    \item 直立的 e :\verb|\upe| $\to \upe$
    \item 直立的 i :\verb|\upi| $\to \upi$
  \end{itemize}


\section{参考文献引用}

\subsection{数学类}

行间\parencite[thm 3.1]{zurek2014quantum}

行间\parencite{zurek2014quantum}



\subsection{文科类}

上标\cite[test]{zurek2014quantum}

上标\cite{zurek2014quantum}



\section{《附件4:关于修订毕业论文注释与参考文献著录格式的通知》中的参考文献效果}

  text\parencite{李晓东rawtype}

  text\parencite{Ahnrawtype}

  text\parencite{Ahnrawtype}

  text\parencite{丁文祥rawtype}

  text\parencite{邱泽奇会议论文集rawtype}

  text\parencite{雷光春rawtype}

  text\parencite{zhangrawtype}

  text\parencite{邱泽奇会议论文rawtype}

  text\parencite{马克思rawtype}

  text\parencite{昂温rawtype}

  text\parencite{Fothrawtype}

  text\parencite{杨国枢rawtype}

  text\parencite{Morisonrawtype}

  text\parencite{张志祥rawtype}

  text\parencite{徐秀英rawtype}

  text\parencite{Aldemitarawtype}

  text\parencite{张凯军rawtype}

  text\parencite{Kosekrawtype}

  text\parencite{文献编写rawtype}

  text\parencite{国防白皮rawtype}

  text\parencite{federalrawtype}

  text\parencite{healthrawtype}

  text\parencite{江向东rawtype}

  text\parencite{萧钮rawtype}

  text\parencite{Dublinrawtype}

文字测试  % 代码示例文件,不需要的话注释掉或者删掉即可

\chapter{引言}

\section{研究背景}
\begin{figure}[tbh]
  \centering
  \includegraphics[width = 5cm]{example-image-a}
  \caption{测试}
\end{figure}
\begin{table}[tbh]
  \caption{测试}
  \centering
  \begin{tblr}{|c|c|}
    11 & 22 \\
    33 & 44 
  \end{tblr}
\end{table}



\subsection{前人工作}


测试 \parencite{邱泽奇建构与分化}
\section{研究背景}

% !TeX root = ../main.tex

\chapter{引用与链接}

\section{脚注}

华中师范大学《附件4:关于修订毕业论文注释与参考文献著录格式的通知》提到

\begin{itemize}
  \item 文科术科的论文注释使用脚注
  \item 理工科的论文注释不使用脚注
\end{itemize}

\subsection{测试}

\zhlipsum[1]



\section{引用文中小节}\label{sec:ref}

如引用小节~\ref{sec:ref}



\section{引用参考文献}

这是一个参考文献引用的范例:“\parencite{邱泽奇建构与分化}提出……”。还可以引用多个文献:“\parencite{丁文祥rawtype,李晓东rawtype}提出……”。


\section{链接相关}


模板使用了 hyperref 包处理相关链接,使用 \verb|\href| 可以生成超链接,默认不显示链接颜色。如果需要输出网址,可以使用 \verb|\url| 命令,示例:\url{https://github.com}。

% \zhlipsum[1-4]
% \zhlipsum[1-4]
% \zhlipsum[1-4]
% \zhlipsum[1-4]
% \zhlipsum[1-4]
% \zhlipsum[1-4]

\begin{equation}
  x^2
\end{equation}
\chapter{问题研究}


\section{已定义好的一些定理环境}

\begin{definition}[测度]
  (参见文献xxx)这是一段文字 $E = m c^2$
\end{definition}

\begin{theorem}
  这是一段文字 $E = m c^2$
\end{theorem}

\begin{proof}
  这是一段文字 $E = m c^2$
\end{proof}

\begin{proof}[定理xx的证明]
  这是一段文字 $E = m c^2$
  \[
    \int_{0}^{1} x^2 dx
  \]
  okkk
\end{proof}

\begin{example}
  这是一段文字 $E = m c^2$
\end{example}

\begin{property}
  这是一段文字 $E = m c^2$
\end{property}

\begin{proposition}
  这是一段文字 $E = m c^2$
\end{proposition}

\begin{corollary}
  这是一段文字 $E = m c^2$
\end{corollary}

\begin{lemma}
  这是一段文字 $E = m c^2$
\end{lemma}

\begin{axiom}
  这是一段文字 $E = m c^2$
\end{axiom}

\begin{antiexample}
  这是一段文字 $E = m c^2$
\end{antiexample}

\begin{conjecture}
  这是一段文字 $E = m c^2$
\end{conjecture}

\begin{question}
  这是一段文字 $E = m c^2$
\end{question}

\begin{claim}
  这是一段文字 $E = m c^2$
\end{claim}

\begin{remark}
  这是一段文字 $E = m c^2$
\end{remark}


\section{测试}

测试文字,测试文字测试文字测试文字,测试文字测试文字测试文字,测试文字测试文字测试文字,测试文字测试文字



\section{测试}


\subsection{测试}

测试文字,测试文字测试文字测试文字,测试文字测试文字测试文字,测试文字测试文字测试文字,测试文字测试文字


\subsection{测试}

测试文字,测试文字测试文字测试文字,测试文字测试文字测试文字,测试文字测试文字测试文字,测试文字测试文字


\subsection{测试}

测试文字,测试文字测试文字测试文字,测试文字测试文字测试文字,测试文字测试文字测试文字,测试文字测试文字


% !TeX root = ../main.tex

\chapter{公式}



\section{公式引用}

\begin{equation}\label{equation:test1}
  K \leq 
  \begin{dcases}
    2 \lfloor \frac{d}{2d - n} \rfloor    , & K \text{为偶数} ; \\
    2 \lfloor \frac{d}{2d - n} \rfloor - 1, & K \text{为奇数}. \\
  \end{dcases}
  % \left\{ 
  %   \begin{array}{cl}
  %     2 \lfloor \frac{d}{2d - n} \rfloor    , & K \text{为偶数} ; \\
  %     2 \lfloor \frac{d}{2d - n} \rfloor - 1, & K \text{为奇数}. \\
  %   \end{array}
  % \right. 
\end{equation}

公式~\eqref{equation:test1} 的引用



% 论文后文部分,参考文献、致谢、附录等
\backmatter


% 参考文献
%********************************************
% 行内引用:\parencite{} or \parencite[]{},下面两个情况要用行内引用
%    - 去掉这个引用句子结构不完整,比如“定理证明可参看[1]”
%    - 英文文献的引用
% 上标引用:\cite{} or \cite[]{},下面情况要用上标引用
%    - 去掉这个引用,句子结构完整,比如“作者提到,‘CCNUthesis 真是一个好模板。’^[1]”
%      其中“^[1]” 表示上标引用
%********************************************

% 输出参考文献
%   适用学位类型:【本|硕|博】
%   无须用户进行任何操作
%   用户要想正确输出参考文献:
%     1. 在 bib 文件中输入正确的参考文献条目信息
%     2. 在正文中正确地使用 \parencite 或 \cite 引用
%     3. 使用 xelatex-biber-xelatex*2 编译链或 latexmk 方式编译 main.tex 文件
\printbibliography


% 附录
% 【硕|博】附录在致谢前
\include{./back/appendix.tex}

% 攻读学位期间取得的研究成果(博)
% \include{./back/publications.tex}


% 致谢
% !TeX root = ../main.tex

\acknowledgements


以简短的文字表达作者对完成论文和学业提供帮助的老师、同学、领导、同事及亲属的感激之情。


\begin{signature}
  夏康玮 \\
  2022年5月22日于珞珈山
\end{signature}


% 附录
% 【本】附录在致谢后
% \include{./back/appendix.tex}


\end{document}