\documentclass[oneside]{CCNUthesis}

\ccnusetup{
  % style 类用于设置论文格式
  style = {
    font = times,
    % 西文字体(包括数学字体)
    % 允许选项:
    %   font = garamond|libertinus|lm|palatino|times|times*|none
    %
    cjk-font = fandol,
    % 中文字体
    % 允许选项:
    %   cjk-font = adobe|fandol|founder|mac|sinotype|sourcehan|windows|none
    %
    % 注意:
    %   1. 中文字体设置高度依赖于系统。各系统建议方案:
    %        windows:cjk-font = windows
    %        mac:    cjk-font = mac
    %        linux:  cjk-font = fandol(默认值)
    %   2. 除 fandol 和 sourcehan 外,其余字体均为商用字体,请注意版权问题
    %   3. 但 fandol 字体缺字比较严重,而 sourcehan 没有配备楷体和仿宋体
    %   4. 这里中西文字体设置均注释掉了,即使用默认设置:
    %        font     = times
    %        cjk-font = fandol
    %   5. 使用 font = none / cjk-font = none 关闭默认字体设置,需手动进行配置
    %
    footnote-style = xits,
    % 脚注编号样式
    % 允许选项:
    %   footnote-style = plain|libertinus|libertinus*|libertinus-sans|
    %                    pifont|pifont*|pifont-sans|pifont-sans*|
    %                    xits|xits-sans|xits-sans*
    % 默认与西文字体保持一致
    %
    hyperlink = none,
    % 超链接样式
    % 允许选项:
    %   hyperlink = border|color|none
    %
    hyperlink-color = default,
    % 超链接颜色
    % 允许选项:
    %   hyperlink-color = default|classic|material|graylevel|prl
    %
    bib-backend = biblatex,
    % 参考文献支持方式
    % 允许选项:
    %   bib-backend = bibtex|biblatex
    %
    % bib-style = numerical,
    % 参考文献样式
    % 允许选项:
    %   bib-style = author-year|numerical|<其他样式>
    % 说明:
    %   author-year  著者—出版年制
    %   numerical    顺序编码制
    %   <其他样式>   使用其他 .bst(bibtex)或 .bbx(biblatex)格式文件
    %
    % cite-style = {},
    % 引用样式
    % 默认为空,即与参考文献样式保持一致
    % 仅适用于 biblatex;如要填写,需保证相应的 .cbx 格式文件能被调用
    %
    bib-resource = {CCNUthesis-main.bib},
    % 参考文献数据源
    % 可以是单个文件,也可以是用英文逗号 “,” 隔开的一组文件
    % 如果使用 biblatex,则必须明确给出 .bib 后缀名
    },
  info = {
    title = {Lipschitz函数的若干性质},
    major = {数学与统计学学院},
    department = {数学与应用数学(试验)},
    level = {2017},
    author = {夏康玮},
    student-id = {2017214294},
    supervisor = {李工宝 \quad 教授},
    keywords = {Lipschitz函数, 可微性, Hausdorff测度, Hausdorff维数},
    keywords* = {Lipschitz functions, Differentiability, Hausdorff measure, Hausdorff dimension},
  },
}

% 需要的宏包可以自行调用
\usepackage{physics}   % 仅为示例用,可删除

% 需要的命令可以自行定义
\newcommand{\hilbertH}{\symcal{H}}   % 仅为示例用,可删除
\newcommand{\ee}{\symrm{e}}   % 仅为示例用,可删除
\newcommand{\ii}{\symrm{i}}   % 仅为示例用,可删除


\begin{document}
% 前置部分包含目录、中英文摘要以及符号表等
\frontmatter
% 目录
\tableofcontents
\ccnusetup{
  style = { fullwidth-stop = catcode }
    % 是否把全角实心句点 “.” 作为默认的句号形状
    % 允许选项:
    %   fullwidth-stop = catcode|mapping|false
    % 说明:
    %   catcode   显式的 “。” 会被替换为 “.”(e.g. 不包括用宏定义保存的 “。”)
    %   mapping   所有的 “。” 会被替换为 “.”(使用 LuaLaTeX 编译则无效)
    %   false     不进行替换
}


% 中文摘要
\begin{abstract}
  中文摘要
\end{abstract}

% 英文摘要
\begin{abstract*}
  English abstract
\end{abstract*}

% 符号表
% 语法与 LaTeX 表格一致:列用 & 区分,行用 \\ 区分
% 如需修改格式,可以使用可选参数:
%   \begin{notation}[ll]
%     $x$ & 坐标 \\
%     $p$ & 动量
%   \end{notation}
% 可选参数与 LaTeX 标准表格的列格式说明语法一致
% 这里的 “ll” 表示两列均为自动宽度,并且左对齐
\begin{notation}[ll]
  $x$                  & 坐标        \\
  $p$                  & 动量        \\
  $\psi(x)$            & 波函数      \\
\end{notation}



% 主体部分是论文的核心
\mainmatter

% 采用多文件编译的方式
% 把每一章放进一个单独的 tex 文件里,并在这里用 \include 导入
% 例如 \chapter{引言}

\section{研究背景}
\begin{figure}[tbh]
  \centering
  \includegraphics[width = 5cm]{example-image-a}
  \caption{测试}
\end{figure}
\begin{table}[tbh]
  \caption{测试}
  \centering
  \begin{tblr}{|c|c|}
    11 & 22 \\
    33 & 44 
  \end{tblr}
\end{table}



\subsection{前人工作}


测试 \parencite{邱泽奇建构与分化}
\section{研究背景}

% 表示插入main所在目录中的body目录下的chapter1.tex文件

% 章节一般包括【引言】、【预备知识】以及最后的【总结与展望】
% 中间部分灵活掌握
% 文件中的标题仅为示例,可根据自己需要进行修改
\chapter{引言}

\section{研究背景}
\begin{figure}[tbh]
  \centering
  \includegraphics[width = 5cm]{example-image-a}
  \caption{测试}
\end{figure}
\begin{table}[tbh]
  \caption{测试}
  \centering
  \begin{tblr}{|c|c|}
    11 & 22 \\
    33 & 44 
  \end{tblr}
\end{table}



\subsection{前人工作}


测试 \parencite{邱泽奇建构与分化}
\section{研究背景}

% !TeX root = ../main.tex

\chapter{引用与链接}

\section{脚注}

华中师范大学《附件4:关于修订毕业论文注释与参考文献著录格式的通知》提到

\begin{itemize}
  \item 文科术科的论文注释使用脚注
  \item 理工科的论文注释不使用脚注
\end{itemize}

\subsection{测试}

\zhlipsum[1]



\section{引用文中小节}\label{sec:ref}

如引用小节~\ref{sec:ref}



\section{引用参考文献}

这是一个参考文献引用的范例:“\parencite{邱泽奇建构与分化}提出……”。还可以引用多个文献:“\parencite{丁文祥rawtype,李晓东rawtype}提出……”。


\section{链接相关}


模板使用了 hyperref 包处理相关链接,使用 \verb|\href| 可以生成超链接,默认不显示链接颜色。如果需要输出网址,可以使用 \verb|\url| 命令,示例:\url{https://github.com}。

% \zhlipsum[1-4]
% \zhlipsum[1-4]
% \zhlipsum[1-4]
% \zhlipsum[1-4]
% \zhlipsum[1-4]
% \zhlipsum[1-4]

\begin{equation}
  x^2
\end{equation}
\chapter{问题研究}


\section{已定义好的一些定理环境}

\begin{definition}[测度]
  (参见文献xxx)这是一段文字 $E = m c^2$
\end{definition}

\begin{theorem}
  这是一段文字 $E = m c^2$
\end{theorem}

\begin{proof}
  这是一段文字 $E = m c^2$
\end{proof}

\begin{proof}[定理xx的证明]
  这是一段文字 $E = m c^2$
  \[
    \int_{0}^{1} x^2 dx
  \]
  okkk
\end{proof}

\begin{example}
  这是一段文字 $E = m c^2$
\end{example}

\begin{property}
  这是一段文字 $E = m c^2$
\end{property}

\begin{proposition}
  这是一段文字 $E = m c^2$
\end{proposition}

\begin{corollary}
  这是一段文字 $E = m c^2$
\end{corollary}

\begin{lemma}
  这是一段文字 $E = m c^2$
\end{lemma}

\begin{axiom}
  这是一段文字 $E = m c^2$
\end{axiom}

\begin{antiexample}
  这是一段文字 $E = m c^2$
\end{antiexample}

\begin{conjecture}
  这是一段文字 $E = m c^2$
\end{conjecture}

\begin{question}
  这是一段文字 $E = m c^2$
\end{question}

\begin{claim}
  这是一段文字 $E = m c^2$
\end{claim}

\begin{remark}
  这是一段文字 $E = m c^2$
\end{remark}


\section{测试}

测试文字,测试文字测试文字测试文字,测试文字测试文字测试文字,测试文字测试文字测试文字,测试文字测试文字



\section{测试}


\subsection{测试}

测试文字,测试文字测试文字测试文字,测试文字测试文字测试文字,测试文字测试文字测试文字,测试文字测试文字


\subsection{测试}

测试文字,测试文字测试文字测试文字,测试文字测试文字测试文字,测试文字测试文字测试文字,测试文字测试文字


\subsection{测试}

测试文字,测试文字测试文字测试文字,测试文字测试文字测试文字,测试文字测试文字测试文字,测试文字测试文字


% !TeX root = ../main.tex

\chapter{公式}



\section{公式引用}

\begin{equation}\label{equation:test1}
  K \leq 
  \begin{dcases}
    2 \lfloor \frac{d}{2d - n} \rfloor    , & K \text{为偶数} ; \\
    2 \lfloor \frac{d}{2d - n} \rfloor - 1, & K \text{为奇数}. \\
  \end{dcases}
  % \left\{ 
  %   \begin{array}{cl}
  %     2 \lfloor \frac{d}{2d - n} \rfloor    , & K \text{为偶数} ; \\
  %     2 \lfloor \frac{d}{2d - n} \rfloor - 1, & K \text{为奇数}. \\
  %   \end{array}
  % \right. 
\end{equation}

公式~\eqref{equation:test1} 的引用



% 后置部分包含参考文献、声明页(自动生成)等
\backmatter

% 参考文献只需要修改CCNUthesis-main.bib文件
% 打印参考文献列表
\printbibliography


% 致谢
\chapter*{\centerline{致 \quad 谢}}
\addcontentsline{toc}{chapter}{\normalfont 致谢}   % 根据旧模版样式,如果有写致谢,则需要把致谢放进目录


感谢各位的使用,欢迎提出issue和bug!
% 附录
\chapter*{附录}
\addcontentsline{toc}{chapter}{\normalfont 附录}

此处可以写调查问卷,访谈记录等

用\verb|xchoices{}|环境可以排版任意个选项,用\verb|\item|分隔即可(现将代码注释掉,如有需要取消注释即可,如果不需要,可自行删除或不管(因为不影响编译效果)

% \begin{xchoices}
%   \item 选项1
%   \item 选项2
%   \item 选项3
%   \item 选项4
% \end{xchoices}

% \begin{xchoices}[
%   items = 2,            % 手动控制每行多少个选项
%   label-style = alph,   % label的样式:alph, Alph, roman, Roman, 
%                         %            quan(带圈数字), chinese, none
%   pre-label = {(},      % label前面的内容
%   post-label = {)},     % label后面的内容
% ]
%   \item 选项1
%   \item 选项2
%   \item 选项3
%   \item 选项4
% \end{xchoices}

% \begin{xchoices}[label-style = quan]
%   \item 选项1
%   \item 选项2
%   \item 选项3
%   \item 选项4
% \end{xchoices}



\end{document}