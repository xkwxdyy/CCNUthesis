\chapter{研究结论}


\section{复习课课堂提问设计原则}

常言道:“温故而知新。”好的复习,可以让学生的学习事半功倍。

在对复习课教学录像分析的基础上,对复习课的课堂提问设计给出以下几点建议:


\subsection{提问层次需合理}

教师在进行复习课的准备时,相比教材的重点与难点,更为重要的是分析学生学情,也就是学生的实际学习水平。在新授课时,相信教师都已研究过教材的重点与难点,并在课上强调过,但是学生的掌握程度究竟如何呢?这是教师在复习课备课时需要解决的问题。教师对学生已有知识以及思维水平有较为准确把握时,才能确立合适的复习课教学目标,从而设计出难度合适的课堂提问。


\subsection{提问方式需灵活}

教师课堂提问应面向全体学生。若教师常常提问个别学生,那么其他少被注意到的学生的积极性便会大打折扣,不利于调动学生参与课堂的积极性。但所有课堂提问都由全体学生一起作答也不是合理的提问方式,因为每个学生都是不同的个体,个体差异性这一影响因素不容忽视,同样的问题对不同的学生而言难度具有显著差异,所以若每个问题都是集体作答,则会大大降低提问的有效性。因此,教师要根据学生以及知识的具体情况,灵活设计课堂提问方式,以促进学生学习积极性,共同进步。


\subsection{提问语言需清晰}

在进行复习这一阶段前,学生已经学习了一系列新知识,但可能是冗杂的,混乱的,不成体系的。此时,若教师在复习课上的提问表达不够清晰明确,可能会对学生本就混乱的知识结果雪上加霜,打击学生学习热情。教师在教学过程中应尽量准确精炼地表达,不要让语言成为学生学习的另一大障碍。



\section{复习课课堂提问设计案例}

课题《数列 $S_n$ 与 $a_n$的关系》

\begin{question}
  我们已经学习过数列的前 $n$ 项和 $S_n$ 和通项公式 $a_n$ 之间的关系,有没有同学可以带领大家复习一遍推导过程?
\end{question}

\begin{designgoal}
  检测学生已有知识水平,复习巩固知识,由 $S_n=a_1+a_2+a_3+ \cdots +a_n$,写出 $S_{n-1}$,通过观察得到$an= 
  \begin{dcases}
    S_1, & n = 1\\
    S_n - S_{n-1}, & n\geq 2
  \end{dcases}
$关系式,感受退位相减的思想方法。
\end{designgoal}

\begin{example}[2020·江西省信丰中学月考]\label{example:江西省信丰中学月考}
  若数列的前 $n$ 项和 $S_{n}=\frac{2}{3} a_{n}+\frac{1}{3}$, 则的通项公式是 $a_{n}= \underline{\hspace*{3em}}$.
\end{example}


\begin{question}
  例~\ref{example:江西省信丰中学月考} 要我们求解的是什么?
\end{question}

\begin{question}
  例~\ref{example:江西省信丰中学月考} 的条件$S_{n}=\frac{2}{3} a_{n}+\frac{1}{3}$是哪两者之间的关系式?
\end{question}


\begin{designgoal}
  带领学生理解题目,区分题目中的条件以及未知量,抓住题眼,培养解题的目标意识。
\end{designgoal}

\begin{question}
  我们如何实现条件到目标的转化?
\end{question}

\begin{designgoal}
  引导学生思考 $S_n$ 与 $a_n$ 的关系,运用知识点解题。
\end{designgoal}

\begin{question}
  为什么转化关系式中要将 $n$ 分为两种情况?
\end{question}

\begin{designgoal}
  让学生明白数学解题需要注意种种细节,此处需要验证 $n=1$ 的情况。
\end{designgoal}

\begin{example}[2019·山西测试]\label{example:山西测试}
  设正项数列 $\left\{a_{n}\right\}$ 的前 $n$ 项和为 $\mathrm{S}_{n}$, 且 $a_{n}^{2}+1=2 a_{n} S_{n}$, 求数列 $\left\{a_{n}\right\}$ 的通项公式.
\end{example}

\begin{question}
  例~\ref{example:山西测试} 与例~\ref{example:江西省信丰中学月考} 有没有相同之处?
\end{question}

\begin{designgoal}
  引导学生主动分析题目的条件以及未知量,并与例~\ref{example:江西省信丰中学月考} 进行对比,熟悉辅助题目的引入。思考辅助题目,可能有助于学生解决其他题目,提升学生类比分析的思想方法。
\end{designgoal}

\begin{question}
  我们能否运用相似方法解决问题?
\end{question}

\begin{designgoal}
  构建知识缺口。学生动笔解题后便会发现这道例题两个式子无法作差相减,而根据上一题的解题经验,本该可以顺利退位相减解题,如此一来,便可以营造知识缺口,诱导学生形成内在的学习动机以及情感,从而启发学生进行缺口的填补。
\end{designgoal}

\begin{question}
  在 $S_{n}$ 与 $a_{n}$ 的相互转化中, 如果无法完成将 $S_{n}$ 转化为 $a_{n}$ 这个过程, 那么我们能不能反过来,将 $a_{n}$ 转化为 $S_n$ 呢?
\end{question}

\begin{designgoal}
  打破思维定势,启发学生意识到式子的转化可以是双向的。
\end{designgoal}


\begin{question}
  例~\ref{example:山西测试} 与例~\ref{example:江西省信丰中学月考} 条件都是 $S_{n}$ 与 $a_{n}$ 的关系式, 解题目标都是 $a_{n}$, 为什么不能使用同样的方法解题呢?区别在哪里?
\end{question}

\begin{designgoal}
  例~\ref{example:江西省信丰中学月考} 中 $S_n$ 前面系数为 1 , 可以顺利退位相减, 但是例~\ref{example:山西测试} 中 $S_n$ 与 $a_n$ 相乘, 无法退位相减。让学生透过现象看本质, 明白有时不能直接求出 $a_{n}$, 需要 先求解 $S_n$, 再求解 $a_n$, 但万变不离其宗, 本质都是利用 $S_n$ 与 $a_n$ 的关系进行转化 构造。
\end{designgoal}

\begin{question}
  同学们能否尝试对比归纳以上两种解题方法的异同?有什么注意点?
\end{question}

\begin{designgoal}
  复习课的课堂中,随时小结,可以加深学生对解题方法的印象。
\end{designgoal}

相同点: 都适用于条件为 $S_{n}$ 与 $a_{n}$ 的关系式的题目。

不同点: 当 $S_{n}$ 的形式相对独立时, 类比写出 $S_{n-1}$, 作差, 得出 $a_{n}$ 的相关等 式, 进而求解。当 $S_{n}$ 的形式不独立时, 将 $a_{n}$ 转化为 $S_{n}-S_{n-1}(n \geq 2)$, 将题目条件 转变为 $S_n$ 与 $S_{n-1}$ 的关系式, 继而得到 $S_n$ 的相关结论, 从而求解 $a_n$ 。

注意点: 需要验证 $n=1$ 的情况。

引导学生从什么时候用、怎么用、有什么注意事项三个方面总结归纳解题方 法, 形成有序的知识框架。


\begin{example}[2020·尤溪县第五中学高一期末]\label{example:尤溪县第五中学高一期末}
  已知数列 $\left\{a_{n}\right\}$ 满足 $2 a_{1}+2^{2} a_{2}+2^{3} a_{3}+\cdots+2^{n} a_{n}=4^{n}-1$, 则 $a_{n}$ 的通项公式 \underline{\hspace*{3em}}.
\end{example}

\begin{question}
  题目条件“$2 a_{1}+2^{2} a_{2}+2^{3} a_{3}+\cdots+2^{n} a_{n}=4^{n}-1$”,仔细观察等式左边,曾经有没有见过类似形式?
\end{question}

\begin{designgoal}
  引导学生观察并回想错位相减法求前 $n$ 项和的形式,积极调动已有的知识。
\end{designgoal}

\begin{question}
  那么我们能不能重新叙述这道题目?
\end{question}

\begin{designgoal}
  设计意图 令 $b_n=2^{n} a_n$, 则题目可以改写为 “已知数列 $\left\{b_{n}\right\}$ 的前 $n$ 项和 $S_{n}=4 n-1$, 求解数列的 $\{a_{n}\}$ 通项公式.”。引导学生及时引人辅助符号, 简化题目, 由难转易。
\end{designgoal}

\begin{question}
  现在我们已知数列 $\{b_{n}\}$ 的前 $n$ 项和, 你能从已知数据中得出什么?
\end{question}

\begin{question}
  得出数列 $\{b_{n}\}$ 的通项公式后, 我们能否由 $b_{n}$ 求解 $a_{n}$ ?
\end{question}

\begin{designgoal}
  我们可以将解题目标看作对岸,而已知数据和未知量之间就是一条鸿沟,要想跨越鸿沟到达对岸,就必须建造一座桥梁,要引导学生主动分析推导已知数据。
\end{designgoal}

\begin{question}
  有没有同学可以分享一下你从例~\ref{example:尤溪县第五中学高一期末} 学习到了什么?
\end{question}

\begin{designgoal}
  引导学生总结题目特征,加深印象。
\end{designgoal}

\begin{question}
  那么例~\ref{example:尤溪县第五中学高一期末} 与前两个例题本质上是否相同?
\end{question}

\begin{designgoal}
  都是利用 $S_n$ 与 $a_n$ 的关系求解的题型,引导学生把握题目的本质特征。
\end{designgoal}

\begin{question}
  请同学们总结一下本节课最大的收获是什么?
\end{question}

\begin{designgoal}
  由学生进行总结,促使学生回顾整堂课的学习内容,从更高层次把握学习内容。
\end{designgoal}



\section{研究不足}

由于疫情,笔者只能选取一师一优课网站上的教学录像视频进行分析,但是教学录像视频与实际课堂也许还有一定的差距。如果能够深入实际课堂,长期记录并统计某几位老师的复习课课堂提问,相信能够得到更加准确且更有分析意义的数据。