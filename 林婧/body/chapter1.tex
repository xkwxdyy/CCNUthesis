\chapter{引言}


教育对一个国家的方方面面都起到重要作用。北宋张横渠曾经说过:“为天地立心,为生民立命。为往圣继绝学,为万世开太平!”教育,关乎一个民族的文化传承,关乎人民的生活起居,关乎世界的和平发展。习近平总书记也明确解释了教育的地位与作用:“教育是提高人民综合素质、促进人的全面发展的重要途径,是民族振兴、社会进步的重要基石。”中国的持续发展使得人民对教育的要求不断提高,推动教育改革对建设具有中国特色的教育制度有十分重要的意义。在过去,对高考成绩的追求使人们往往集中注意力于能够提高成绩的教学手段即“填鸭式教学”而忽略了由此引发的种种问题,例如学生的创新能力。成绩的提升不能以牺牲学生创造力为代价,因此,寻求有效教学手段促进学生成绩提升的同时培养学生自主思考的能力,协调学生成绩提升和“死记硬背”之间的矛盾是当前教育发展进程中亟待解决的问题。

从古至今,提问便是一种重要的教学手段。中国的伟大教育家孔子曾经说过:“不愤不启,不悱不发,举一隅不以三隅反,则不复也。” 孔子认为,教师应当在学生冥思苦想而不得解、思绪万千而无以诉的时候启发学生。而在这启发式教学的过程中,最常用的教学手段便是诘问,即提问。古希腊的教育家苏格拉底使用“产婆术”帮助人们获取知识。苏格拉底在与别人谈话的时候,会不断发问,以揭示对方话语中自相矛盾之处,从具体事例出发,不断深入,逐步反驳错误思想,从而走向理性认识。

近年来,课堂提问这一教学手段已引起了世界各国的广泛关注,并成为了众多学者探讨的焦点。不论是美国教学论专家克拉克和斯塔尔所提出的课堂提问的19种功能,还是还是国内学者姚安娣所提出的提问的7种作用,都在告诉我们,课堂提问不只是简单的一问一答的过程,而是一种具有深远影响与深刻意义的教学行为,值得我们不断深入探讨。

从现有文献来看,课堂提问的研究大多以普通教育作为考察对象,而鲜有学者围绕数学课堂教学的提问设计展开深度探讨,而针对数学复习课型的课堂提问研究更是少之又少。本文将以数学课堂教学录像作为研究对象,采用录像分析法,通过反复观看录像,对复习课这种数学课型的课堂提问进行收集,运用布鲁姆学习分类法对问题进行分类,进一步分析在复习课这一数学课型当中课堂提问应遵循何种原则。在新课改的背景下研究课堂提问这一教学手段,可以扩展教育学界对提问教学的认识,推进与之相关的有效性提问、提问数量、教室理答方式等研究进程,为数学教学的课堂提问设计提供必要的参考依据。