\documentclass{CCNUthesis}

\ccnusetup{
  % 个人信息
  info = {
    % 主标题,会自动换行
      % 如果换行点不满意,可以用 \\ 手动换行
    title = {
      高中数学复习课堂的提问设计
    },
    % 副标题,如果有的话取消下面三行代码的注释并且填写内容
    % subtitle = {
      % 这是一个副标题
    % },
    % 论文的英文标题
    title* = {
      The design of questioning in senior math review class
    }, 
    % 学院
    department = {数学与统计学学院},
    % 专业
    major = {数学与应用数学(师范)},
    % 年级
    level = {2018级},
    % 姓名
    author = {林婧},
    % 学号
    student-id = {2018213711},
    % 指导教师
    supervisor = {刘敏思 \quad 教授},
    % 论文中文关键词,用英文“,”隔开
    keywords = {
      高中数学,
      课堂提问,
      复习课,
      录像分析
    },
    % 论文英文关键词,用英文“,”隔开
    keywords* = {
      senior maths,
      classroom questioning,
      review class,
      video analysis
    },
    % 如有需要可以手动调整封面底的年和月,否则默认为「编译时」的年和月
    % year    = {2022},
    % month = {4}
  },
  style = {
    cjk-font = mac,
      % 中文字体
      % 允许选项:
      %   cjk-font = adobe|fandol|founder|mac|sinotype|sourcehan|windows|none
      % 注意:
      %   1. 中文字体设置高度依赖于系统。各系统建议方案:
      %        windows:cjk-font = windows
      %        mac:    cjk-font = mac
      %        linux:  cjk-font = fandol(默认值)
      %   2. 除 fandol 和 sourcehan 外,其余字体均为商用字体,请注意版权问题
      %   3. 但 fandol 字体缺字比较严重,而 sourcehan 没有配备楷体和仿宋体
    bib-resource = {CCNUthesis-main.bib}
      % 参考文献数据源,需要加bib后缀
  }
}

%%%%% 需要的宏包可以在此处自行调用 %%%%%



% 需要的命令环境可以自行定义
\ctexset{chapter/break={}}


\begin{document}

\ccnusetup{
  style = {fullwidth-stop = false}
    % 是否把全角实心句点 “.” 作为默认的句号形状,即正文中输入“。” 最终编译效果为“. ”
    % 一般科技类文章需要替换,防止“. ”与“。”混淆
    % 【放在此处是为了防止干扰版权页的句号设置】
    % 允许选项:
    %   fullwidth-stop = catcode|mapping|false
    % 说明:
    %   catcode   显式的 “。” 会被替换为 “.”(e.g. 不包括用宏定义保存的 “。”)
    %   mapping   所有的 “。” 会被替换为 “.”(使用 LuaLaTeX 编译则无效)
    %   false     不进行替换
}



% \frontmatter开启论文前置部分
% 前置部分包含目录、中英文摘要以及符号表等
\frontmatter


% 目录
\tableofcontents


% 中文摘要
\begin{abstract}
  在新课改的背景下,当前的数学课堂提问仍存在着许多误区。目前多数对于课堂提问的研究都是面向普通教育的,少有针对数学学科教学复习课型提问相关研究。本研究目的在于厘清高中数学复习课课堂提问有效性及影响因素,为复习课课堂提问设计提供参考依据。通过分析两节函数复习课教学录像的课堂提问数据,依据布鲁姆六大教学目标层次,采用文献研究法和录像分析法,对数学复习课课堂提问现状进行了探究,并对比了不同提问层次、提问方式的有效性。分析结果显示,课堂提问层次需合理,提问方式需灵活,提问语言需清晰。在此基础上,提供了数列复习课提问设计案例,供需要者提供一些参考。
\end{abstract}

% 英文摘要
\begin{abstract*}
  In the context of the new curriculum reform, there are still many misunderstandings about asking questions in the current math class. Currently, the majority of classroom questioning research is focused on general education, with only a few studies devoted to the classroom questioning methods in the mathematical review class. The purpose of this study is to clarify the effectiveness and influencing factors of questioning in high school mathematics review class, and to provide a reference for the design of review class questions. The literature research method and the video analysis method were used to explore the current situation of the mathematics review class questions, and the effectiveness of different question levels and questioning methods was compared by analyzing the classroom questioning data of the teaching videos of the two function review lessons, according to the six teaching objective levels of Bloom. According to the findings, the degree of questioning in the classroom should be acceptable, the manner in which questions are asked should be flexible, and the language used to ask questions should be clear. On this foundation, a review class questioning design case is presented, which can serve as a resource for most teachers.
\end{abstract*}


% 符号表
% 语法与 LaTeX 表格一致:列用 & 区分,行用 \\ 区分
% 如需修改格式,可以使用可选参数:
%   \begin{notation}[ll]
%     $x$ & 坐标 \\
%     $p$ & 动量
%   \end{notation}
% 可选参数与 LaTeX 标准表格的列格式说明语法一致
% 这里的 “ll” 表示两列均为自动宽度,并且左对齐
% 注:如果不需要符号表的话就把"\begin{notation}...\end{notation}"注释掉
% \begin{notation}[ll]
%   $x$                  & 坐标        \\
%   $p$                  & 动量        \\
%   $\psi(x)$            & 波函数      \\
% \end{notation}




% \mainmatter进入论文主体部分
% 主体部分是论文的核心
\mainmatter

% 主体采用多文件编译的方式
% 即把每一章放进一个单独的 tex 文件里,并在这里用 \include 导入
% 例如 \chapter{引言}

\section{研究背景}
\begin{figure}[tbh]
  \centering
  \includegraphics[width = 5cm]{example-image-a}
  \caption{测试}
\end{figure}
\begin{table}[tbh]
  \caption{测试}
  \centering
  \begin{tblr}{|c|c|}
    11 & 22 \\
    33 & 44 
  \end{tblr}
\end{table}



\subsection{前人工作}


测试 \parencite{邱泽奇建构与分化}
\section{研究背景}

% 表示插入 main 所在目录中的 body 目录下的 chapter1.tex 文件


% % !TeX root = ../main.tex

\chapter{常用命令环境示例}

此章用于展示一些常用的命令环境的效果,用户在具体写论文过程中可以用来参考效果,如果不需要此章出现在正文中,只需要将主文件 \verb|main.tex| 文件中的 \verb|% !TeX root = ../main.tex

\chapter{常用命令环境示例}

此章用于展示一些常用的命令环境的效果,用户在具体写论文过程中可以用来参考效果,如果不需要此章出现在正文中,只需要将主文件 \verb|main.tex| 文件中的 \verb|% !TeX root = ../main.tex

\chapter{常用命令环境示例}

此章用于展示一些常用的命令环境的效果,用户在具体写论文过程中可以用来参考效果,如果不需要此章出现在正文中,只需要将主文件 \verb|main.tex| 文件中的 \verb|\input{./body/chapter0.tex}| 代码注释掉即可(不建议删除,因为随时可以取消注释查看效果)。


\section{列表环境}


如果要修改 \verb|enumerate| 环境的 label 样式的话:

\begin{enumerate}
  \item 第一项
  \item 第二项
  \item 第三项
  \item 第四项
\end{enumerate}

\begin{enumerate}[1)]
  \item 第一项
  \item 第二项
  \item 第三项
  \item 第四项
\end{enumerate}

\begin{enumerate}[a.]
  \item 第一项
  \item 第二项
  \item 第三项
  \item 第四项
\end{enumerate}

\begin{enumerate}[(A)]
  \item 第一项
  \item 第二项
  \item 第三项
  \item 第四项
\end{enumerate}

test
\begin{enumerate}[(i)]
  \item 第一项
  \item 第二项
  \item 第三项
  \item 第四项
\end{enumerate}

\begin{enumerate}[I]
  \item 第一项
  \item 第二项
  \item 第三项
  \item 第四项
\end{enumerate}

\begin{enumerate}[label = \textbf{断言} \Alph*]
  \item 一般来说使用断言,推荐使用 claim 环境
  \item 但是如果真要有一些断言的层级分化的话
  \item 可以考虑用 enumerate 环境的 label 选项
\end{enumerate}

\begin{enumerate}[\textbf{断言} A]
  \item 1
  \item 2
\end{enumerate}


\section{已定义好的一些数学定理环境}

定理环境内的括号,不管是中文还是西文括号,都不会出现倾斜,不需要像旧模板一样需要用手动用 \verb|\textit| 调整
\begin{definition}[测度]
  (参见文献xxx) 这是一段文字 $E = m c^2$  (中文括号)和 (西文括号)
\end{definition}

\begin{theorem}
  这是一段文字 $E = m c^2$
\end{theorem}


\begin{proof}
  这是一段文字 $E = m c^2$
\end{proof}

\begin{proof}[定理xx的证明]
  这是一段文字 $E = m c^2$
\end{proof}

\begin{example}
  这是一段文字 $E = m c^2$
\end{example}

\begin{property}
  这是一段文字 $E = m c^2$
\end{property}

\begin{proposition}
  这是一段文字 $E = m c^2$
\end{proposition}

\begin{corollary}
  这是一段文字 $E = m c^2$
\end{corollary}

\begin{lemma}
  这是一段文字 $E = m c^2$
\end{lemma}

\begin{axiom}
  这是一段文字 $E = m c^2$
\end{axiom}

\begin{counterexample}
  这是一段文字 $E = m c^2$
\end{counterexample}

\begin{conjecture}
  这是一段文字 $E = m c^2$
\end{conjecture}

\begin{question}
  这是一段文字 $E = m c^2$
\end{question}

\begin{claim}
  这是一段文字 $E = m c^2$
\end{claim}

\begin{remark}
  这是一段文字 $E = m c^2$
\end{remark}

\begin{theorem}[Cauchy]\label{thm:test}
  这是一个定理
  \begin{equation}\label{eq:test1}
    a^2 + b^2 = c^2 \geq 0
  \end{equation}

  \begin{equation}\label{eq:test2}
    a^2 + b^2 = c^2 \geq 0
  \end{equation}
\end{theorem}

我想引用定理~\ref{thm:test} 和公式~\ref{eq:test2}


定理括号测试:

\begin{theorem}
  测试
  \begin{enumerate}
    \item 中文(括号)没输入空格的效果
    \item 中文 (括号) 输入空格的效果
    \item 西文(括号)没输入空格的效果
    \item 西文 (括号) 输入空格的效果
  \end{enumerate}
\end{theorem} 


\begin{proof}
  test
  \[
    a^2 + b^2 = c^2
  \]
\end{proof}

\begin{proof}
  test
  \[
    a^2 + b^2 = c^2  \qedhere
  \]
\end{proof}

\section{浮动体使用}

用 \verb|\label| 引用时,只需要将其放在 \verb|\caption| 的下一行即可。

和定理类环境的引用相同,建议 label 的名称格式为 \verb|figure:xxx| 或 \verb|table:xxx| 其中 \verb|xxx| 可以写中文,尽可能言简意赅地写这个图或表的内容描述,也尽可能写出图表的“独一无二性”,方便自己记忆,也防止在图表一多的时候不知道引用哪一个。

\verb|\figure| 的 \verb|\caption| 是放在 \verb|\includegraphics| 的下方,而 \verb|\table| 的 \verb|\caption| 是放在 \verb|tabular| 或 \verb|tblr| 环境的上方。

\begin{figure}[htbp]
  \centering
  \includegraphics[width = 5cm]{example-image-a}
  \caption{测试}
  \label{figure:test}
\end{figure}

\begin{table}[htbp]
  \centering
  \caption{测试}
  \label{table:test}
  \begin{tabular}{|c|c|}
    11 & 22 \\
    33 & 44 
  \end{tabular}
\end{table}

图 \ref{figure:test} 和表 \ref{table:test} 用来测试两个浮动体和交叉引用



\section{部分数学符号的输入}

本节主要是一些数学符号的输入介绍


\subsection{直体符号}

科技类论文中,建议一些数学符号使用直体(“up”前缀表示直体)
  \begin{itemize}
    \item 直立的 pi :\verb|\uppi| $\to \uppi$
    \item 直立的 e :\verb|\upe| $\to \upe$
    \item 直立的 i :\verb|\upi| $\to \upi$
  \end{itemize}


\section{参考文献引用}

\subsection{数学类}

行间\parencite[thm 3.1]{zurek2014quantum}

行间\parencite{zurek2014quantum}



\subsection{文科类}

上标\cite[test]{zurek2014quantum}

上标\cite{zurek2014quantum}



\section{《附件4:关于修订毕业论文注释与参考文献著录格式的通知》中的参考文献效果}

  text\parencite{李晓东rawtype}

  text\parencite{Ahnrawtype}

  text\parencite{Ahnrawtype}

  text\parencite{丁文祥rawtype}

  text\parencite{邱泽奇会议论文集rawtype}

  text\parencite{雷光春rawtype}

  text\parencite{zhangrawtype}

  text\parencite{邱泽奇会议论文rawtype}

  text\parencite{马克思rawtype}

  text\parencite{昂温rawtype}

  text\parencite{Fothrawtype}

  text\parencite{杨国枢rawtype}

  text\parencite{Morisonrawtype}

  text\parencite{张志祥rawtype}

  text\parencite{徐秀英rawtype}

  text\parencite{Aldemitarawtype}

  text\parencite{张凯军rawtype}

  text\parencite{Kosekrawtype}

  text\parencite{文献编写rawtype}

  text\parencite{国防白皮rawtype}

  text\parencite{federalrawtype}

  text\parencite{healthrawtype}

  text\parencite{江向东rawtype}

  text\parencite{萧钮rawtype}

  text\parencite{Dublinrawtype}

文字测试| 代码注释掉即可(不建议删除,因为随时可以取消注释查看效果)。


\section{列表环境}


如果要修改 \verb|enumerate| 环境的 label 样式的话:

\begin{enumerate}
  \item 第一项
  \item 第二项
  \item 第三项
  \item 第四项
\end{enumerate}

\begin{enumerate}[1)]
  \item 第一项
  \item 第二项
  \item 第三项
  \item 第四项
\end{enumerate}

\begin{enumerate}[a.]
  \item 第一项
  \item 第二项
  \item 第三项
  \item 第四项
\end{enumerate}

\begin{enumerate}[(A)]
  \item 第一项
  \item 第二项
  \item 第三项
  \item 第四项
\end{enumerate}

test
\begin{enumerate}[(i)]
  \item 第一项
  \item 第二项
  \item 第三项
  \item 第四项
\end{enumerate}

\begin{enumerate}[I]
  \item 第一项
  \item 第二项
  \item 第三项
  \item 第四项
\end{enumerate}

\begin{enumerate}[label = \textbf{断言} \Alph*]
  \item 一般来说使用断言,推荐使用 claim 环境
  \item 但是如果真要有一些断言的层级分化的话
  \item 可以考虑用 enumerate 环境的 label 选项
\end{enumerate}

\begin{enumerate}[\textbf{断言} A]
  \item 1
  \item 2
\end{enumerate}


\section{已定义好的一些数学定理环境}

定理环境内的括号,不管是中文还是西文括号,都不会出现倾斜,不需要像旧模板一样需要用手动用 \verb|\textit| 调整
\begin{definition}[测度]
  (参见文献xxx) 这是一段文字 $E = m c^2$  (中文括号)和 (西文括号)
\end{definition}

\begin{theorem}
  这是一段文字 $E = m c^2$
\end{theorem}


\begin{proof}
  这是一段文字 $E = m c^2$
\end{proof}

\begin{proof}[定理xx的证明]
  这是一段文字 $E = m c^2$
\end{proof}

\begin{example}
  这是一段文字 $E = m c^2$
\end{example}

\begin{property}
  这是一段文字 $E = m c^2$
\end{property}

\begin{proposition}
  这是一段文字 $E = m c^2$
\end{proposition}

\begin{corollary}
  这是一段文字 $E = m c^2$
\end{corollary}

\begin{lemma}
  这是一段文字 $E = m c^2$
\end{lemma}

\begin{axiom}
  这是一段文字 $E = m c^2$
\end{axiom}

\begin{counterexample}
  这是一段文字 $E = m c^2$
\end{counterexample}

\begin{conjecture}
  这是一段文字 $E = m c^2$
\end{conjecture}

\begin{question}
  这是一段文字 $E = m c^2$
\end{question}

\begin{claim}
  这是一段文字 $E = m c^2$
\end{claim}

\begin{remark}
  这是一段文字 $E = m c^2$
\end{remark}

\begin{theorem}[Cauchy]\label{thm:test}
  这是一个定理
  \begin{equation}\label{eq:test1}
    a^2 + b^2 = c^2 \geq 0
  \end{equation}

  \begin{equation}\label{eq:test2}
    a^2 + b^2 = c^2 \geq 0
  \end{equation}
\end{theorem}

我想引用定理~\ref{thm:test} 和公式~\ref{eq:test2}


定理括号测试:

\begin{theorem}
  测试
  \begin{enumerate}
    \item 中文(括号)没输入空格的效果
    \item 中文 (括号) 输入空格的效果
    \item 西文(括号)没输入空格的效果
    \item 西文 (括号) 输入空格的效果
  \end{enumerate}
\end{theorem} 


\begin{proof}
  test
  \[
    a^2 + b^2 = c^2
  \]
\end{proof}

\begin{proof}
  test
  \[
    a^2 + b^2 = c^2  \qedhere
  \]
\end{proof}

\section{浮动体使用}

用 \verb|\label| 引用时,只需要将其放在 \verb|\caption| 的下一行即可。

和定理类环境的引用相同,建议 label 的名称格式为 \verb|figure:xxx| 或 \verb|table:xxx| 其中 \verb|xxx| 可以写中文,尽可能言简意赅地写这个图或表的内容描述,也尽可能写出图表的“独一无二性”,方便自己记忆,也防止在图表一多的时候不知道引用哪一个。

\verb|\figure| 的 \verb|\caption| 是放在 \verb|\includegraphics| 的下方,而 \verb|\table| 的 \verb|\caption| 是放在 \verb|tabular| 或 \verb|tblr| 环境的上方。

\begin{figure}[htbp]
  \centering
  \includegraphics[width = 5cm]{example-image-a}
  \caption{测试}
  \label{figure:test}
\end{figure}

\begin{table}[htbp]
  \centering
  \caption{测试}
  \label{table:test}
  \begin{tabular}{|c|c|}
    11 & 22 \\
    33 & 44 
  \end{tabular}
\end{table}

图 \ref{figure:test} 和表 \ref{table:test} 用来测试两个浮动体和交叉引用



\section{部分数学符号的输入}

本节主要是一些数学符号的输入介绍


\subsection{直体符号}

科技类论文中,建议一些数学符号使用直体(“up”前缀表示直体)
  \begin{itemize}
    \item 直立的 pi :\verb|\uppi| $\to \uppi$
    \item 直立的 e :\verb|\upe| $\to \upe$
    \item 直立的 i :\verb|\upi| $\to \upi$
  \end{itemize}


\section{参考文献引用}

\subsection{数学类}

行间\parencite[thm 3.1]{zurek2014quantum}

行间\parencite{zurek2014quantum}



\subsection{文科类}

上标\cite[test]{zurek2014quantum}

上标\cite{zurek2014quantum}



\section{《附件4:关于修订毕业论文注释与参考文献著录格式的通知》中的参考文献效果}

  text\parencite{李晓东rawtype}

  text\parencite{Ahnrawtype}

  text\parencite{Ahnrawtype}

  text\parencite{丁文祥rawtype}

  text\parencite{邱泽奇会议论文集rawtype}

  text\parencite{雷光春rawtype}

  text\parencite{zhangrawtype}

  text\parencite{邱泽奇会议论文rawtype}

  text\parencite{马克思rawtype}

  text\parencite{昂温rawtype}

  text\parencite{Fothrawtype}

  text\parencite{杨国枢rawtype}

  text\parencite{Morisonrawtype}

  text\parencite{张志祥rawtype}

  text\parencite{徐秀英rawtype}

  text\parencite{Aldemitarawtype}

  text\parencite{张凯军rawtype}

  text\parencite{Kosekrawtype}

  text\parencite{文献编写rawtype}

  text\parencite{国防白皮rawtype}

  text\parencite{federalrawtype}

  text\parencite{healthrawtype}

  text\parencite{江向东rawtype}

  text\parencite{萧钮rawtype}

  text\parencite{Dublinrawtype}

文字测试| 代码注释掉即可(不建议删除,因为随时可以取消注释查看效果)。


\section{列表环境}


如果要修改 \verb|enumerate| 环境的 label 样式的话:

\begin{enumerate}
  \item 第一项
  \item 第二项
  \item 第三项
  \item 第四项
\end{enumerate}

\begin{enumerate}[1)]
  \item 第一项
  \item 第二项
  \item 第三项
  \item 第四项
\end{enumerate}

\begin{enumerate}[a.]
  \item 第一项
  \item 第二项
  \item 第三项
  \item 第四项
\end{enumerate}

\begin{enumerate}[(A)]
  \item 第一项
  \item 第二项
  \item 第三项
  \item 第四项
\end{enumerate}

test
\begin{enumerate}[(i)]
  \item 第一项
  \item 第二项
  \item 第三项
  \item 第四项
\end{enumerate}

\begin{enumerate}[I]
  \item 第一项
  \item 第二项
  \item 第三项
  \item 第四项
\end{enumerate}

\begin{enumerate}[label = \textbf{断言} \Alph*]
  \item 一般来说使用断言,推荐使用 claim 环境
  \item 但是如果真要有一些断言的层级分化的话
  \item 可以考虑用 enumerate 环境的 label 选项
\end{enumerate}

\begin{enumerate}[\textbf{断言} A]
  \item 1
  \item 2
\end{enumerate}


\section{已定义好的一些数学定理环境}

定理环境内的括号,不管是中文还是西文括号,都不会出现倾斜,不需要像旧模板一样需要用手动用 \verb|\textit| 调整
\begin{definition}[测度]
  (参见文献xxx) 这是一段文字 $E = m c^2$  (中文括号)和 (西文括号)
\end{definition}

\begin{theorem}
  这是一段文字 $E = m c^2$
\end{theorem}


\begin{proof}
  这是一段文字 $E = m c^2$
\end{proof}

\begin{proof}[定理xx的证明]
  这是一段文字 $E = m c^2$
\end{proof}

\begin{example}
  这是一段文字 $E = m c^2$
\end{example}

\begin{property}
  这是一段文字 $E = m c^2$
\end{property}

\begin{proposition}
  这是一段文字 $E = m c^2$
\end{proposition}

\begin{corollary}
  这是一段文字 $E = m c^2$
\end{corollary}

\begin{lemma}
  这是一段文字 $E = m c^2$
\end{lemma}

\begin{axiom}
  这是一段文字 $E = m c^2$
\end{axiom}

\begin{counterexample}
  这是一段文字 $E = m c^2$
\end{counterexample}

\begin{conjecture}
  这是一段文字 $E = m c^2$
\end{conjecture}

\begin{question}
  这是一段文字 $E = m c^2$
\end{question}

\begin{claim}
  这是一段文字 $E = m c^2$
\end{claim}

\begin{remark}
  这是一段文字 $E = m c^2$
\end{remark}

\begin{theorem}[Cauchy]\label{thm:test}
  这是一个定理
  \begin{equation}\label{eq:test1}
    a^2 + b^2 = c^2 \geq 0
  \end{equation}

  \begin{equation}\label{eq:test2}
    a^2 + b^2 = c^2 \geq 0
  \end{equation}
\end{theorem}

我想引用定理~\ref{thm:test} 和公式~\ref{eq:test2}


定理括号测试:

\begin{theorem}
  测试
  \begin{enumerate}
    \item 中文(括号)没输入空格的效果
    \item 中文 (括号) 输入空格的效果
    \item 西文(括号)没输入空格的效果
    \item 西文 (括号) 输入空格的效果
  \end{enumerate}
\end{theorem} 


\begin{proof}
  test
  \[
    a^2 + b^2 = c^2
  \]
\end{proof}

\begin{proof}
  test
  \[
    a^2 + b^2 = c^2  \qedhere
  \]
\end{proof}

\section{浮动体使用}

用 \verb|\label| 引用时,只需要将其放在 \verb|\caption| 的下一行即可。

和定理类环境的引用相同,建议 label 的名称格式为 \verb|figure:xxx| 或 \verb|table:xxx| 其中 \verb|xxx| 可以写中文,尽可能言简意赅地写这个图或表的内容描述,也尽可能写出图表的“独一无二性”,方便自己记忆,也防止在图表一多的时候不知道引用哪一个。

\verb|\figure| 的 \verb|\caption| 是放在 \verb|\includegraphics| 的下方,而 \verb|\table| 的 \verb|\caption| 是放在 \verb|tabular| 或 \verb|tblr| 环境的上方。

\begin{figure}[htbp]
  \centering
  \includegraphics[width = 5cm]{example-image-a}
  \caption{测试}
  \label{figure:test}
\end{figure}

\begin{table}[htbp]
  \centering
  \caption{测试}
  \label{table:test}
  \begin{tabular}{|c|c|}
    11 & 22 \\
    33 & 44 
  \end{tabular}
\end{table}

图 \ref{figure:test} 和表 \ref{table:test} 用来测试两个浮动体和交叉引用



\section{部分数学符号的输入}

本节主要是一些数学符号的输入介绍


\subsection{直体符号}

科技类论文中,建议一些数学符号使用直体(“up”前缀表示直体)
  \begin{itemize}
    \item 直立的 pi :\verb|\uppi| $\to \uppi$
    \item 直立的 e :\verb|\upe| $\to \upe$
    \item 直立的 i :\verb|\upi| $\to \upi$
  \end{itemize}


\section{参考文献引用}

\subsection{数学类}

行间\parencite[thm 3.1]{zurek2014quantum}

行间\parencite{zurek2014quantum}



\subsection{文科类}

上标\cite[test]{zurek2014quantum}

上标\cite{zurek2014quantum}



\section{《附件4:关于修订毕业论文注释与参考文献著录格式的通知》中的参考文献效果}

  text\parencite{李晓东rawtype}

  text\parencite{Ahnrawtype}

  text\parencite{Ahnrawtype}

  text\parencite{丁文祥rawtype}

  text\parencite{邱泽奇会议论文集rawtype}

  text\parencite{雷光春rawtype}

  text\parencite{zhangrawtype}

  text\parencite{邱泽奇会议论文rawtype}

  text\parencite{马克思rawtype}

  text\parencite{昂温rawtype}

  text\parencite{Fothrawtype}

  text\parencite{杨国枢rawtype}

  text\parencite{Morisonrawtype}

  text\parencite{张志祥rawtype}

  text\parencite{徐秀英rawtype}

  text\parencite{Aldemitarawtype}

  text\parencite{张凯军rawtype}

  text\parencite{Kosekrawtype}

  text\parencite{文献编写rawtype}

  text\parencite{国防白皮rawtype}

  text\parencite{federalrawtype}

  text\parencite{healthrawtype}

  text\parencite{江向东rawtype}

  text\parencite{萧钮rawtype}

  text\parencite{Dublinrawtype}

文字测试   % 常用命令环境示例,不需要时注释掉即可
\chapter{引言}

\section{研究背景}
\begin{figure}[tbh]
  \centering
  \includegraphics[width = 5cm]{example-image-a}
  \caption{测试}
\end{figure}
\begin{table}[tbh]
  \caption{测试}
  \centering
  \begin{tblr}{|c|c|}
    11 & 22 \\
    33 & 44 
  \end{tblr}
\end{table}



\subsection{前人工作}


测试 \parencite{邱泽奇建构与分化}
\section{研究背景}

% !TeX root = ../main.tex

\chapter{引用与链接}

\section{脚注}

华中师范大学《附件4:关于修订毕业论文注释与参考文献著录格式的通知》提到

\begin{itemize}
  \item 文科术科的论文注释使用脚注
  \item 理工科的论文注释不使用脚注
\end{itemize}

\subsection{测试}

\zhlipsum[1]



\section{引用文中小节}\label{sec:ref}

如引用小节~\ref{sec:ref}



\section{引用参考文献}

这是一个参考文献引用的范例:“\parencite{邱泽奇建构与分化}提出……”。还可以引用多个文献:“\parencite{丁文祥rawtype,李晓东rawtype}提出……”。


\section{链接相关}


模板使用了 hyperref 包处理相关链接,使用 \verb|\href| 可以生成超链接,默认不显示链接颜色。如果需要输出网址,可以使用 \verb|\url| 命令,示例:\url{https://github.com}。

% \zhlipsum[1-4]
% \zhlipsum[1-4]
% \zhlipsum[1-4]
% \zhlipsum[1-4]
% \zhlipsum[1-4]
% \zhlipsum[1-4]

\begin{equation}
  x^2
\end{equation}
\chapter{问题研究}


\section{已定义好的一些定理环境}

\begin{definition}[测度]
  (参见文献xxx)这是一段文字 $E = m c^2$
\end{definition}

\begin{theorem}
  这是一段文字 $E = m c^2$
\end{theorem}

\begin{proof}
  这是一段文字 $E = m c^2$
\end{proof}

\begin{proof}[定理xx的证明]
  这是一段文字 $E = m c^2$
  \[
    \int_{0}^{1} x^2 dx
  \]
  okkk
\end{proof}

\begin{example}
  这是一段文字 $E = m c^2$
\end{example}

\begin{property}
  这是一段文字 $E = m c^2$
\end{property}

\begin{proposition}
  这是一段文字 $E = m c^2$
\end{proposition}

\begin{corollary}
  这是一段文字 $E = m c^2$
\end{corollary}

\begin{lemma}
  这是一段文字 $E = m c^2$
\end{lemma}

\begin{axiom}
  这是一段文字 $E = m c^2$
\end{axiom}

\begin{antiexample}
  这是一段文字 $E = m c^2$
\end{antiexample}

\begin{conjecture}
  这是一段文字 $E = m c^2$
\end{conjecture}

\begin{question}
  这是一段文字 $E = m c^2$
\end{question}

\begin{claim}
  这是一段文字 $E = m c^2$
\end{claim}

\begin{remark}
  这是一段文字 $E = m c^2$
\end{remark}


\section{测试}

测试文字,测试文字测试文字测试文字,测试文字测试文字测试文字,测试文字测试文字测试文字,测试文字测试文字



\section{测试}


\subsection{测试}

测试文字,测试文字测试文字测试文字,测试文字测试文字测试文字,测试文字测试文字测试文字,测试文字测试文字


\subsection{测试}

测试文字,测试文字测试文字测试文字,测试文字测试文字测试文字,测试文字测试文字测试文字,测试文字测试文字


\subsection{测试}

测试文字,测试文字测试文字测试文字,测试文字测试文字测试文字,测试文字测试文字测试文字,测试文字测试文字


% !TeX root = ../main.tex

\chapter{公式}



\section{公式引用}

\begin{equation}\label{equation:test1}
  K \leq 
  \begin{dcases}
    2 \lfloor \frac{d}{2d - n} \rfloor    , & K \text{为偶数} ; \\
    2 \lfloor \frac{d}{2d - n} \rfloor - 1, & K \text{为奇数}. \\
  \end{dcases}
  % \left\{ 
  %   \begin{array}{cl}
  %     2 \lfloor \frac{d}{2d - n} \rfloor    , & K \text{为偶数} ; \\
  %     2 \lfloor \frac{d}{2d - n} \rfloor - 1, & K \text{为奇数}. \\
  %   \end{array}
  % \right. 
\end{equation}

公式~\eqref{equation:test1} 的引用
\clearpage
\chapter{研究结论}


\section{复习课课堂提问设计原则}

常言道:“温故而知新。”好的复习,可以让学生的学习事半功倍。

在对复习课教学录像分析的基础上,对复习课的课堂提问设计给出以下几点建议:


\subsection{提问层次需合理}

教师在进行复习课的准备时,相比教材的重点与难点,更为重要的是分析学生学情,也就是学生的实际学习水平。在新授课时,相信教师都已研究过教材的重点与难点,并在课上强调过,但是学生的掌握程度究竟如何呢?这是教师在复习课备课时需要解决的问题。教师对学生已有知识以及思维水平有较为准确把握时,才能确立合适的复习课教学目标,从而设计出难度合适的课堂提问。


\subsection{提问方式需灵活}

教师课堂提问应面向全体学生。若教师常常提问个别学生,那么其他少被注意到的学生的积极性便会大打折扣,不利于调动学生参与课堂的积极性。但所有课堂提问都由全体学生一起作答也不是合理的提问方式,因为每个学生都是不同的个体,个体差异性这一影响因素不容忽视,同样的问题对不同的学生而言难度具有显著差异,所以若每个问题都是集体作答,则会大大降低提问的有效性。因此,教师要根据学生以及知识的具体情况,灵活设计课堂提问方式,以促进学生学习积极性,共同进步。


\subsection{提问语言需清晰}

在进行复习这一阶段前,学生已经学习了一系列新知识,但可能是冗杂的,混乱的,不成体系的。此时,若教师在复习课上的提问表达不够清晰明确,可能会对学生本就混乱的知识结果雪上加霜,打击学生学习热情。教师在教学过程中应尽量准确精炼地表达,不要让语言成为学生学习的另一大障碍。



\section{复习课课堂提问设计案例}

课题《数列 $S_n$ 与 $a_n$的关系》

\begin{question}
  我们已经学习过数列的前 $n$ 项和 $S_n$ 和通项公式 $a_n$ 之间的关系,有没有同学可以带领大家复习一遍推导过程?
\end{question}

\begin{designgoal}
  检测学生已有知识水平,复习巩固知识,由 $S_n=a_1+a_2+a_3+ \cdots +a_n$,写出 $S_{n-1}$,通过观察得到$an= 
  \begin{dcases}
    S_1, & n = 1\\
    S_n - S_{n-1}, & n\geq 2
  \end{dcases}
$关系式,感受退位相减的思想方法。
\end{designgoal}

\begin{example}[2020·江西省信丰中学月考]\label{example:江西省信丰中学月考}
  若数列的前 $n$ 项和 $S_{n}=\frac{2}{3} a_{n}+\frac{1}{3}$, 则的通项公式是 $a_{n}= \underline{\hspace*{3em}}$.
\end{example}


\begin{question}
  例~\ref{example:江西省信丰中学月考} 要我们求解的是什么?
\end{question}

\begin{question}
  例~\ref{example:江西省信丰中学月考} 的条件$S_{n}=\frac{2}{3} a_{n}+\frac{1}{3}$是哪两者之间的关系式?
\end{question}


\begin{designgoal}
  带领学生理解题目,区分题目中的条件以及未知量,抓住题眼,培养解题的目标意识。
\end{designgoal}

\begin{question}
  我们如何实现条件到目标的转化?
\end{question}

\begin{designgoal}
  引导学生思考 $S_n$ 与 $a_n$ 的关系,运用知识点解题。
\end{designgoal}

\begin{question}
  为什么转化关系式中要将 $n$ 分为两种情况?
\end{question}

\begin{designgoal}
  让学生明白数学解题需要注意种种细节,此处需要验证 $n=1$ 的情况。
\end{designgoal}

\begin{example}[2019·山西测试]\label{example:山西测试}
  设正项数列 $\left\{a_{n}\right\}$ 的前 $n$ 项和为 $\mathrm{S}_{n}$, 且 $a_{n}^{2}+1=2 a_{n} S_{n}$, 求数列 $\left\{a_{n}\right\}$ 的通项公式.
\end{example}

\begin{question}
  例~\ref{example:山西测试} 与例~\ref{example:江西省信丰中学月考} 有没有相同之处?
\end{question}

\begin{designgoal}
  引导学生主动分析题目的条件以及未知量,并与例~\ref{example:江西省信丰中学月考} 进行对比,熟悉辅助题目的引入。思考辅助题目,可能有助于学生解决其他题目,提升学生类比分析的思想方法。
\end{designgoal}

\begin{question}
  我们能否运用相似方法解决问题?
\end{question}

\begin{designgoal}
  构建知识缺口。学生动笔解题后便会发现这道例题两个式子无法作差相减,而根据上一题的解题经验,本该可以顺利退位相减解题,如此一来,便可以营造知识缺口,诱导学生形成内在的学习动机以及情感,从而启发学生进行缺口的填补。
\end{designgoal}

\begin{question}
  在 $S_{n}$ 与 $a_{n}$ 的相互转化中, 如果无法完成将 $S_{n}$ 转化为 $a_{n}$ 这个过程, 那么我们能不能反过来,将 $a_{n}$ 转化为 $S_n$ 呢?
\end{question}

\begin{designgoal}
  打破思维定势,启发学生意识到式子的转化可以是双向的。
\end{designgoal}


\begin{question}
  例~\ref{example:山西测试} 与例~\ref{example:江西省信丰中学月考} 条件都是 $S_{n}$ 与 $a_{n}$ 的关系式, 解题目标都是 $a_{n}$, 为什么不能使用同样的方法解题呢?区别在哪里?
\end{question}

\begin{designgoal}
  例~\ref{example:江西省信丰中学月考} 中 $S_n$ 前面系数为 1 , 可以顺利退位相减, 但是例~\ref{example:山西测试} 中 $S_n$ 与 $a_n$ 相乘, 无法退位相减。让学生透过现象看本质, 明白有时不能直接求出 $a_{n}$, 需要 先求解 $S_n$, 再求解 $a_n$, 但万变不离其宗, 本质都是利用 $S_n$ 与 $a_n$ 的关系进行转化 构造。
\end{designgoal}

\begin{question}
  同学们能否尝试对比归纳以上两种解题方法的异同?有什么注意点?
\end{question}

\begin{designgoal}
  复习课的课堂中,随时小结,可以加深学生对解题方法的印象。
\end{designgoal}

相同点: 都适用于条件为 $S_{n}$ 与 $a_{n}$ 的关系式的题目。

不同点: 当 $S_{n}$ 的形式相对独立时, 类比写出 $S_{n-1}$, 作差, 得出 $a_{n}$ 的相关等 式, 进而求解。当 $S_{n}$ 的形式不独立时, 将 $a_{n}$ 转化为 $S_{n}-S_{n-1}(n \geq 2)$, 将题目条件 转变为 $S_n$ 与 $S_{n-1}$ 的关系式, 继而得到 $S_n$ 的相关结论, 从而求解 $a_n$ 。

注意点: 需要验证 $n=1$ 的情况。

引导学生从什么时候用、怎么用、有什么注意事项三个方面总结归纳解题方 法, 形成有序的知识框架。


\begin{example}[2020·尤溪县第五中学高一期末]\label{example:尤溪县第五中学高一期末}
  已知数列 $\left\{a_{n}\right\}$ 满足 $2 a_{1}+2^{2} a_{2}+2^{3} a_{3}+\cdots+2^{n} a_{n}=4^{n}-1$, 则 $a_{n}$ 的通项公式 \underline{\hspace*{3em}}.
\end{example}

\begin{question}
  题目条件“$2 a_{1}+2^{2} a_{2}+2^{3} a_{3}+\cdots+2^{n} a_{n}=4^{n}-1$”,仔细观察等式左边,曾经有没有见过类似形式?
\end{question}

\begin{designgoal}
  引导学生观察并回想错位相减法求前 $n$ 项和的形式,积极调动已有的知识。
\end{designgoal}

\begin{question}
  那么我们能不能重新叙述这道题目?
\end{question}

\begin{designgoal}
  设计意图 令 $b_n=2^{n} a_n$, 则题目可以改写为 “已知数列 $\left\{b_{n}\right\}$ 的前 $n$ 项和 $S_{n}=4 n-1$, 求解数列的 $\{a_{n}\}$ 通项公式.”。引导学生及时引人辅助符号, 简化题目, 由难转易。
\end{designgoal}

\begin{question}
  现在我们已知数列 $\{b_{n}\}$ 的前 $n$ 项和, 你能从已知数据中得出什么?
\end{question}

\begin{question}
  得出数列 $\{b_{n}\}$ 的通项公式后, 我们能否由 $b_{n}$ 求解 $a_{n}$ ?
\end{question}

\begin{designgoal}
  我们可以将解题目标看作对岸,而已知数据和未知量之间就是一条鸿沟,要想跨越鸿沟到达对岸,就必须建造一座桥梁,要引导学生主动分析推导已知数据。
\end{designgoal}

\begin{question}
  有没有同学可以分享一下你从例~\ref{example:尤溪县第五中学高一期末} 学习到了什么?
\end{question}

\begin{designgoal}
  引导学生总结题目特征,加深印象。
\end{designgoal}

\begin{question}
  那么例~\ref{example:尤溪县第五中学高一期末} 与前两个例题本质上是否相同?
\end{question}

\begin{designgoal}
  都是利用 $S_n$ 与 $a_n$ 的关系求解的题型,引导学生把握题目的本质特征。
\end{designgoal}

\begin{question}
  请同学们总结一下本节课最大的收获是什么?
\end{question}

\begin{designgoal}
  由学生进行总结,促使学生回顾整堂课的学习内容,从更高层次把握学习内容。
\end{designgoal}



\section{研究不足}

由于疫情,笔者只能选取一师一优课网站上的教学录像视频进行分析,但是教学录像视频与实际课堂也许还有一定的差距。如果能够深入实际课堂,长期记录并统计某几位老师的复习课课堂提问,相信能够得到更加准确且更有分析意义的数据。



% \backmatter开启后置部分,包含参考文献、致谢、附录等
\backmatter


%%%% 参考文献 %%%%
%%%%%%%%%%%%%%%%%%%%%%%%%%
%%% 行内引用:\parencite{} or \parencite[]{},下面两个情况要用行内引用
%    - 去掉这个引用句子结构不完整,比如“定理证明可参看[1]”
%    - 英文文献的引用
%%% 上标引用:\cite{} or \cite[]{},下面情况要用上标引用
%    - 去掉这个引用,句子结构完整,比如“作者提到,‘CCNUthesis真是一个好模版’^[1]”
%      其中“^[1]” 表示上标引用
%%%%%%%%%%%%%%%%%%%%%%%%%%
% 打印参考文献列表
\printbibliography



%%%% 致谢 %%%%
% \chapter*{\centerline{致 \quad 谢}}
\addcontentsline{toc}{chapter}{\normalfont 致谢}   % 根据旧模版样式,如果有写致谢,则需要把致谢放进目录


感谢各位的使用,欢迎提出issue和bug!



%%%% 附录 %%%%
% 没有附录内容的把下面的代码注释掉即可
% \chapter*{附录}
\addcontentsline{toc}{chapter}{\normalfont 附录}

此处可以写调查问卷,访谈记录等

用\verb|xchoices{}|环境可以排版任意个选项,用\verb|\item|分隔即可(现将代码注释掉,如有需要取消注释即可,如果不需要,可自行删除或不管(因为不影响编译效果)

% \begin{xchoices}
%   \item 选项1
%   \item 选项2
%   \item 选项3
%   \item 选项4
% \end{xchoices}

% \begin{xchoices}[
%   items = 2,            % 手动控制每行多少个选项
%   label-style = alph,   % label的样式:alph, Alph, roman, Roman, 
%                         %            quan(带圈数字), chinese, none
%   pre-label = {(},      % label前面的内容
%   post-label = {)},     % label后面的内容
% ]
%   \item 选项1
%   \item 选项2
%   \item 选项3
%   \item 选项4
% \end{xchoices}

% \begin{xchoices}[label-style = quan]
%   \item 选项1
%   \item 选项2
%   \item 选项3
%   \item 选项4
% \end{xchoices}

\end{document}
